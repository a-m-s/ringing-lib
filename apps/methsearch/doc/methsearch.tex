\documentclass[a4paper,11pt,oneside]{book}

%\usepackage{lmodern}

% We're using xelatex: configure for unicode fonts
% (The cm-default option to fontspec is needed to render math symbols
% in certain versions of fontspec.)
\usepackage{xunicode}
\usepackage[cm-default]{fontspec}
\usepackage{xltxtra}

% Computer Modern warps my fragile little eyes
\setromanfont[Mapping=tex-text]{Linux Libertine O}
\setsansfont[Mapping=tex-text]{Linux Biolinum O}


\chardef\vapost="0027

% Sans-serif chapter headings
\usepackage{sectsty}
\allsectionsfont{\rm\sffamily} % \rm disables bold

% Page header showing chapter
\usepackage{fancyhdr}
\pagestyle{fancy}
\fancyhead{}\fancyfoot{} % Clear
\renewcommand{\chaptermark}[1]{\markboth{#1}{}}
\fancyhead[L]{\nouppercase{\leftmark}}
\fancyhead[R]{\thepage}
\addtolength{\headheight}{2pt} % Default is based on 10pt, not 11pt

% Better verbtaim text
\usepackage{fancyvrb}
\RecustomVerbatimEnvironment{Verbatim}{Verbatim}
  {xleftmargin=2em,fontsize=\small}
\VerbatimFootnotes

\usepackage{amsmath}  % for tfrac
\usepackage{upquote}

% URLs
\usepackage{url}
\DeclareUrlCommand\url{\def\UrlLeft{<}\def\UrlRight{>}\urlstyle{tt}}

% We want an index
\usepackage{makeidx}
\usepackage{index}
\makeindex

% Tables with lines spanning multiple rows
\usepackage{tabularx}

\usepackage{footnote}

% Don't use US-style dates
\usepackage[level,nodayofweek]{datetime}

\usepackage{rotating}
\usepackage{multirow}
\usepackage[small]{caption}


% Coloured links, etc.
\usepackage[xetex,%
            citecolor=black,filecolor=black,linkcolor=black,urlcolor=black,%
            pdfauthor={Richard A. Smith},%
            pdftitle={A user's guide to methsearch and associated tools},%
            bookmarks,colorlinks,hyperfootnotes,hyperindex]{hyperref}

% Code to make indexes how we want them...
\newdimen\optwidth\newdimen\loptwidth\newdimen\fspecwidth
\setbox0=\hbox{\footnotesize\verb+-+}\optwidth=\wd0
\setbox0=\hbox{\footnotesize\verb+--+}\loptwidth=\wd0
\setbox0=\hbox{\footnotesize\verb+$+}\fspecwidth=\wd0
% \textitidx -- a definition formated in italics and indexed
\def\textitidx#1{\textit{#1}\index{#1}}
% \oid  -- option index defintion -- the main definition of a short option 
% \oi   -- option index -- a secondary reference to a short option
% \loid -- long option index definition -- the main definition for a long opt
% \loi  -- long option index -- a secondary reference to a long option
\newcommand{\ulink}[1]{\underline{\hyperpage{#1}}}
\newcommand{\oidx}[2]{\index{#1@{\hspace*{-\optwidth}\texttt{-}#2}|ulink}}
\newcommand{\oid}[2]{\oidx{#1}{\texttt{#1}}%
  \index{#2@{\hspace*{-\loptwidth}\texttt{--#2}}|see{\texttt{-#1}}}}
\newcommand{\oidF}[2]{\oidx{F#1}{\texttt{F#1}}%
  \index{F#2@{\hspace*{-\optwidth}\texttt{-F#2}}|see{\texttt{-F#1}}}}
\newcommand{\oidM}[1]{\oidx{M#1}{\texttt{M#1}}}
\newcommand{\oi}[1]{\index{#1@{\hspace*{-\optwidth}\texttt{-}\texttt{#1}}}}
\newcommand{\loid}[1]{\index{#1@{\hspace*{-\loptwidth}\texttt{--#1}}|ulink}}
\newcommand{\loi}[1]{\index{#1@{\hspace*{-\loptwidth}\texttt{--#1}}}}
\newcommand{\optdesc}[2]{\index{#2|see{\texttt{-#1}}}}
% \fspec[d] -- format specifier [definition]
\newcommand{\fspecd}[1]{\index{#1@{\hspace*{-\fspecwidth}\texttt{\$#1}}|ulink}}
\newcommand{\fspec}[1]{\index{#1@{\hspace*{-\fspecwidth}\texttt{\$#1}}}}
% symbol index
\newcommand{\symidx}[2]{} %{\index{{#1}@{\texttt{#1}}|see{#2}}}
\newcommand{\ttcmdidx}[1]{\texttt{#1}\index{#1@{\texttt{#1}}}}

% \methsearch -- the program name
\def\methsearch{\texttt{meth\-search}}
% \sref -- section ref
\newcommand{\sref}[1]{\hyperref[#1]{\S\ref{#1}}}

\def\half{\tfrac{1}{2}}
\def\quarter{\tfrac{1}{4}}

\makeatletter
\def\printidx#1{\@print@index[default][#1]}
\makeatother

% And now for something different...
\title{A user's guide to \methsearch\ and associated tools}
\author{Richard A.\ Smith\\\url{richard@ex-parrot.com}}

\begin{document}

\frontmatter
\maketitle

\tableofcontents
\addcontentsline{toc}{chapter}{Contents}
% General paragraph style -- this has to be after the table of contents
\parskip=1em
\parindent=0em
\addtolength{\footnotesep}{4pt}

\mainmatter

\chapter{Getting started}

This chapter gives a brief introduction to \methsearch\ --- what it is,
what it can do, and who it's aimed at (\sref{what_is_it}).  
This is followed by an introduction to the command line (\sref{cli}),
aimed particularly at those unfamiliar with it; it can safely be
skipped by those familiar with Unix-style command line interfaces.  
Finally, are 
a slightly non-trivial worked example (\sref{eg_doubles}),
details of how to obtain \methsearch\ (\sref{obtain}), and its
current development status (\sref{dev_status}).


\section{What is it?}\label{what_is_it}

\methsearch\ is a tool for searching for bell-ringing methods.
If you're not a bell ringer or don't know what a method is, 
\methsearch\ almost certainly isn't for you.
Even if you are a ringer and do know what a method is, unless you have 
an active interest in method construction or composition, 
this tool is unlikely to be of interest.

\methsearch\ is designed to exhaustively search for methods satisfying
some particular criteria.  For example, the `standard' 41 surprise minor
methods are the only methods that satisfy a particular set of criteria.
In this case, it's possible to verify this with pencil and paper, and 
indeed this was first discovered a hundred or more years ago.  But 
\methsearch\ allows you to verify this in a fraction of a second.

A comparable list of major methods runs to tens of millions of entries ---
clearly beyond the scope of manual analysis, but on a reasonably 
fast computer, \methsearch\ can enumerate them in a few hours and provide 
you with statistical information on them.

Perhaps you're looking for an interesting new method to ring?  
\methsearch\ can help here too, but first you need an idea of what you want.
There is no magic option that says `find me the best possible method'.
Computers a good at answering quantitative questions, not qualitative ones.
Only after `best' has been defined precisely enough that a computer 
can understand it can \methsearch\ help you find your method.

\methsearch\ can also be helpful in finding methods to drop into a 
composition.  Want to replace Stonebow in Hull's 23 spliced major 
or Ariel in Pipe's cyclic 6 spliced maximus?  Maybe \methsearch\ can help,
but to find a good replacement you'll need a good understanding of the problem.

Above all, \methsearch\ is a program designed to help clever 
people do sophisticated things.  If you have little or no understanding of
method construction or composition, it's unlikely to allow you to find
a killer method or composition overnight.

\section{Command line interface}\label{cli}\index{command line interface}

\methsearch\ is a command line utility.  
What does that mean?  Well, for one thing, it
means that there is no whizzy graphical user interface.
Instead it is controlled by a sequence of often-cryptic commands
and options typed at your system's \textitidx{command prompt} or 
\textitidx{terminal}.  To some users this will be a totally alien concept.  

\subsection{Using \methsearch\ on Windows}\index{Windows}

It is rare for most Windows users to encounter tools that require the use 
of the command line interface, however Windows does provide a terminal 
that allows you to use \methsearch.  To start the terminal, on the 
`Start' menu, choose `Run' and then type in `\ttcmdidx{cmd.exe}'.%
\footnote{On Windows 95, 98 or ME, this will be `\texttt{command.com}'.%
\index{command.com@\texttt{command.com}}}
You should then see a new window, probably black with a white font, 
containing something like the following.  (The exact details will 
depend on your version of Windows.)

\begin{Verbatim}
Microsoft Windows [Version 6.0.6000]
Copyright (c) 2006 Microsoft Corporation.  All rights reserved.

C:\Users\Richard>
\end{Verbatim}

The last line is called the prompt and indicates
that the terminal is waiting for you to type something in; 
this should always end in a `\verb+>+' character. %\symidx{>}{Windows prompt}
First you need to change directory into the directory that contains
your copy of \texttt{methsearch.exe} (and any associated tools and libraries
you may also have got).  If you've put it in a directory called
`\verb+C:\Documents and Settings\Richard\methsearch+', for example,
then you should type:

\begin{Verbatim}
cd C:\Documents and Settings\Richard\methsearch
\end{Verbatim}

In some versions of Windows, you can avoid this by
starting the command prompt in the right
directory by right clicking on the folder and choosing 
`Open Command Prompt Here' on the context menu.
You can now check that \methsearch\ is accessible by typing:

\begin{Verbatim}
methsearch --version
\end{Verbatim}
\oi{V}

After pressing enter, this should print some brief information about the 
version of \methsearch\ that you are running.

\subsection{The Unix philosophy}\label{unixphil}

Although the command line interface originally developed through necessity
--- early computers had no facilities for graphical output --- its use
has continued thanks largely to the \textitidx{Unix philosophy}.  This is 
a set of cultural norms for programs that can be summarised as follows:

\begin{itemize}
\item[---] Write programs that do one thing and do it well.
\item[---] Write programs to work together. 
\item[---] Write programs to handle text streams, 
because that is a universal interface.
\end{itemize}

The result is programs that are powerful and flexible even if this is 
at the cost of being less user-friendly.%
\footnote{Or as one wag put it, ``Unix is user-friendly. 
It just isn't promiscuous about which users it's friendly with.''}
\methsearch\ attempts to conform to these ideals.  As a result,
\methsearch\ is designed to search for methods and filter the results:
it is not designed to search or prove compositions containing those methods,
although for reasons of efficiency it has limited facilities for this.
Other programs exist for searching and proving compositions and
\methsearch\ does not attempt to duplicate their functionality.

\subsection{Command line options}

To control \methsearch's behaviour, you need to specify 
\textit{command line options}\index{command line option|see{options}}%
\symidx{-}{options}\index{options}  which tell it what to look for and how to 
present the information it finds.  For example, to obtain a list of plain
minimus methods with double symmetry,\index{symmetry!glide} you could type:%
\index{example!minimus, plain double}

\begin{Verbatim}
methsearch -b4 -d
\end{Verbatim}

In this example, there are two command line options: `\verb+-b4+' and 
`\verb+-d+'.  The former\oi{b} tells \methsearch\ to look for minimus methods
(i.e.\ methods on four bells), the latter tells it to restrict its search
to double methods.\oi{d}  The \verb+-d+ option is a 
\textit{boolean option}\index{options!boolean}: 
either it's specified or it's not --- it's not parametrised in anyway; 
for example, 
it doesn't make sense to ask for methods that are at least 40\% double.
By contrast, the \verb+-b+ option requires a 
\textit{mandatory argument}\index{options!mandatory argument}: 
namely the `\verb+4+' in this example.  It is not useful
to ask for methods that are on bells without saying how many bells there 
should be.

By convention, command line options can sometimes be merged into one 
argument.  This can be confusing to the uninitiated as it results in a much 
more terse command invocations.
The rule is that any number of boolean options can be concatenated,
followed, optionally, by at most one parametrised options.  For example,
the following \methsearch\ invocation searches for all plain minor methods
that are double, have regular (i.e.\ Plain Bob) lead ends, 
and no adjacent places.\index{example!minor, plain double}

\begin{Verbatim}
methsearch -b6 -drj
\end{Verbatim}

As before, the \verb+-b+ option specifies the number of bells.  The last
part of the command line, `\verb+-drj+', is three separate options merged
into a single argument: specified individually they would be written 
`\verb+-d -r -j+'.  
The \verb+-d+\oi{d} and \verb+-r+\oi{r} options, meaning double and 
regular respectively, are both boolean options.  

The \verb+-j+ option\oi{j} specifies that the method should have no 
adjacent places:\index{places!adjacent} in other words, 
the method place notation must be formed solely from 
the changes \verb+x+, \verb+14+, \verb+16+ and \verb+36+.  
A \verb+12+ change would not be allowed because first and seconds place 
are adjacent.
However, the \verb+-j+ option is more complicated as it has an 
\textit{optional argument}\index{options!optional argument}.  
It is also possible to require no more than 
two adjacent places via \verb+-j2+.  
This allows the \verb+12+ or \verb+1256+ changes, but not \verb+1236+.
Similarly, \verb+-j3+ allows \verb+1236+ but not \verb+1234+.  
Because the \verb+-j+ option has an argument, albeit an optional one, 
if it is one of several options merged together on the command line, it 
must be last.  Thus `\verb+-drj+' is equivalent to `\verb+-rdj+', but
specifying `\verb+-djr+' would result in an error as \verb+-j+ is not
a boolean option.

Each command line option also has a 
\textit{long form}\index{options!long form}.  These all begin
with two hyphens rather than one, and have long names that are intended to 
make their meaning more obvious to a casual reader instead of the 
normal single-character ones.  The other difference is that when they
have arguments, the argument is preceded by an `\verb+=+'.  For example,
`\verb+-d+' can be written `\verb+--double+', 
and `\verb+-b4+' can be written `\verb+--bells=4+'.  
Certain rarely-used options only have a long form: for example,
the option to require regular (i.e.\ Plain Bob) half-leads is 
\verb+--regular-hls+ --- there is no equivalent short form.
Perhaps unsurprisingly, it is not possible to merge together the long forms 
of arguments in a way analogous to \verb+-drj+.


\subsection{Output and redirection}\index{shell!redirection|(}
\index{redirection|see{shell, redirection}}\label{redirection}

By default, \methsearch\ prints the place notation and lead head 
of each matching method on the screen as it finds them.   When very few
methods are found, this is satisfactory as the results can be 
inspected manually.  The two preceding examples were both designed to
yield very few matching methods (three and one, respectively);
for example, the result from the first example (plain minimus methods 
with double symmetry) was:

\begin{Verbatim}
X.14.X.34.X.14.X.12       1423
34.14.12.14.12.14.34.14   1423
34.14.12.34.12.14.34.12   1342
\end{Verbatim}
\index{output}

A quick inspection confirms these to be Double Bob, Double Court and
Double Canterbury as expected.  

Should we need these results again, the search only took a fraction of a 
second to complete and so is easy to repeat.  But what if the search produces
lots of results or takes a long time to complete?  Fortunately, it is 
possible to \textit{redirect} output to a file where it can be examined 
in detail.\symidx{>}{shell, redirection}  This is handled by the terminal,%
\footnote{Actually, it's handled by the \textitidx{shell}, but this distinction
is not important here.}
and in all popular terminals (Unix or Windows) has the syntax:

\begin{Verbatim}
methsearch -b6 -sd > minor.txt
\end{Verbatim}
\index{example!minor, plain double}

In this example we are looking for plain minor methods with both palindromic
and double symmetry (\verb+-s+ and \verb+-d+ respectively)\oi{s}\oi{d}%
\index{symmetry!palindromic}\index{symmetry!glide}
and saving the list of methods into a file called
`\verb+minor.txt+'.  This file can then be opened and studied in your 
favourite text editor (for example, Notepad on Windows).%
\index{shell!redirection|)} 
Another strategy
is to use a \textitidx{pager} such as \ttcmdidx{more}
to allow you to scroll through the output
without saving it.%
\footnote{\verb+more+ only allows you to scroll forwards through the output,
however it has the advantage of being present on most systems, including
Windows.  Other pagers, such as \ttcmdidx{less}
allow you to scroll backwards too.
\verb+less+ is present on many Unix systems, and a Windows version can be
downloaded from \url{http://www.greenwoodsoftware.com/less/}.}  
The usual syntax for this is with a \textitidx{pipe}\symidx{"|}{pipe} 
which feeds the output of one command (in this case \methsearch) 
to the next command (the pager, \verb+more+):

\begin{Verbatim}
methsearch -b6 -sd | more
\end{Verbatim}

In most pagers, pressing either space or return will scroll down a page.

\subsection{Help!}\label{help}

As with most command line utilities, you can obtain a brief summary 
of valid options to \methsearch\ by using the \verb+--help+ option.%

\begin{Verbatim}
methsearch --help
\end{Verbatim}
\loi{help}

This user's guide aims to provide more extensive documentation as 
well as some worked examples of how it can be used in real world situations.
Don't be afraid to experiment.  If you're unlucky, you might
ask \methsearch\ to perform a search that will never complete.  
The \verb+-u+ option provides a good way of monitoring how quickly 
the search is progressing.  If the search looks like it is going to take
too long, press Ctrl-C to abort it, and try again with additional constraints.
Finally, if you're stuck, drop me an email:
my email address is on the title page.

\section{An example — doubles methods}\label{eg_doubles}
\index{example!doubles, plain|(}

The Central Council's 1980 \textit{Doubles Collection}%
\index{Central Council!Doubles Collection@\textit{Doubles Collection}}
lists 177 methods with one plain hunt bell.  
In its words, this is an exhaustive list of 
\begin{quote}
\ldots all 177 methods that have one plain hunt bell and three or four leads in
the plain course, which are symmetrical about the treble's lead and lie,
and which are not bobbed leads of other methods.
\end{quote}
Producing such a list is exactly the sort of task for which 
\methsearch\ is ideally suited, but first we need to translate the problem
into a form that methsearch can understand.  On a Windows system, we would
type:
\begin{Verbatim}[commentchar=~]
methsearch -b5 -sAC -R"$q\t$D" -Q"$o>=3" -m"*(1|2|23)"
~ -- a comment to get vim's syntax highlighter back in sync:  $
\end{Verbatim}
And on most Unix systems (including most Linux shells), 
the double quotes around the \verb+-R+ and \verb+-Q+ options' arguments 
\textit{must} be replaced with single quotes;%
\footnote{This is because the shell 
\textit{interpolates}\index{shell!interpolation} text in double quotes,
but not in single quotes,  and will therefore interpret 
\verb+$q+, \verb+$D+ and \verb+$o+ as a shell or environment variable, and 
run \methsearch\ with the value of this variable (probably nothing) instead.
If you see an error message saying 
\texttt{Binary operator ">=" needs first argument}
this is likely the cause.}
it would be good practice to do it with \verb+-m+ option too:
\begin{Verbatim}[commentchar=~]
methsearch -b5 -sAC -R'$q\t$D' -Q'$o>=3' -m'*(1|2|23)'
\end{Verbatim}
Note that the \verb+'+ character in many of the examples here is the character
you get by pressing the apostrophe key on your keyboard.  Some systems will
display it as a straight (symmetrical) quote, others as a curved close quote.

Here, \verb+-b5+ means doubles (i.e.\ on five bells), \verb+-s+ means 
symmetric about the treble's lead and lie, \verb+-A+ means include methods
with fewer than four leads in the plain course,\oi{A} 
and \verb+-C+ means print the total number of methods found.\oi{C}
The remaining options are more interesting.  In both cases, their argument
is enclosed in quotes;  this is because otherwise the shell would interpret
the \verb+>+ and \verb+|+ characters as redirection and a pipe as described
in \sref{redirection}.\index{shell!quotation}.

The \verb+-R"$q\t$D"+\oi{R} option controls what information is printed 
about each method.  
The argument to the \verb+-R+ option is called a \textitidx{format string} 
and is described in detail in \sref{fmtstr}.
In this case, we request the place notation in a 
concise notation (\verb+$q+)\fspec{q} and the lead head code using the 
traditional codes used in the 1980 collection (\verb+$D+)\fspec{D}.  
The remaining part of the argument, \verb+\t+, instructs \methsearch\ to 
place a \textitidx{tab character} between the two pieces of information.
Most terminals display this as some (varying) number of spaces; however, 
it has the advantage that, if the output is saved to a file, it can 
be loaded in a spreadsheet\index{spreadsheet} as 
\textit{tab separated values} and the two values will appear in 
separate spreadsheet columns.  

The \verb+-Q"$o>=3"+ option states that the number of leads in
the plain course (\verb+$o+)\fspec{o} must be at least three.  
(Strictly speaking, the rubric stated ``three or four leads in the 
plain course'' which we would write \verb+-Q"$o==3 || $o==4"+, 
but as plain courses with more than four leads are not possible, 
this is unnecessarily pedantic.)  The argument to the \verb+-Q+
option is called an \textitidx{expression} and is described in detail
in \sref{expr}.

The final \verb+-m"*(1|2|23)"+\oi{m} specifies that the last change
in the method must be a \verb+1+, \verb+125+ or \verb+123+ change.
The syntax to this option is documented in \sref{pn}.
This is because the only other possibility, a \verb+145+ change, corresponds
to a bobbed lead of some other method.  (In fact, the possibility exists that
some method with a \verb+145+ lead end might exist that is not the bobbed
lead of some other valid method, for example because the other three lead
end variants have some property that we consider to be undesirable.)

If we run the command, we expect to see a list of 177 methods. 
And \ldots we don't:

% Warning: this table contains tabs
\begin{Verbatim}[obeytabs=true,commandchars=\\\{\}]
&3.1.2.1.2,1	E
&3.1.2.1.2,23	T
&3.1.2.1.2,2	K
&3.1.2.1.4,1	D
	\vdots
&5.4.5.3.34,23	U
&5.4.5.3.34,2	J
&5.4.5.3.5,23	R
&5.4.5.3.5,2	B

Found 258 methods
\end{Verbatim}

Why are there not 177 methods?  The clue is in the very first one,
\verb+&3.1.2.1.2,1+ --- consider the path of the second.  It starts 
by doing eight blows at lead.  However, the Central Council don't allow
such methods: ``no bell shall make more than 
four consecutive blows in the same position in a plain course''.%
\footnote{The Central Council decisions can found at 
\url{http://www.cccbr.org.uk/decisions/}.\index{Central Council!decisions}
The quoted text is in decision (E)A.1(f).}
Adding a \verb+-p4+ option\oi{p} to the command line takes this into account
and produces the desired list of 177 methods.  
When using \methsearch, it is not uncommon to discover that it has not
accounted for some such implicit assumption about the desired method, 
and it therefore finds too many methods.

% methsearch -b5 -p4 -sCA -Q'$o>=3 && ( $10h ne "145" || $(methsearch -b5 -p4 -sqA -Q"\\$$o>=3" --raw-count --limit=1 -m"$Q,(1|123|125$)") == 0 )' -R'$q\t$D'
\index{example!doubles, plain|)}

\section{Obtaining \methsearch}\label{obtain}

One way to obtain \methsearch\ is to download the source code and
compile it yourself.  Such things are easy under Linux where most 
distributions will provide the necessary tools as standard.  However,
Windows users will likely find this process harder.  To compound matters,
whilst the source code for \methsearch\ should be portable to a whole 
range of different compilers and operating systems, life is rarely that 
straightforward, and in all likelihood if you're not using a combination
that I've tested, you'll end up having to fix a handful of small problems.

I used to provide Windows binaries compiled with Microsoft's 
Visual C++ compiler\index{Microsoft Visual C++} which produced excellent
binaries.  Unfortunately, since 2007 I've no longer had access to a Windows
machine on which to compile these.  Instead, I now compile Windows binaries
on a Linux machine using a \textitidx{cross compiler} with less satisfactory 
results.  What's    
really needed is for someone with some programming knowledge and a Windows      
machine to take an interest in the project and (with my assistance, 
if necessary) provide Windows binaries.

Meantime, the current Windows 
binary of \methsearch, the Windows binaries of the other associated tools 
(see \sref{interop}),%
\index{downloading methsearch@downloading \methsearch}\index{Windows!binaries}%
\index{manual, downloading}
and the PDF version of this manual can be found at the following locations:

{\parsep=0em\begin{itemize}\footnotesize\itemsep=0em
\item \url{http://ex-parrot.com/~richard/methsearch/methsearch-win32.tar.gz}
\item \url{http://ex-parrot.com/~richard/methsearch/ringingtools-win32.tar.gz}
\item \url{http://ex-parrot.com/~richard/methsearch/methsearch.pdf}
\end{itemize}}

\subsection{Compiling from source on Linux}
\index{compiling methsearch@compiling \methsearch}

To compile from source, it is best to get the current version from CVS.%
\index{CVS} Unlike the main releases, this does not come with a copy of the 
\verb+configure+ script; this can be generated using the \verb+autoreconf+
command.  The package is then configured, compiled and installed in the
standard way:
\begin{Verbatim}
U=:pserver:anonymous:
H=ringing-lib.cvs.sourceforge.net
P=/cvsroot/ringing-lib
export CVSROOT=$U@$H:$P
unset U H P
cvs login
cvs -z3 co -P ringing-lib
unset CVSROOT
cd ringing-lib
autoreconf
./configure --prefix=$HOME
make
make install
\end{Verbatim}

This should have installed the Ringing Class Library%
\index{Ringing Class Library} and all of its
associated programs (including \methsearch) into your home directory.
Depending how your system is configured, you may need to add this
to your \textitidx{path}\index{path} so that the the shell can find
methsearch.  If, when trying to run \methsearch, you see an error message
saying `\texttt{command not found}', you 
probably need to run the following:\footnote{Or add it to your 
\verb+~/.bashrc+ so they are always available.}\fspec{PATH}
\begin{Verbatim}
export PATH=$PATH:$HOME/bin
\end{Verbatim}

\section{Development status}
\label{dev_status}

\methsearch\ is a fairly mature piece of software and has been in active
use since 2002, since which time a great many new features have been added.
Some features will have been tested extensively in many different 
combinations and are likely to be bug-free.  However, it's quite possible 
that some less-commonly used features may interact with each other in odd ways.
If you ever find something which you believe to be a bug,%
\index{bugs, reporting} do please email me.

\methsearch\ is also a work in progress.  
It does not presently not support every possible type of method, 
or allow methods to be filtered in every possible way.
Generally speaking, new functionality gets added as and when I want it
or someone requests it.  For example, at present it has very poor support 
of alliance, hybrid or treble place methods,\index{hybrid methods}%
\index{alliance methods}\index{treble place methods}
and its support for principles\index{principles}
is quite limited.
If there is missing functionality that you would particularly like to use, 
drop me an email and I'll see what I can do about adding it.

Finally, \methsearch\ is open source software.\index{open source}
This means that anyone 
can download a copy of the source code and extend it.%
\footnote{The source code for \methsearch\ is included with the 
Ringing Class Library which can be downloaded from 
\url{http://ringing-lib.sourceforge.net/}.  Note
that releases of the library are made relatively infrequently and if you
want a recent copy, it is best to check the code out from CVS.}
For example, I have no great interest in writing a graphical user interface 
to \methsearch, but if you're a programmer and want to, there is nothing to
prevent you from doing so.  All that is asked in return is that if you
make any extensions publicly available, you also make the source code to them
freely available.%
\footnote{Specifically, \methsearch\ is covered by the GNU General Public 
License, version 2 or above,\index{licence} which can be viewed at
\url{http://www.gnu.org/licenses/gpl.html}.}
I'm always happy to add such contributions to the Ringing Class Library.

\chapter{Command line options}

This chapter contains a complete list of all of \methsearch's command
line options together with documentation on their use and meaning.
For convenience, the options have been divided into eight sections based
on their purpose: specifying the method stage and class (\sref{stage_class}),
applying constraints to places made within the method (\sref{int_struct}),
constraining the lead ends and heads (\sref{le_lh}), and half leads 
(\sref{hls}), requiring specific symmetries (\sref{symmetry}),
partially specifying the method's place notation (\sref{pn}),
configuring \methsearch's output (\sref{output_opt}), sampling methods
randomly (\sref{random}), and other miscellaneous options (\sref{misc_opt}).  
Options can also be stored in response files (\sref{respfile}).

\section{Stage and class}\label{stage_class}

\index{class|(}
\index{stage|see{bells!number of}}

\begin{tabularx}{\textwidth}{llX}
\texttt{-b}&\texttt{--bells=N}&The number of bells\\
\texttt{-G}&\texttt{--treble-dodges=N}&The number of times the 
  treble dodges in each position\\
\texttt{-S}&\texttt{--surprise}&Require a surprise method\\
\texttt{-T}&\texttt{--treble-bob}&Require a treble bob method\\
&\texttt{--delight}&Require a delight method\\
&\texttt{--3rds-place-delight}
  &Require a thirds place delight method\\
&\texttt{--4ths-place-delight}
  &Require a fourths place delight method\\
\texttt{-Z}&\texttt{--treble-path=PATH}&Require the specified treble path\\
\texttt{-U}&\texttt{--hunts=N}&The number of hunt bells\\
\texttt{-n}&\texttt{--changes-per-lead=N}&The lead length\\
\end{tabularx}

The \verb+-b+ option is the only option that is always required.%
\oid{b}{bells}\index{bells!number of}
Its argument is the number of bells and must be an integer between 2 and 33, 
inclusive.  
The lower limit is simply because, trivially, there is only one possible
piece of ringing of any given length on one bell,
and I have not yet got around to coding all of the special cases required 
to make \methsearch\ handle it.
The upper limit is not a fundamental limitation in \methsearch\ --- rather it 
is a limit on what place notations can be displayed.%
\footnote{After 33 bells, the numbers 1--9, the symbols 0, E, T 
and the letters A--Z, excluding E, I, O, T and X have all been used.}

By default, \methsearch\ looks for plain methods.\index{plain methods}
The \verb+-G+ option\oid{G}{treble-dodges}
switches this to treble-dodging methods.\index{treble-dodging methods}
Most commonly this will be
\verb+-G1+ meaning methods where the treble dodges once in every
dodging position.  Higher values can be specified, for example \verb+-G3+
has the treble dodging three times in each position.  \verb+-G0+ is valid
and is equivalent to not specifying a \verb+-G+ option.  Treble-dodging methods
are only supported on even stages.

The \verb+-S+, \verb+-T+ and \verb+--delight+ options%
\oid{S}{surprise}\oid{T}{treble-bob}\loid{delight}
require there to be at internal 
places made at every (for \verb+-S+), none (for \verb+-T+) or 
some but not all (for \verb+--delight+) of the 
occasions when the treble moves between dodging position.  These will
usually be specified together with a \verb+-G1+ option in which case the
effect is to limit the search to surprise, treble bob or delight methods, 
respectively, however they can be applied to even-stage plain methods too.%
\footnote{In particular, an \verb+-SG0+ search, on six bells at least, 
finds methods with a distinctly Double Court feel to them.  This is hardly 
surprising as the Central Council's former court method\index{court methods}
class was close to being to plain as surprise is to treble dodging.}
Both options are only supported on even stages.

The \verb+--3rds-place-delight+ and \verb+--4ths-place-delight+ options
only apply to methods where the treble travels between lead and sixths place.
Typically this will be a treble dodging minor method, but they can be used
on higher stages in conjunction with \verb+-Z-6+ option.  Where these options
are permitted, all delight methods will either be characterised as 3rds or
4ths place delight.

The \verb+-U+ option\oid{U}{hunts} specifies the number of 
coursing hunt bells required.
For example, \verb+-U2+ will find twin-hunt methods.\index{twin-hunt methods}
All of the hunt bells
will be required to follow the same path --- this means that 
Grandsire\index{Grandsire} will be found with an appropriate \verb+-U2+ search,
but a slow course method\index{slow course methods}
(which the Central Council also define as a twin-hunt method, albeit of 
a different type) will not be.  With \verb+-U+$n$, the hunt bells will always 
be the front $n$ bells --- thus, New Grandsire\index{New Grandsire} 
(that is, Grandsire rotated so that 1 and 3 hunt) 
will never be found; nor will such peculiarities as 
Bedfont Slow Course Doubles\index{Bedfont Slow Course Doubles} 
in which 1 and 5 plain hunt.  
The \verb+-U+ option can be used in conjunction with the \verb+-G+ option to
find treble dodging methods with multiple hunt bells.

The \verb+-U+ option can also be used to find principles by specifying 
\verb+-U0+.\index{principles}  
When this is done, the \verb+-n+$n$ option\oid{n}{changes-per-lead} 
must also be used to specify length of the lead unless \methsearch\ is 
running in filter mode in which case the \verb+-n+$n$ option can be omitted and
its value, $n$, inferred from the length of the methods being filtered.%
\optdesc{n}{lead length!specifying}
The combination \verb+-AU0+ is useful when filtering as it avoids the need to
specifying details such as the lead length, number of hunt bells, or treble
path.   Searching for principles is not especially well
optimised in the code, nor is it particularly well tested.

The \verb+-Z+ option is used to specify the path of the treble (or
whichever bell is the primary hunt bell in methods with unusual treble paths).%
\oid{Z}{treble-path}\index{little methods|(}
At present, the only treble paths supported by this option are little versions
of other supported treble paths --- i.e.\ little plain or treble dodging.
The syntax \verb+-Z+$x$\verb+-+$y$ specifies the 
range of places that the treble passes through --- i.e. that the treble hunts
(or treble-dodges if \verb+-G+ is specified) from place $x$ to place $y$.
The places should be specified by bell symbol (see \sref{pn}) --- thus a plain
maximus method is equivalent to \verb+-Z1-T+ and not \verb+-Z1-12+.  
If the lowest place, $x$, is lead, it can be omitted; 
likewise, if the highest place, $y$, is the back, it can be omitted.  
Thus \verb+-Z-4+ is a traditional treble-to-%
fourths-place little method, and \verb+-Z-+ is equivalent to no \verb+-Z+ 
option as it means front to back.

\methsearch\ supports little methods in which the primary hunt bell does not
reach lead.  For example methods such as 
Very Little Bob Minor\index{Very Little Bob Minor} 
(in which the third hunts between thirds place and fourths) 
can be found with \verb+-Z3-4+, and Double Bishopstoke Little Surprise Major%
\index{Double Bishopstoke Little Surprise Major} 
(in which the third treble-dodges between thirds and sixths) 
can be found with \verb+-G1+ \verb+-Z3-6+.

\index{little methods|)}
\index{class|)}
\index{class|seealso{\textit{specific class names}}}

\subsection{Historical classes}\label{hist_class}

\begin{tabularx}{\textwidth}{llX}
&\texttt{--strict-delight}
  &Delight methods (1905--69)\\
&\texttt{--exercise}
  &Exercise methods (1955--69)\\
&\texttt{--strict-exercise}
  &Exercise methods (1905--55)\\
&\texttt{--pas-alla-tria}
  &`Pas-alla-tria' methods (1905--55)\\
&\texttt{--pas-alla-tessera}
  &`Pas-alla-tessera' methods (1905--55)\\
\end{tabularx}

Until 1969, the Central Council recognised a fourth class of treble 
dodging methods: namely exercise methods\index{exercise methods}.  
At that time, a delight method had to have internal places at 
all but one occasion when the treble moves between dodging positions; 
methods with internal places at all but two or fewer occasions 
were classified as exercise methods.  To find methods 
that would formerly have been classified as exercise, or as delight in this
stricter sense, use the \verb+--exercise+\loid{exercise} and 
\verb+--strict-delight+\loid{strict-delight} options.

Similarly, until 1955, the exercise class was restricted to methods with
internal places made at exactly all but two occasions when the treble moved
between dodging positions.  Methods with internal places at all but three 
and all but four occasions, respectively, were nominally called 
`pas-alla-tria' and `pas-alla-tessera',\footnote{These were sometimes spelt
as one word: viz., `pasallatria' and `pasallatessera'.  A variety of 
spellings can be found, including `pasalltria' and `pasallatessara'.
They presumably derive from the Ancient Greek \textit{πας αλλα τρια} and 
\textit{πας αλλα τεσσερα} meaning `all but three' and `all but four'.}
though I'm not aware of any evidence that these class names were actually 
outside the writings of H.~Law James.\index{James, H.~Law}
The \verb+--strict-exercise+\loid{strict-exercise},
\verb+--pas-alla-tria+\loid{pas-alla-tria} and 
\verb+--pas-alla-tessera+\loid{pas-alla-tessera} opt\-ions
can be used to find these types of methods.

\section{Internal structure}\label{int_struct}

\begin{tabularx}{\textwidth}{llX}
\texttt{-p}&\texttt{--blows-per-place[=N]}&
  At most \texttt{N} consecutive blows in one place\\
\texttt{-l}&\texttt{--places-per-change[=N]}&
  At most \texttt{N} places in any change\\
\texttt{-j}&\texttt{--max-adj-places[=N]}&
  At most \texttt{N} mutually adjacent places in any change\\
\texttt{-w}&\texttt{--right-place}&
  Require right place methods\\
\texttt{-f}&\texttt{--no-78s}&
  Disallow penultimate places above the treble\\
\texttt{-y}&\texttt{--symmetric-sections}&
  Require each treble section to be symmetric\\
&\texttt{--long-le-place}&
  Allow an exception to -pN across the lead-end\\
\end{tabularx}

The Central Council requires methods (except minimus methods) to have
no more than four consecutive blows in any one place.  In practice, on six
or more bells, it is common to want to limit this to two consecutive blows.
The \verb+-p+$n$ option\oid{p}{blows-per-place}%
\optdesc{p}{blows, maximum consecutive!specifying}
controls this by limiting the search to methods
with at most $n$ consecutive blows in any one place.  Omitting $n$ 
(i.e.\ just writing \verb+-p+) is equivalent to specifying \verb+-p2+.  
Omitting the option entirely allows an arbitrary number of consecutive blows
in one place.

The \verb+-l+$n$ option\oid{l}{places-per-change}
can be used to restrict \methsearch\ to methods 
involving changes with at most $n$ places.  (Note: this option uses the 
letter `\verb+l+', not the digit `\verb+1+'.) 
For example, on six bells it is common to want to avoid single changes, 
in which case \verb+-l2+ would be specified.  
This means that changes such as \verb+1234+ or \verb+1236+ will be avoided.  
If $n$ is omitted (i.e.\ if just \verb+-l+ is specified), a
value of 2 is assumed.  Omitting the option entirely allows arbitrary 
changes.%
\footnote{Except that the \textit{null change} -- the change where no bells
move -- is still prohibited.  Use \verb+-Fx+ if you want to allow this.  
See \sref{basic_false} for details.}

The \verb+-j+$n$ option\oid{j}{max-adj-places}\index{places!adjacent}
is similar to the \verb+-l+$n$ option.  It 
limits the number of adjacent places permitted to $n$.  Thus, on six bells,
with \verb+-j1+, the method place notation must be formed solely from
the changes \verb+x+, \verb+14+, \verb+16+ and \verb+36+.
A \verb+12+ change would not be allowed because first and seconds place
are adjacent.  Specifying \verb+-j2+ would allow changes such as 
\verb+12+ or \verb+1256+ changes, but not \verb+1236+ (as this has three
adjacent places). Similarly, \verb+-j3+ allows \verb+1236+ but not \verb+1234+.
Omitting $n$ is equivalent to specifying \verb+-j1+.  Note that the \verb+-j+
option applies to the lead-end and half-lead changes too --- this means that
no seconds place methods will ever be listed.

The \verb+-w+ option\oid{w}{right-place} restricts the search to 
right place methods.\index{wrong place|see{\texttt{-w}}}
On even stages the meaning of this is obvious --- every other change,
starting with the first one, must be a \verb+x+.  On odd stages, this is
taken to mean no places below the treble in every other change starting 
with the first, and no places above the treble in every other change starting
with the second.  That might sound counter-intuitive when written down,
but it yields the expected results.  (For example, it makes Plain Bob Doubles
right place but not Reverse Canterbury.)

The \verb+-f+ option\oid{f}{no-78s}\index{places!penultimate}
is used to exclude any methods with penultimate places
made above the treble.  For example, in a major method this excludes all
\verb+78+ changes (and related changes such as \verb+1478+) except at the 
half-lead.  Despite the name of the long form of this option, \verb+--no-78s+,
this option is not specific to eight bells.  Penultimate places are often
disliked because it makes it hard to get a touch, and impossible to get
an extent, which doesn't involve the two tenors sounding backwards at the 
end of the row at backstroke.  
Traditionally, this is considered musically poor; however this
is perhaps gradually changing.  (Note that there is no guarantee even with
\verb+-f+ that such rows can be avoided at backstroke in an extent: Carlisle
Surprise Minor\index{Carlisle Surprise Minor} is an example of a method 
where \verb+65+s at backstroke cannot be avoided.)

The \verb+-y+ option\oid{y}{symmetric-sections} limits the search to
methods which have symmetric work during each of the treble's dodges.  
This has no effect in a plain method, but in a treble dodging method
(with one treble dodge per position), it limits each section to those
of the form \verb+a.b.a.c+.  For example, a method starting \verb+3-3.4+
will be permitted, but one starting \verb+34-3.4+ will not.  This option
can be useful in reducing an otherwise unmanageable search to one of 
a feasible length.

The \verb+--long-le-place+ option prevents checking of the \verb+-p+$n$ 
option across the lead-end symmetry point.  This allows up to $2n$ 
consecutive blows in one place across the lead-end, whilst prohibiting more
than $n$ blows anywhere else.  Thus, Grandsire Minor\index{Grandsire Minor}
is found with \verb+-b6 -U2 -p2 --long-le-place+, though Reverse Grandsire
is not.

To a limited degree, the \verb+-m+ option (\sref{pn}) allows exception
to be made restrictions imposed by the \verb+-l+, \verb+-j+, \verb+-w+ and
\verb+f+ options.

\section{Lead ends and lead heads}\label{le_lh}
\index{lead end|(}\index{lead head|(}

\begin{tabularx}{\textwidth}{llX}
\texttt{-r}&\texttt{--regular}&Require regular (Plain Bob) lead heads\\
\texttt{-c}&\texttt{--cyclic}&Require cyclic lead heads\\
\texttt{-e}&\texttt{--restricted-le}&
  Require \texttt{12}, \texttt{1N}, \texttt{1} or \texttt{12N} 
  lead end changes\\
\texttt{-E}&\texttt{--prefer-restricted-le}&
  Prefer \texttt{12}, \texttt{1N}, \texttt{1} or \texttt{12N} 
  lead end changes\\
\texttt{-A}&\texttt{--all-methods}&
  Allow all lead heads, including differentials or differential hunters,
  slow course methods, and other methods with multiple hunt bells\\
&\texttt{--offset-cyclic}&
  Require a method that produces cyclic music when started from
  treble's snap\\
\end{tabularx}

The term \textit{lead end} is often used ambiguously to refer to 
(i) the row at handstroke when the treble is leading full, 
(ii) the following row, (iii) the change between these two rows.  
In this documentation, the \textit{lead end} refers is used exclusively for 
the first, \textit{lead head} for the second, 
and \textit{lead end change} for the third.

The \verb+-r+ option\oid{r}{regular}%
\index{Plain Bob lead ends|seealso{\texttt{-r}}}
tells \methsearch\ to only look at methods which have
regular lead heads.  For a \verb+-U+$n$ search this means that the lead head
(the backstroke when the treble is leading) must be a row that appears at
backstroke when the back $n$ bells hunt.  For single- and twin-hunt methods
this means a Plain Bob or Grandsire lead head, respectively.  Note that, 
unless \verb+-A+ is additionally specified, regular differential lead heads
are not permitted.  Thus, royal methods with lead head \verb+1860492735+ 
will be found with \verb+-Ar+ but not with just \verb+-r+.

The \verb+-c+ option\oid{c}{cyclic} 
is similar except that it requires cyclic lead heads
--- that is, a lead head where the working bells form a wrap of rounds.  As
with \verb+-r+, differentials are not permitted unless \verb+-A+ is 
additionally specified, so a cyclic royal method with lead head 
\verb+1567890234+ will be found with \verb+-Ac+ but not with just \verb+-c+.
When used in conjunction with the \verb+-k+ option (specifying rotational 
symmetry), \methsearch\ is able to reduce the search space considerably
for single-hunt cyclic methods by knowing that half-lead must also be cyclic.
The \verb+-c+ option may be specified when searching for principles with 
\verb+-U0+.

The \verb+-e+ option\oid{e}{restricted-le} 
limits the search to methods with a 
\texttt{12}, \texttt{1N}, \texttt{1} or \texttt{12N} lead end change 
(the former two on even stages, the latter two on odd stages).  In the most
common case of regular methods with palindromic symmetry 
(i.e.\ when \verb+-rs+ is specified), this is redundant.

The \verb+-E+ option\oid{E}{prefer-restricted-le} 
is similar to \verb+-e+ except that methods with 
unorthodox lead end changes are only permitted if neither of the 
orthodox lead end changes produce acceptable methods.  
For example, if the seconds
place variant would produce a one lead course (disallowed without \verb+-A+)
and the $n$th place variant would produce four blows behind (disallowed with
\verb+-p2+), a fourth place variant would be considered.%
\footnote{The primary motivation for this option is to allow the 
lists of methods from the Central Council's 
\textit{Plain Minor Collection} and \textit{Treble Dodging Minor Methods}
to be reproduced.  These collections distinguish fourths place methods that 
are bob leads of other methods from other fourths place methods.  
The \verb+-E+ option allows \methsearch\ to make this distinction too.}
In determining whether the seconds and $n$ths place variants are acceptable,
\methsearch\ applies all options provided to \methsearch\ \textit{except}
\verb+-F+ options (which govern falseness), \verb+-Q+ options
(that apply a final layer of filtering to the result set) and 
\verb+--start-at+ (which allows a search to be restarted part-way through).
This subtlety is subject to change in a future version of \methsearch.

By default, \methsearch\ requires all of the working bells (i.e.\ those
not covered by a \verb+-U+$n$ option) to be in a single cycle performing
the same work.  In other words, differentials and differential hunters are
never produced as they, by definition, have multiple sets of working bells;
nor are methods with multiple hunt bells (except as permitted by \verb+-U+$n$
options).  
By specifying \verb+-A+,\oid{A}{all-methods} all such methods are included.  
To include only certain additional types of method --- for example, to include
differentials, but not hunters or differential hunters when doing an
\verb+-U0+ search, the \verb+-Q+ option should be used.%
\index{differentials}\index{differential hunters}
Typically this will involve tests of the \verb+$u+ and/or \verb+$o+ variables.

The \verb+--offset-cyclic+ option\loid{offset-cyclic} is very specialised.%
\index{cyclic, offset}   It limits the search
to methods that, when started from the backstroke snap (or last backstroke
snap for \verb+-G+$n$ with $n>1$) produce a cyclic row one lead further on.
Spinning Jennie Delight Royal\index{Spinning Jennie Delight Royal} is an 
example of such a method.  
Because \verb+--offset-cyclic+ actually effects the row at the backstroke
snap, it can be used in conjunction with other options governing the lead head;
however, because the lead head and backstroke snap are so close together, 
doing so severely limits the opening few changes of the method.

\index{lead end|)}\index{lead head|)}

\section{Half leads}\label{hls}\index{half lead|(}

\begin{tabularx}{\textwidth}{lX}
\texttt{--regular-hls}&Require regular half leads\\
\texttt{--cyclic-hle}&Require cyclic half lead ends\\
\texttt{--cyclic-hlh}&Require cyclic half lead heads\\
\texttt{--rev-cyclic-hle}&Require reverse cyclic half lead ends\\
\texttt{--rev-cyclic-hlh}&Require reverse cyclic half lead heads\\
\end{tabularx}

For the purpose of this documentation, 
the row immediately before the half lead change (that is, the 
change when the treble is lying behind) is referred to the 
\textit{half lead end} and the row immediately after is referred to as the
\textit{half lead head}.

The \verb+--regular-hls+ option\loid{regular-hls}
specifies that the method must have regular
half leads.  This means that when the half lead end and the half lead head
are written backwards, they must be regular lead heads and lead ends 
or vice versa.  For example, both Beverley\index{Beverley Surprise Minor} 
and Surfleet Surprise Minor\index{Surfleet Surprise Minor}
have regular half leads even though the two half lead rows come in different
orders in the two methods.  Cambridge Surprise Minor%
\index{Cambridge Surprise Minor}, however, does not have
regular half leads, instead having completely different rows around the half
lead.  All regular palindromic double (\verb+-rsd+) methods necessarily have 
regular half heads.  If you want to use half-lead calls, it can be useful
to have regular half leads.

The \verb+--cyclic-hle+ and \verb+--cyclic-hlh+ options%
\loid{cyclic-hle}\loid{cyclic-hlh} specify that the
half lead end or half lead head, respectively. should be descending cyclic 
row (e.g.\ \verb+456231+ for a single-hunt minor method).
The \verb+--rev-cyclic-hle+ and \verb+--rev-cyclic-hlh+ options%
\loid{rev-cyclic-hle}\loid{rev-cyclic-hlh} specify that 
the half lead end or half lead head, respectively, should be ascending cyclic 
row (e.g.\ 432651 for a single-hunt minor method).

\index{half lead|)}

\section{Symmetry}\label{symmetry}\index{symmetry|(}

\begin{tabularx}{\textwidth}{llX}
\texttt{-s}&\texttt{--symmetric}&Require palindromic methods\\
\texttt{-d}&\texttt{--double}&Require glide symmetric methods\\
\texttt{-k}&\texttt{--rotational}&Require rotationally symmetric methods\\
&\texttt{--mirror}&Require mirror symmetric methods\\
&\texttt{--floating-sym}&Allow the symmetry points in arbitrary places\\
\end{tabularx}

These options govern the symmetry that the method must have.%
\footnote{The definitive mathematical treatment of symmetry in methods
is given in a paper by Martin Bright which can be found at 
\url{http://www.boojum.org.uk/ringing/symmetry.pdf}.}

\textit{Palindromic symmetry}\index{symmetry!palindromic}, 
specified with \verb+-s+,\oid{s}{symmetric}
is the most common form of symmetry. 
For methods with an odd number of hunt bells, the method must have a plane
of symmetry through the lead end change as Plain Bob does.  For plain methods
with an even number of hunts, the plane of symmetry must be through the 
following change as with Grandsire.  For the esoteric case of treble dodging 
methods with an even number of hunts, the symmetry plane must pass through
the middle of the treble's 1--2 up dodge(s).

\textit{Rotational symmetry}\index{symmetry!rotational}, 
specified with \verb+-k+,\oid{k}{rotational} means that the method
is invariant under a rotation about the quarter lead.  
Brave New World Bob Royal is an example of a method with just this symmetry.

\textit{Glide symmetry}\index{symmetry!glide},
specified with \verb+-d+,\oid{d}{double} means that the method is 
invariant when reversed (i.e.\ when the frontwork move to the back and 
vice versa) and then shifted by half a lead.  
The Central Council refers to this form of symmetry as 
\textit{double symmetry}\index{symmetry!double|see{glide}}, 
although this term is sometime incorrectly
reserved to mean the combination of glide and palindromic symmetry.  
Methods with just glide symmetry are relatively rare;y rung; 
Double Resurrection Cyclic Bob Royal is an example of one.

\textit{Mirror symmetry}\index{symmetry!mirror} 
is similar to glide symmetry, and is required
by the \verb+--mirror+ option.\loid{mirror}  
It means that method is invariant when 
each change is reversed without a subsequent shift of half a lead.  It 
is found in Mirror Bob Major.  Mirror symmetry heavily restricts the 
possible changes available for producing a method: for example, on six bells,
only \verb+x+, \verb+16+, \verb+34+ and \verb+1256+ are allowed.
By the nature of the symmetry, mirror symmetric methods with a hunt bell
will always have a second hunt bell following the same path but half a lead
out of sync; as a result \verb+-A+ will always need to be specified too.
Under these definitions, Original on even stages has mirror symmetry rather 
than glide symmetry.  This is because shifting the method by half a lead 
(one change) moves the cross changes to the \verb+1N+ changes.

For methods with hunt bells, the \verb+-s+ and \verb+-k+ options prescribe
the symmetry points of the method.  This means that it is, exceptionally, 
possible for an \verb+-s+ or \verb+-k+ search to omit a method with 
palindromic or rotational symmetry, but where the symmetry point is in
an unorthodox position.  Such a method can only exist if there is one (or 
more) additional hunt bells following the same path as the treble but that is
not explicitly required by a \verb+-U+ option --- this means that this 
possibility can only occur if \verb+-A+ is specified too.

When searching for principles (that is, with \verb+-U0+), the planes of 
symmetry (for \verb+-s+) are normally fixed at the half-lead and lead-end,
and or point of rotational symmetry (for \verb+-k+) is normally fixed at
the quarter and three-quarter leads.  
The \verb+--floating-sym+\loid{floating-sym} allows to be at arbitrary points;
however, it makes the search \textit{much} slower as the requirement for
floating symmetry point is enforced by filtering the results of the search,
instead of pruning the search tree during the search.

If the principle is to have an odd 
number of changes per lead, then the only meaningful symmetries are
rotational and mirror symmetry.\footnote{Palindromic symmetry is possible
in theory, but only in trivially false methods.}  \methsearch\ does not
currently support rotation symmetry in methods of odd length.
%\methsearch's handling of symmetry in
%principles is not very sophisticated --- in particular, a number of 
%obvious opportunities for optimisation are neglected.  This may be addressed
%in a future version of \methsearch.

If a method has any two of palindromic, rotational or glide symmetry,
it necessarily has the third one.  It is therefore unnecessary, though 
harmless, to specify all three of \verb+-skd+.

\index{symmetry|)}

\section{Place notation}\label{pn}\index{place notation|(}

\begin{tabularx}{\textwidth}{llX}
\texttt{-m}&\texttt{--mask=MASK}&Specify part of the place notation\\
&\texttt{--prefix=PN}&Specify the first part of the place notation\\
&\texttt{--start-at=PN}&Resume a search from a given place notation\\
&\texttt{--changes=LIST}&Allow only changes from the list\\
\end{tabularx}

The \verb+-m+ option\oid{m}{mask} provides a 
powerful mechanism for specifying part
of the method's place notation.  Its argument is a \textitidx{method mask}. 
At its simplest, a method mask can just be the complete place notation for 
a method which restricts \methsearch\ to looking at just that one method.

\index{bells!symbols for} 

\begin{table}[h]\centering
{\footnotesize\begin{tabular}{cccccccccc cccccccc}
1&2&3&4&5&6&7&8&9&10&11&12&13&14&15&16&17&18\\
\texttt{1}&\texttt{2}&\texttt{3}&\texttt{4}&\texttt{5}&\texttt{6}&\texttt{7}&\texttt{8}&\texttt{9}&\texttt{0}&\texttt{E}&\texttt{T}&\texttt{A}&\texttt{B}&\texttt{C}&\texttt{D}&\texttt{F}&\texttt{G}\\
\end{tabular}}
\caption*{}
{\footnotesize\begin{tabular}{cc cccccccccc ccc}
19&20&21&22&23&24&25&26&27&28&29&30&31&32&33\\
\texttt{H}&\texttt{J}&\texttt{K}&\texttt{L}&\texttt{M}&\texttt{N}&\texttt{P}&\texttt{Q}&\texttt{R}&\texttt{S}&\texttt{U}&\texttt{V}&\texttt{W}&\texttt{Y}&\texttt{Z}\\
\end{tabular}}
\caption{\label{tab:bellsym}The default set of bell symbols.}
\end{table}

The \textit{place notation} can be expressed in any standard form.  
Bells are represented using the case-insensitive alphabet shown in
table~\ref{tab:bellsym}.%
\footnote{This alphabet can be overridden with the \verb+$BELL_SYMBOLS+
environment variable.\fspecd{BELL\_SYMBOLS}  
For example, setting \verb+BELL_SYMBOLS="123456789ABCDEF"+
tells \methsearch\ to use hexadecimal digits instead of the conventional 
symbols to represent bells (on up to 15 bells).  
The \verb+$BELL_SYMBOLS+ variable must be at least
as long as the number of bells specified with \verb+-b+.  If the variable
is entirely in upper case or entirely in lower case, 
then bell symbols are considered to be case insensitive; 
if it is mixed case, they are assumed to be case sensitive.  
The default alphabet is \textit{not} compatible with
the alphabet used in Abel\index{Abel} on 20 or more bells as Abel uses 
\verb+I+ to represent bell number 20.  The convention used by \methsearch\ is
the one used in the Central Council's method libraries.}
External places may be stated
explicitly or left implicit, the cross change can be denoted \verb+x+, 
\verb+X+ or \verb+-+ as preferred,\index{x@\texttt{x}|see{cross change}}%
\index{cross change}\symidx{-}{cross change}%
\footnote{Part 1 of the Central Council's \textit{Doubles Collection}
uses \verb+x+ in a non-standard way to denote any single change in which only
the treble moves.  For example, it gives the place notation of the first 
half-lead of Huntley Place Doubles as \texttt{x.1.x.x.5}. 
This use is not supported in \methsearch.}
and changes are separated using a \verb+.+\symidx{.}{place notation} 
which may be omitted next to cross
changes.  Symmetric blocks of place notation are prefixed with a 
\verb+&+\symidx{\&}{place notation} to save repetition, asymmetric blocks 
may (though needn't) be prefixed with a \verb-+-,\symidx{+}{place notation} 
and blocks are separated by a \verb+,+.\symidx{,}{place notation}  For example,
the place notation for Plain Bob Minor can be written most concisely
as \verb+&-1-1-1,2+ or in full as \verb-X.16.X.16.X.16.X.16.X.16.X.12-.%
\footnote{These are the two variants produced by the \verb+$q+ and
\verb+$p+ variables described in \sref{variables}.}

In a method mask, a specific change can be left unspecified by writing 
\verb+?+\symidx{?}{method mask} in its place; 
a sequence of several consecutive 
unspecified changes can be written \verb+*+.\symidx{*}{method mask}  
A method mask can have at most one \verb+*+
because otherwise \methsearch\ would be unable 
to calculate how many changes are represented by each \verb+*+.

A change needn't be completely unspecified or fully specified: it is also
possible to give a list of alternatives for a given change.  The list is 
enclosed in \verb+(+\ldots\verb+)+\index{parentheses} 
and the options are delimited by \verb+|+.\symidx{"|}{method mask}  
For example, \verb+&(34|5).1.5.1.5,2+ denotes Reverse Canterbury
or Plain Bob Doubles, as well as two false methods produced by putting a
\verb+34+ in one half and a \verb+5+ in the other half.  This point is 
important: the \verb+&+ only states that the mask is symmetric, not that
the method derived from the mask is symmetric --- use \verb+-s+ for that.

Parentheses cannot be used to enclose longer fragments of place notation.
For example, it is \textit{not} possible to specify that a method must have 
either a Kent\index{Kent Treble Bob} or an Oxford\index{Oxford Treble Bob} 
start by writing \verb+&(-34-|34-34).*,?+.

Finally, a mask can specify separate mask for the above
and below works.\index{above work}\index{below work}  The above and below 
work masks are separated by a \verb+/+.\symidx{/}{method mask}
Within a mask, the \verb+/+ operator has higher 
precedence\index{operators!precedence} than the 
\verb+,+ operator --- in other words, 
\texttt{$a$, $b$ / $c$} means the mask $a$
followed by the mask produced by combining $b$ over $c$.  For example,
to specify a doubles method with either Plain Bob or Reverse Canterbury
above the treble, you would write \verb+&(34|5).1.5.1.5/*,2+.

When specifying masks on the command line, it is often necessary to enclose 
them in quotes.\index{quoting|see{shell, quotation}}\index{shell!quotation}
This is because certain symbols, particularly \verb+&+ and
\verb+|+ have meanings that the shell might try to interpret.  On Unix systems,
this can usually be done with either single or double quotes; under the 
Windows command prompt, double quotes appear to be necessary:

\begin{Verbatim}
methsearch -b5 -s -m"&(34|5).1.5.1.5/*,2"
\end{Verbatim}

When searching for a method with hunt bells (i.e.\ a search other than with
\verb+-U0+), the mask can override the requirements set by \verb+w+ and
\verb+-f+ that the method is right-place or has no penultimate places name
above the treble.  Thus \verb+-w -m"&3*,?"+ will look for methods
that are right place except for starting with a \verb+3+ place notation.  

When searching for methods without hunt bells (i.e.\ with \verb+-U0+), a 
greater range of options can be overriden.  Currently they are \verb+-w+,
\verb+-f+, \verb+-l+, \verb+-j+, \verb+--changes+, \verb+--mirror+ and 
\verb+-Fx+ (insofar as it prohibits the null change\index{null change}).

The \verb+--prefix+ option\loid{prefix} specifies the place notation for the
opening section of the method.  This option is deprecated in favour
of the \verb+-m+ option.  It is orthogonal to \verb+-m+ meaning that
if both \verb+--prefix+ and \verb+-m+ are specified then the method must
satisfy both.

The \verb+--start-at+ option\loid{start-at} allows a search to resumed 
mid-way through.  This can be used to resume a search that was cancelled 
earlier by passing the place notation of the last method found before the 
search was cancelled.\index{resuming a search}  Restarting a random search
is only meaningful if the same random seed is used throughout and passed
to \verb+--seed+.

The \verb+--changes+ option\loid{changes} restricts the changes that
\methsearch\ will consider to those in the comma-separated list provided.  
For example, to find methods made solely using the \verb+x+, \verb+14+ and
\verb+18+ changes, you could write \verb+--changes=x,14,18+.  
Alternatively, the \verb+--changes+ option can be used to prohibit certain 
changes.  This is done by prefixing the list of changes with a \verb+!+,
so, for example, \verb+--changes=!1234+ will allow any changes except for
\verb+1234+.  (In Unix shells, it will probably be necessary to quote the
argument to prevent the shell from interpreting the \verb+!+.)

This option can be useful for generating a method (and particularly principles)
where the course forms a group\index{group}.  Any collection of changes 
will generate a group, though often the group is simply the whole extent,
$S_n$ (or the in-course half of it, $A_n$) which is of little interest.
And the changes used in a method whose plain course is a group must be 
generate that group.  So to find a method whose plain course is a group,
we just need to identify the changes needed to generate the group and put
them in a \verb+--changes+ argument.  
For example, Striking Minor\index{Striking Minor} forms a group of order 120 
generated by \verb+x+, \verb+14+ and \verb+36+.%
\footnote{This group is of some interest.  It is isomorphic to $S_5$, 
but is not conjugate to the standard representation of $S_5$ as the 
permutations of five bells.  It is the dual of the standard representation
of $S_5$ under the outer automorphism of $S_6$.
Hudson's group\index{group!Hudson's} is its even-parity subgroup.}
\methsearch\ can identify other such principles with a search such as:

\begin{Verbatim}
methsearch -b6 -U0 -n20 --changes=x,14,36 -p2 -s
\end{Verbatim}

The \verb+--changes+ and \verb+--mask+ options are orthogonal, and similarly
both options are orthogonal to \methsearch's other options that effect the
permitted changes such as \verb+-l+, \verb+-j+ and \verb+-f+ 
(\sref{int_struct}).  For \methsearch\ to find a method, it must satisfy all
of the requirements.  In particular, a \verb+--mask+ cannot be used to 
override these restrictions in one part of the method.  This feature
may be included in a future version of \methsearch.

\index{place notation|)}

\section{Output options}\label{output_opt}\index{output|(}

The options in the previous sections control what
\methsearch\ searches for.  By contrast, the options here control
what \methsearch\ does when it has found a method.

\begin{tabularx}{\textwidth}{llX}
\texttt{-q}&\texttt{--quiet}&Suppress output of methods as they are found\\
\texttt{-R}&\texttt{--format=FMT}&Format methods in the specified way\\
\texttt{-o}&\texttt{--out-file=FILE}&Write methods to the given file\\
\texttt{-O}&\texttt{--out-format=TYPE}&
  In conjunction with \texttt{-o} to specify the file format of \texttt{FILE}\\
\texttt{-L}&\texttt{--library=LIB}&Use method library to look up method names\\
\texttt{-H}&\texttt{--frequencies=FMT}&Count frequencies of method properties\\
\texttt{-C}&\texttt{--count}&Count the methods found\\
&\texttt{--raw-count}&A more concise version of \texttt{--count}\\
&\texttt{--node-count}&Count the search tree nodes visited\\
\texttt{-u}&\texttt{--status}&Keep a running status of progress\\
&\texttt{--status-freq=N}&Display the status every \texttt{N} nodes\\
\end{tabularx}

\methsearch\ has three main types of output: method output,
statistical output, and counts.  If several types of output are to be
produced, they come in the order method, stats, counts, and each type
is separated by a blank line.

\textit{Method output} is when \methsearch\ prints details of each
matching method as it is found.  This is enabled by default and 
can be suppressed with the \verb+-q+ option.\oid{q}{quiet}  
The information that \methsearch\ prints, 
together with the formatting of it, is controlled
by the \verb+-R+\oi{R} option which is discussed in \sref{variables}.
By default, \methsearch\ prints just the methods' place notation and lead head
--- i.e.\ omitting the \verb+-R+ option is equivalent to 
specifying \verb+-R$p\t$l+ (unless \verb+-I+ is supplied in which
case \verb+-R$p\t$a+ is assumed).

The \verb+-o+ option\oid{o}{out-file} tells \methsearch\ to write all 
method output (but not other types of output) into a file.  
This is different from using
shell redirection\index{shell!redirection} 
with \verb+>+ as that redirects all standard output,%
\footnote{Command line utilities have two types of output:
\textitidx{standard output} and \textitidx{standard error}, both of which
are displayed together on the terminal.  \methsearch\ uses
standard output for all of its output except error messages and the 
status messages from \verb+-u+.\oi{u}  Standard redirection with \verb+>+ only
redirects standard output, not standard error.}
not just method output.  One reason to favour \verb+-o+ over shell redirection 
is that when used with \verb+-u+, shell redirection can result in very 
flickery status line.  
(This is due to technical limitations of the terminal and not a bug
in \methsearch.)  The special combination \verb+-o-+ is equivalent to an
omitted \verb+-o+ option.

The \verb+-O+ option\oid{O}{out-format} can be used to further control 
what is written to the \verb+-o+ file.  At present three arguments to \verb+-O+ 
are supported: \verb+fmt+, the default, which does formatted output as 
specified by the \verb+-R+ option; \verb+utf8+, which is like \verb+fmt+ but
enables Unicode\index{Unicode} characters to be output in UTF-8\index{UTF-8}; 
and \verb+xml+, which tells \methsearch\ to produce an XML file containing 
the methods.\index{XML|(}%
\footnote{The XML conforms to the \textit{schema} published by the 
online method database project which is documented at
\url{http://methods.ringing.org/xml.html}.  This schema is not the one 
current used by the Central Council who are using gratuitously bastardised
version of it.}
It is possible to produce builds of \methsearch\ without XML support,
and, obviously, if you are using such a build, \verb+-Oxml+ will not be 
supported.  If \verb+-Oxml+ is specified then a \verb+-o+ option must be 
supplied (and not \verb+-o-+).

It is possible to use \verb+-R+ to get \methsearch\ to print the name of
methods it finds.  To do this it needs access to one or more method libraries
containing the relevant methods.  These may be Central Council text 
format collections,\index{Central Council!method collections, online}%
\footnote{Available from \url{http://www.methods.org.uk/method-collections/}.}
the antiquated MicroSiril library format,\index{MicroSiril libraries}%
\footnote{MicroSiril libraries are not recommended as they have a number 
of problems, most notably not properly supporting methods with a space in
their name (such as Plain Bob).}
or, if XML is supported, XML libraries.\index{XML|)}
The \verb+-L+ option\oid{L}{library} specifies the library file to load.  
If multiple library files are needed, for example one for surprise, one for
delight, and a third for treble bob, multiple \verb+-L+ options can be 
supplied.  If the filename is not an absolute path (i.e.\ if it does not
begin with a \verb+/+ on Unix, or \verb+\+ or \verb+X:\+ for some \verb+X+ 
on Windows), and if the 
\verb+$METHOD_LIBRARY_PATH+\fspecd{METHOD\_LIBRARY\_PATH} or 
\verb+$METHLIBPATH+\fspecd{METHLIBPATH} \textitidx{environment variable}%
\footnote{The two variables are synonymous.  If both are set, 
\verb+$METHLIBPATH+ is ignored.}
is set to a colon-delimited list of directories, these directories are 
search in turn for the file.  How an environment variable is set depends 
on your system.  The three commands below are for Windows, most Unix-like 
systems, and finally the Unix csh.\index{csh@\texttt{csh}}

\begin{Verbatim}
set METHLIBPATH=C:\Documents and Settings\Richard\methodlibs
export METHLIBPATH=/home/richard/methodlibs
setenv METHLIBPATH /home/richard/methodlibs
\end{Verbatim}

\textit{Statistical output}\index{statistics} 
is for gathering basic statistics such as
the number of methods with each lead head.  It is controlled by the 
\verb+-H+ option\oi{H} which is discussed in \sref{variables}.  By default
no statistical output is produced.

\methsearch\ maintains two \textit{counts} while it searches for methods.  
First it maintains a \textit{method count} which is simply the number of
matching methods.  
The \verb+-C+ option\oid{C}{count} prints this value at the end of the
search; the \verb+--raw-count+ option\loid{raw-count} 
is similar except that it prints
just the number without any associated verbiage.  (The latter can be
useful if another program is reading the output of \methsearch.)

Secondly, it maintains a \textit{node count}.  This is the number of whole
or partial methods that \methsearch\ has considered while performing the 
search.  This can be displayed with the \verb+--node-count+ option.%
\loid{node-count}  Unlike the method count, this gives a good idea of the
complexity of the search.\index{complexity}
In 2009, a fast computer performing a straightforward search 
should be able to do perhaps as many as 500,000 $\hbox{nodes}/\hbox{s}$;
however as the complexity of the search increases, so its speed will decrease
considerably.

When performing a very long search, especially one that finds few methods,
it can be difficult to know how far through the search \methsearch\ has got.
The \verb+-u+ option\oid{u}{status} tells \methsearch\ to display a 
\textit{status line}
which is updated by default every 10,000 nodes to show the first 68 characters 
of the place notation of the current node.\footnote{The number of characters
to display can be overridden with the \verb+$COLUMNS+ environment variable%
\fspec{COLUMNS} in which case \verb+$COLUMNS+$-$12 characters are printed. 
Many shells set this variable automatically.}
The frequency with which this is displayed can be overridden with the 
\verb+--status-freq+ option.\footnote{This option also controls the 
frequency with which timeouts (\sref{random}) are tested.}
This information is displayed on
\textitidx{standard error} meaning that it is not captured by shell
redirection using \verb+>+.
\index{output|)}

\section{Random sampling}\label{random}

\begin{tabularx}{\textwidth}{lX}
\texttt{--limit=N}&Terminate the search after finding \texttt{N} methods\\
\texttt{--random}&Randomise the order in which methods are listed\\
\texttt{--loop[=N]}&
  Repeatedly search for one random method, indefinitely or \texttt{N} times\\
\texttt{--seed=N}&Seed the random number generator with \texttt{N}\\
\texttt{--timeout=N}&Timeout the request after \texttt{N} seconds\\
\end{tabularx}

The \verb+--limit+ option\loid{limit} is used to terminate after a 
certain number of
methods have been found.  This can be useful if all you care about is 
whether any method exists with a particular set of properties.  In this 
context, the combination \verb+-q+ \verb+--limit=1+ \verb+--raw-count+ 
can be useful when \methsearch\ is being invoked by a script.  
The output is just a \verb+0+ or \verb+1+ 
depending on whether the criteria can be satisfied.  
%The \verb+--limit+
%option cannot be used in conjunction with the \verb+--random-count+ option.

The \verb+--random+ option\loid{random} tells \methsearch\ to randomise the 
order in which it traverses the search space.  It is important to note, however,
that \methsearch\ still does a depth-first search and it is only the 
order of the child nodes at each point in the search tree that is randomised.%
\footnote{The reason for this choice of implementation is that a truly 
randomised search would require \methsearch\ to store every single method 
it finds, which in turn would used a large amount of memory.  
If you wish to genuinely randomise the methods, the \verb+-R+ option to the 
standard \verb+sort+\index{sort@\texttt{sort}} utility will do this.}
This means that all methods beginning with a given place notation sequence
will still be listed together, however the first method found will be fairly
random.\footnote{Even the first method isn't selected with uniform probability
from the set of possibles.  This is because methods in a part of the
search space with few matches are more likely to turn up.  This means that
trivial underwork variants are underrepresented.}  
%Used in conjunction with \verb+--limit=1+, this option is good way of selecting
%a method at random.

The \verb+--loop=+$n$ option\loid{loop} is used to modify the behaviour of
a random search so that it does a searches for a single method $n$ times.  
In contrast to \verb+--random+ \verb+--limit=+$n$ which will find $n$ distinct 
methods close together in the search space, 
\verb+--random+ \verb+--loop=+$n$ will
restart the search after each method it finds.  This means that methods
will be distributed throughout the search space.  However there is no guarantee
that the $n$ methods will be distinct (i.e.\ that there will be no repetition),
but if the number of methods requested is much less than the total, it is 
likely.  If the \verb+--loop+ option is given without an argument, the loop
runs indefinitely.

The \verb+--seed+ option\loid{seed} seeds the random number generator with 
the given value.  If you are invoking \methsearch\ with \verb+--random+ 
lots of times in a short period of time,
it is important to use this because \methsearch\ simply uses the current time 
(in seconds) to seed the random number generator.%
\footnote{Frequently, the \verb+--loop+ option provides a better means of 
doing this.  However, when required, in \verb+bash+ and some other popular 
Linux shells, the \verb+$RANDOM+ shell variable\fspec{RANDOM} provides a 
good way of generating a random seed that can be recorded if necessary and 
passed to \methsearch.}
Similarly, if you want the methods to occur in an arbitrary yet 
reproducible order, this is helpful.

The \verb+--timeout=+$n$ option\loid{timeout} tells \methsearch\ to give up
after $n$~seconds on a search (or, with \verb+--loop+, an interation of the 
search loop).%
\footnote{The \verb+--timeout+ option will not kill a command invocation
(\sref{cmdinv}) that is taking too long, and the accuracy of its timing
is not high.  It works by getting the time every 10000 search nodes and
testing whether the specified number of seconds has elapsed.  The frequency
with which the timeout is tested is controlled by the \verb+--status-freq+
command.}
When searching for random methods from a search space with large regions
in which no valid methods can exist, for example asymmetric surprise major,
it is worth using \verb+--timeout+ to restart the search if it 
appears to be in such a region.  In such a situation it is possible to
combine both an arguementless \verb+--loop+ option and a \verb+--limit=+$n$
option.  In this case, the \verb+--loop+ option ensures that each search
iteration finds just one method, and so the methods are distributed throughout
the search space, and the \verb+--limit=+$n$ option ensures that $n$ methods
are found.  Had \verb+--loop=+$n$ been used, $n$ search iteration would occur,
but as some of these might have timed out, the number of methods found
could be less than $n$.

As an example, the following command will generate 23 utterly structureless 
random treble dodging major methods:%
\index{example!major, random structureless}
\begin{Verbatim}[commandchars=\%~~]
methsearch -b8 -G1 --random --loop --timeout=1 --limit=23
\end{Verbatim}


\section{Miscellaneous options}\label{misc_opt}

\begin{tabularx}{\textwidth}{llX}
\texttt{-?}&\texttt{--help}&Print an help message and exit\\
\texttt{-V}&\texttt{--version}&Print version information and exit\\
\texttt{-I}&\texttt{--filter}&Run in filter mode on standard input\\
&\texttt{--filter-lib}&Run in filter mode on specified libraries\\
&\texttt{--invert-filter}
                &Invert filter so only non-matching methods are listed\\
\texttt{-Q}&\texttt{--require} 
                &Apply an additional requirement to the search results\\
\texttt{-M}&\texttt{--music}&Configure how \methsearch\ evaluates musicality\\
\texttt{-F}&\texttt{--falseness}&Configure how falseness is checked\\
\texttt{-P}&\texttt{--parity-hack}&Require an equal number of rows of each 
  parity for each position of the treble\\
\end{tabularx}

The \verb+--help+\loid{help} option was mentioned in \sref{help}.
It prints a summary of all the options that \methsearch\ knows about.

The \verb+--version+\oid{V}{version} option prints the version number of 
\methsearch.  (In practice, this is less useful than it might be.  Currently,
formal releases of \methsearch\ are made very infrequently, and most copies
in use are development snapshots.)

The \verb+-I+\oid{I}{filter} option is used to change \methsearch\ from
searching for methods \textit{ab initio} to filtering 
a list of methods supplied on \textitidx{standard input}.
The input format is a list of methods, one per line, with the
place notation first on the line followed, optionally, by additional text
that is ignored.%
\footnote{A UTF-8\index{UTF-8} byte order mark (BOM)\index{byte order mark} 
is permitted, though not required, at the start of the first line.  
This has the hexadecimal encoding \texttt{EF BB BF}.  
Other Unicode\index{Unicode} formats (such as UTF-16 or UTF-32) are not 
permitted; nor, in general, are any 
non-ASCII-compatible character sets.\index{character set}}
Any ext after the place notation is called the \textit{filter payload} 
and can be accessed by the \verb+$a+ method variable (\sref{variables}).
The output from methsearch \verb+-R"$q +\ldots\verb+"+ is a suitable input.
This is particularly useful if you want to do a lengthy search, but are not
entirely sure how to restrict the results initially.  Dump everything into
one big file:

\begin{Verbatim}[commandchars=\\\{\},commentchar=~]
methsearch -b\textit{n} -R"$q" \textit{\rm\it[options]} > methods.txt
~ -- a comment to get vim's syntax highlighter back in sync:  $
\end{Verbatim}

\ldots and then filter the output as many times as you want:

\begin{Verbatim}[commandchars=\\\{\}]
methsearch -b\textit{n} -I \textit{\rm\it[other options]} < methods.txt
\end{Verbatim}

Generally, doing a search once and filtering it several times is much faster
than repeating the search once per attempt to filter it.

The \verb+--filter-lib+\loid{filter-lib} option is the same, except that
instead of reading methods from standard input, it reads methods from
all the libraries specified with \verb+-L+\oi{L} options.  This is useful
for finding named methods with particular properties.\index{named methods}

The \verb+--invert-filter+\loid{invert-filter} option is used in conjunction
with either of the options that enable filter mode (\verb+-I+ or 
\verb+--filter-lib+).  It inverts the sense
of the filtering so that \methsearch\ outputs every method that does 
\textit{not} match the search criteria.  
The method count (per \verb+-C+, below) 
and any statistical output (\sref{stats}) is similarly inverted.

The \verb+-Q+\oid{Q}{require} provides a way of imposing more complex
requirements on the methods that \methsearch\ finds.  Its argument is 
an \textit{expression} whose is described in \sref{expr}.  The expression
is evaluated for each method found, and only if it evaluates to true 
is the method found.  Many other options can also be rewritten as a
\verb+-Q+ option; almost invariably this results in a loss of 
efficiency, sometimes spectacularly so.

The \verb+-M+\oi{M} and \verb+-F+\oi{F} options 
configure how methsearch analyses the musicality and 
falseness of a method, respectively.  Music is discussed in further detail in 
\sref{music}, while falseness is covered in \sref{falseness}.

The \verb+-P+\oid{P}{parity-hack} option is rather specialised.  
It imposes requirements on the parity structure\index{parity} of the method.
For a conventional treble-dodging method\index{treble-dodging methods},
this option imposes a necessary (but not sufficient) criterion for the 
existence of a true bobs-only extent.  This is particularly relevant in minor.
This option is discussed further in \sref{extent}.

\section{Response files}\label{respfile}\index{response files|(}

\begin{tabularx}{\textwidth}{lX}
\texttt{@}\textit{filename}&Import commands from the file
\end{tabularx}

In addition to taking command line options on the command line, \methsearch\ 
can read them from a file which is referenced on the command line by
prefixing the filename with an \verb+@+.  Such files are called 
\textit{response files}, and are fairly common on Windows where they can be 
used to work around deficiencies in the Windows shell.  For example, suppose
there's a file called \verb+legal-methods+ containing the following text:
\begin{Verbatim}
-p4 -Q"$u < $B-$u"
\end{Verbatim}
This can be referenced on the command line, with the effect that the 
following two commands are equivalent:
\begin{Verbatim}
methsearch -b6 -s @legal-methods
methsearch -b6 -sp4 -Q"$u < $B-$u"
\end{Verbatim}

In addition to the fact that this allows commonly-used groups of 
options to be saved with a meaningful name so they can be reused,
this also makes it easier to write complex searches so that they can
be run equally easily on Windows and on various Unix shells.

Options in response files can be quoted using either single or double
quotes (\verb+'+ or \verb+"+) --- unlike in most popular shells, both forms of 
quotes are treated the same.\footnote{In particular, there is no need to 
escape a \verb+$+ with a \verb+\+ inside double quotes as there is in most 
Unix shells,\index{shell!escaping} although it is harmless to do so.}
Thus in the example above, in most Unix shells, the quotes on the command
line need replacing with single quotes, but in the response file, either
can be used.\index{shell!quotation}
Single quotes can be nested within double quotes, or vice versa; if an
additional levels of quotes are needed, the innermost ones can be escaped
from the response file by preceding them with a \verb+\+, as in
most shells. 


\index{response files|)}

\chapter{Formatted output}\label{fmtout}
\index{output|(}

The options that control output were listed in \sref{output_opt}.  
This chapter starts by discussing the format strings (\sref{fmtstr}) 
that occur in the arguments to the \verb+-R+\oi{R} and \verb+-H+\oi{H} 
options and provide \methsearch's main way of communicating 
information about the methods back to the user.  
The expression syntax (\sref{expr}) used by the \verb+-Q+\oi{Q} 
option is also documented here as it shares much in common with format strings.
Finally, \sref{stats} looks specifically at the \verb+-H+ option for 
statistical output.

\section{Format strings}\label{fmtstr}

The argument to the \verb+-R+ and \verb+-H+ options\oid{R}{format}\oi{H} is a 
\textitidx{format string}.  This is a string which specifies the 
information to be printed when each method is found (in the case of
the \verb+-R+ option) or in each line of the statistical output 
(in the case of the \verb+-H+ option).  At its simplest, the format
string might just contain literal text\index{literal text}, 
for example, \verb+-RFound+ which would simply print `\verb+Found+' 
each time a method was found.
This, however, isn't very useful as it coveys no information about 
the method.

In addition to literal text, the format string may contain
\textit{variables} (\sref{variables}), 
\textit{character escape sequences} (\sref{charesc}),
\textit{command invocations} (\sref{cmdinv}) and
\textit{expressions} (\sref{expr}).  

\subsection{Method variables}\label{variables}
\index{variables|(}\index{method parameters|(}

\textit{Variables} are pieces of text beginning with a 
\verb+$+\symidx{\$}{variables}
that are replaced by a \textit{method parameter} --- a piece of information
about the method.\footnote{A \verb+%+\symidx{\%}{variable} 
can be used as an alternative to
\verb+$+ in variables, although not in expressions such as in
the argument to \verb+-Q+.  This use of \verb+%+ is deprecated.}
For example, \verb+-R$q+ causes the place notation%
\index{place notation} of each method to be printed out as they are found.
Some variables allow (and a few require) an integer to be specified
between the \verb+$+ and the variable letter.  
An example is \verb+$12h+ which prints the 12th change of the method.

It is usually necessary to enclose format strings in quotation marks.%
\index{shell!quotation}  This is needed so that the shell does not try to
interpret it itself.  Spaces and the \verb+$+ 
are both prone to interpretation by the shell, and both of these commonly
occur in format strings.  On Windows, double quotes should be used,
(e.g.\ \verb+-R"$q"+); on Unix-like systems, single quotes are needed 
(e.g.\ \verb+-R'$q'+) to prevent the shell from 
interpolating\index{shell!interpolation} the \verb+$+ sign.  See
\sref{eg_doubles} for an example of this.  The quote characters are never
seen by \methsearch\ --- they exist solely to make sure that the options
typed in on the command line reach \methsearch\ in their intended form.

{\def\D{\texttt{\$}}\def\N{$n$\/}\def\No{\textit{[$n$]}\/}
\def\F#1{\texttt{#1}\fspecd{#1}}
\begin{tabularx}{\textwidth}{lX}
\D\F{p}&The place notation (verbose format) --- see \sref{pn}\\
\D\F{q}&The place notation (concise format)\\
\D\F{Q}&As \texttt{\$q} but without the lead end change%
  \index{place notation!printing|see{\texttt{\$p}, %
    \texttt{\$q} \textit{and} \texttt{\$Q}}}\\
\D\N\F{h}&Change number \N\ from the method ($n=1$ for first change)%
  \index{changes, printing|see{\texttt{\$h}}}\\
\D\N\F{r}&Row number \N\ from the method ($n=0$ for opening rounds)%
  \index{rows, printing|see{\texttt{\$r}}}\\
\D\F{l}&The lead head row%
  \index{lead head!printing|see{\texttt{\$l}}}\\
\D\No\F{L}&The number of changes per lead%
  \index{lead length!printing|see{\texttt{\$L}}}\\
\D\No\F{b}&Maximum consecutive blows in one place%
  \index{blows, maximum consecutive!printing|see{\texttt{\$b}}}\\
\D\No\F{o}&The number of leads per course\\
\D\No\F{u}&The number of hunt bells\\
\D\No\F{B}&The number of bells\\
\D\F{d}&The lead head code (modern symbol)\\
\D\F{D}&The lead head code (traditional symbol)
  \index{lead head!code|see{\texttt{\$d} \textit{and} \texttt{\$D}}}\\
\D\F{y}&A string representing the symmetries of the method%
  \index{symmetry!printing|see{\texttt{\$y}}}\\
\D\F{n}&The method name (excluding class and stage)\\
\D\F{N}&The method title (including class and stage)\\
\D\F{C}&The class name, e.g.\ `Treble Bob'%
  \index{class!printing name|see{\texttt{\$C}}}\\
\D\F{S}&The stage name, e.g.\ `Royal'%
  \index{stage!printing name|see{\texttt{\$S}}}\\
\D\No\F{M}&The musical score --- see \sref{music}\\
\D\F{F}&The falseness groups --- see \sref{falserest}\\
\D\N\F{P}&The path for bell \N\\
\D\F{O}&The coursing order (defined so the tenors course)\\
\D\No\F{s}&An index of the staticity of the method\\
\D\No\F{i}&The method identifier from the CC libraries\\
\D\No\F{a}&The filter payload --- see \sref{misc_opt}\\
\D\No\textit{\#}&The number of that method\\
\D\F{T}&The current time (formatted as hh\texttt{:}mm\texttt{:}ss)\\
\D\No\F{c}&The number of matching methods --- see \sref{stats}\\
\D\No\texttt{?}&The exit status of the last command invocation --- 
             see \sref{cmdinv}\\
\end{tabularx}}

In the above table, $n$ stands for an integer which must be supplied.
For example, in a surprise major method, \verb+-R$32h+ will print the
lead end change, but \verb+-R$h+ without a numeric argument would give
an error message.\fspec{h}  It may seem illogical that the \verb+$r+
option\fspec{r} uses \verb+$0r+ for the first row (that is, rounds), and 
\verb+$h+ uses \verb+$1h+ for the first change.  This is because there are
$n+1$ relevant rows in the lead, but only $n$ changes.

In most circumstances an offset, $n$, that's longer than the lead length
will result in a error message when \methsearch\ starts.  The exception is
when filtering over methods with variable length leads (e.g.\ with \verb+-IAU0+
and no \verb+-n+$n$ option).  In this case, an expression error%
\index{expression error} (\sref{type_sys}) results from an out-of-bounds 
access.

In the table, \textit{[$n$]} means an optional integer.  In each case,
this is used as a \textitidx{field width}.  This is used to left-pad
numbers with spaces up to the specified with in order to align 
numbers with different numbers of digits.  This is most useful with 
\verb+$c+\fspec{c}.  For example, \verb+-H"$5c $y"+ contains the number 
methods with each type of symmetry.   The \verb+5+ tells \methsearch\ 
to pad each count to five digits.\footnote{C programmers may recognise
that this is inspired by the \verb+printf+ format specifiers, 
for example \verb+%d+ versus \verb+%5d+.}

A small number of variables can only be used in specific 
contexts.  The place notation, name and identifier variables, 
\verb+$p+, \verb+$q+, \verb+$Q+, \verb+$n+, \verb+$N+ and \verb+$i+,
cannot be included in \verb+-H+ options.
%This is not a technical restriction \textit{per se}, rather that as 
%place notations are unique to a method and names nearly so, it would rarely
The \verb+$c+ variable can only be included in \verb+-H+ format strings, 
and the \verb+$#+ and \verb+$T+ variables can only be included in \verb+-R+ 
format strings.
\index{variables|)}\index{method parameters|)}

Many of the method variables above are self explanatory; however a few
bear further comment.  Options such as \verb+$b+\fspec{b} (the maximum
consecutive blows in one place) return the maximum number actually found
in the particular method, rather than simply returning the value passed
to the \verb+-p+ option.  The number of hunt bells, \verb+$u+\fspec{u} 
includes any bell that is back in its starting place at the lead head.

The modern lead head code, \verb+$d+\fspec{d}, is the lead head as defined
in modern Central Council collections and enumerated in 
table~\ref{tab:mod-lhcode}.
It is a lower case letter \verb+a+--\verb+r+ (excluding \verb+i+); 
on nine or more bells, it may also have a numeric suffix.  For compatibility
with MicroSiril\index{MicroSiril libraries}, the letter \verb+z+ is used
to denote all lead head types without a code, including methods with regular
(i.e.\ Plain Bob) lead heads, but with a short course, such as a royal
method with lead head \verb+1795038264+.


\begin{sidewaystable}[p]\centering{\tt\tiny
\def\head#1{\multicolumn{3}{l|}{{\rm\footnotesize#1}}}
\def\lasthead#1{\multicolumn{3}{l}{{\rm\footnotesize#1}}}
\begin{tabular}{lll| lll| lll| lll| lll| lll| lll}
\head{Minimus}&\head{Minor}&\head{Major}&\head{Royal}&\head{Maximus}&
\head{Fourteen}&\lasthead{Sixteen}\\\hline
342&a&g&35264&a&g&3527486&a&g&352749608&a &g &3527496E8T0&a &g 
  &3527496E8A0BT&a &g &3527496E8A0CTDB&a &g \\
   & & &56342&b&h&5738264&b&h&573920486&b &h &57392E4T608&b &h 
  &57392E4A6B8T0&b &h &57392E4A6C8D0BT&b &h \\
   & & &     & & &7856342&c&j&795038264&z &z &795E3T20486&c &j 
  &795E3A2B4T608&c &j &795E3A2C4D6B8T0&z &z \\
   & & &     & & &       & & &907856342&c1&j1&9E7T5038264&c1&j1
  &9E7A5B3T20486&c1&j1&9E7A5C3D2B4T608&c1&j1\\
   & & &     & & &       & & &         &  &  &ET907856342&c2&j2
  &EA9B7T5038264&c2&j2&EA9C7D5B3T20486&z &z \\
   & & &     & & &       & & &         &  &  &           &  &  
  &ABET907856342&c3&j3&ACED9B7T5038264&z &z \\
   & & &     & & &       & & &         &  &  &           &  &  
  &             &  &  &CDABET907856342&c4&j4\\\hline
   & & &     & & &       & & &         &  &  &           &  &  
  &             &  &  &DBCTA0E89674523&d4&k4\\
   & & &     & & &       & & &         &  &  &           &  &  
  &BTA0E89674523&d3&k3&BTD0C8A6E492735&z &z \\
   & & &     & & &       & & &         &  &  &T0E89674523&d2&k2
  &T0B8A6E492735&d2&k2&T0B8D6C4A2E3957&z &z \\
   & & &     & & &       & & &089674523&d1&k1&08T6E492735&d1&k1
  &08T6B4A2E3957&d1&k1&08T6B4D2C3A5E79&d1&k1\\
   & & &     & & &8674523&d&k&860492735&z &z &8604T2E3957&d &k 
  &8604T2B3A5E79&d &k &8604T2B3D5C7A9E&z&z  \\
   & & &64523&e&l&6482735&e&l&648203957&e &l &648203T5E79&e &l 
  &648203T5B7A9E&e &l &648203T5B7D9CEA&e &l \\
423&f&m&42635&f&m&4263857&f&m&426385079&f &m &42638507T9E&f &m 
  &42638507T9BEA&f &m &42638507T9BEDAC&f &m \\\hline
   & & &25364&p&r&2537486&p&r&253749608&p &r &2537496E8T0&p &r
  &2537496E8A0BT&p &r &2537496E8A0CTDB&p &r \\
   & & &     & & &       & & &297058364&p1&r1&297E5T30486&p1&r1
  &297E5A3B4T608&p1&r1&297E5A3C4D6B8T0&p1&r1\\
   & & &     & & &       & & &         &  &  &           &  &  
  &2AEB9T7058364&p2&r2&2AEC9D7B5T30486&p2&r2\\
   & & &     & & &       & & &         &  &  &           &  &  
  &2TB0A8E694735&q2&s2&2TB0D8C6A4E3957&q2&r2\\
   & & &     & & &       & & &280694735&q1&s1&2806T4E3957&q1&s1
  &2806T4B3A5E79&q1&s1&2806T4B3D5C7A9E&q1&r1\\
   & & &24635&q&s&2463857&q&s&246385079&q &s &24638507T9E&q &s
  &24638507T9BEA&q &s &24638507T9BEDAC&q &r \\\hline
\end{tabular}
\caption*{} % For vertical padding

\begin{tabular}{lll| lll| lll| lll| lll| lll}
\head{Doubles}&\head{Triples}&\head{Caters}&\head{Cinques}&\head{Sextuples}&
\lasthead{Septuples}\\\hline
2453&a&g&246375&a&g&24638597&a &g &24638507E9&a &g &24638507T9AE&a &g 
  &24638507T9BECA&a &g \\
    & & &267453&b&h&26849375&b &h &268403E597&b &h &268403T5A7E9&b &h 
  &268403T5B7C9AE&b &h \\
    & & &      & & &28967453&c &j &2806E49375&z &z &2806T4A3E597&c &j 
  &2806T4B3C5A7E9&c &j \\
    & & &      & & &        &  &  &20E8967453&c1&j1&20T8A6E49375&c1&j1
  &20T8B6C4A3E597&c1&j1\\
    & & &      & & &        &  &  &          &  &  &2TA0E8967453&c2&j2
  &2TB0C8A6E49375&c2&j2\\
    & & &      & & &        &  &  &          &  &  &            &  &  
  &2BCTA0E8967453&c3&j3\\\hline
    & & &      & & &        &  &  &          &  &  &            &  &  
  &2CABET90785634&d3&k3\\
    & & &      & & &        &  &  &          &  &  &2AET90785634&d2&k2
  &2AEC9B7T503846&d2&k2\\
    & & &      & & &        &  &  &2E90785634&d1&k1&2E9A7T503846&d1&k1
  &2E9A7C5B3T4068&d1&k1\\
    & & &      & & &29785634&d &k &297E503846&z &z &297E5A3T4068&d &k 
  &297E5A3C4B6T80&d &k \\
    & & &275634&e&l&27593846&e &l &27593E4068&e &l &27593E4A6T80&e &l 
  &27593E4A6C8B0T&e &l \\
2534&f&m&253746&f&m&25374968&f &m &2537496E80&f &m &2537496E8A0T&f &m 
  &2537496E8A0CTB&f &m \\\hline
4253&p&r&426375&p&r&42638597&p &r &42638507E9&p &r &42638507T9AE&p &r 
  &42638507T9BECA&p &r \\
    & & &      & & &86947253&p1&r1&8604E29375&p1&r1&8604T2A3E597&p1&r1
  &8604T2B3C5A7E9&p1&r1\\
    & & &      & & &        &  &  &          &  &  &T0A8E6947253&p1&r1
  &T0B8C6A4E29375&p1&r1\\
    & & &      & & &        &  &  &          &  &  &EA9T70583624&q1&s1
  &EA9C7B5T302846&q1&r1\\
    & & &      & & &79583624&q1&s1&795E302846&q1&s1&795E3A2T4068&q1&s1
  &795E3A2C4B6T80&q1&r1\\
3524&q&s&352746&q&s&35274968&q &s &3527496E80&q &s &3527496E8A0T&q &s 
  &3527496E8A0CTB&q &r \\\hline
\end{tabular}}
\caption{\label{tab:mod-lhcode}%
Modern lead head codes as shown by \texttt{\$d} for methods on
four to sixteen bells.  For each lead head, the first
code is for a seconds place lead end (or thirds place for twin hunt methods), 
and the second code is for an $n$th place lead end.}
\end{sidewaystable}

The traditional lead head code,
\verb+$D+\fspec{D}, only applies on five or six bells.  It is a capital
letter, or \verb+?+ if no code applies; they are listed in 
table~\ref{tab:trad-lhcode5} for doubles%
\footnote{The codes \verb+A+--\verb+K+ (excluding \verb+I+) 
and \verb+Q+--\verb+V+ for doubles are defined in the Central Council's
1980 \textit{Doubles Collection}.%
\index{Central Council!Doubles Collection@\textit{Doubles Collection}}
These are augmented by \verb+M+, \verb+N+ and \verb+W+--\verb+Z+ for
fourths place methods which will be defined in the Central Council's 
forthcoming doubles collection.}
and table~\ref{tab:trad-lhcode6} for minor methods.%
\footnote{The codes \verb+G+--\verb+O+ (excluding \verb+I+) for regular minor 
methods are defined in the Central Council's 1961 
\textit{Collection of Minor Methods}\index{Central Council!Collection of 
Minor Methods@\textit{Collection of Minor Methods}}.  These are
augmented by \verb+P+--\verb+Y+ for irregular methods 
in Michael Foulds' \textit{Spliced Treble Bob Minor} series of books,%
\index{Foulds, Michael!Spliced Treble Bob Minor@\textit{Spliced 
Treble Bob Minor} series}.}
For minor, six possible fourths place lead heads are given Greek letters%
\footnote{These are μ, θ, δ, Σ, λ and Ψ which are defined in
Anthony S.\ Bishop's \textit{A Universal System for Extents of Treble
Dodging Minor Methods}\index{Bishop, Anthony S.!A Universal System for 
Extents of Treble Dodging Minor Methods@\textit{A Universal System for
Extents of Treble Dodging Minor Methods}}.  It is unfortunate that a more
systematic allocation of Greek lead head codes has not been adopted, but
unless and until an alternative scheme comes into use, \methsearch\ will
use this one.}
which are only displayed by \verb+$D+ if UTF-8 output is enabled with the 
\verb+-Outf8+ option.\index{UTF-8}\index{Unicode}\oi{O}

\begin{table}[ht]
\centering{\scriptsize\tt
\def\co#1{\multirow{2}{*}{#1}}
\begin{tabular}{ll@{\qquad} ll@{\qquad} ll@{\qquad} ll@{\qquad} ll@{\qquad} ll}
13254&\co{A}&14523&\co{B}&12354&\co{G}&13245&\co{H}&14325&\co{J}&12543&\co{K}\\
13524&      &14253&      &12534&      &13425&      &14235&      &12453&      \\
\\
     &      &15342&\co{C}&12435&\co{D}&12543&\co{E}&14325&\co{F}\\
     &      &13524&      &14253&      &15234&      &13452&      \\
\\
15432&\co{Q}&14523&\co{R}&12435&\co{S}&12543&\co{T}&14325&\co{U}&15342&\co{V}\\
15423&      &14532&      &12453&      &12534&      &14352&      &15324&      \\
\\
14523&\co{M}&15432&\co{N}&12435&\co{W}&14325&\co{X}&12543&\co{Y}&13542&\co{Z}\\
15423&      &14532&      &14235&      &13425&      &15243&      &13542&      \\
\end{tabular}}
\caption{\label{tab:trad-lhcode5}
Traditional lead head codes for doubles methods, as printed by \texttt{\$D},
shown with their corresponding lead ends and lead heads.}
\end{table}

\begin{table}[ht]
\centering{\scriptsize\tt
\def\firsthead#1{\multicolumn{3}{c}{\rm\small#1}}
\def\head#1{\multicolumn{2}{c}{\rm\small#1}}
\begin{tabular}{llll@{\qquad} lll@{\qquad} lll@{\qquad} lll@{\qquad} ll}
\firsthead{2nds / 6ths}&&\head{2nds}&&\head{6ths}&&\head{4ths}&&\head{4ths}\\
135264&G&L&& 142563&P&& 152364&T&& 145362&X&& 162534&Y\\
156342&H&M&& 154632&S&& 165243&V&& 136524&\rm μ&& 146325&\rm Ψ\\
164523&J&N&& 165324&R&& 146532&U&& 152643&\rm θ&& 154263&\rm Σ\\
142635&K&O&& 136245&Q&& 134625&W&& 164235&\rm δ&& 135642&\rm λ\\
\end{tabular}}
\caption{\label{tab:trad-lhcode6}
Traditional lead head codes for minor methods, as printed by \texttt{\$D}.}
\end{table}


The symmetry code, \verb+$y+\fspec{y}, is a string containing some of 
the letters \verb+P+, \verb+M+, \verb+G+ and \verb+R+ standing for
for palindromic, mirror, glide and rotation symmetry, respectively.  
See \sref{symmetry} for the definitions of these types of symmetry.  For
methods with none of these symmetries, the string is empty.\index{symmetry}

The method class string, \verb+$C+\fspec{C}, includes the tags `differential'
and `little', even though the Central Council do not consider `differential'
technically to be a class.  It does
not include `principle', `treble-dodging' or `plain' because these tags never
appear in method names.  (In the first case, an empty string is used; in 
the latter two, the classes are subdivided.)   It uses
the Central Council's current (2009) method classification%
\index{classification of methods} --- there is, at present, no support for 
older classification schemes, such as those that included the now-obsolete 
classes court\index{court methods}, imperial\index{imperial methods}, 
college\index{college methods} and exercise\index{exercise methods}.  

In the terminology of the Central Council's 
decisions\index{Central Council!decisions}, 
\verb+$N+ contains the method's \textit{title}\index{title, method} 
and \verb+$n+ contains its \textit{name}\index{name, method}.
The title, \verb+$N+\fspec{N}, includes all classes that should be 
included in the method name, even when the method is unnamed.  Unnamed methods
are assigned the name `Untitled'.  In the case of Little Bob (and only that 
case), the name, \verb+$n+, is the empty string; for all other methods,
\verb+"$N"+ is equivalent to \verb+"$n $C $S"+.

The path, \verb+$+$n$\verb+P+,\index{path}\fspec{P} contains the symbol for
each place in the path of bell $n$ during the lead (including both the initial
lead head row and the final lead head row).  For example, the path of the 2
in Plain Bob Doubles is \verb+21123455434+.  Although \verb+$P+'s value
looks like an integer, it should generally be treated as a string; for example,
in an expression, it should be compared with the \verb+eq+ operator 
(\sref{oper}) instead of \verb+==+.  This is because the \verb+==+ operator
implicitly converts its arguments to integers, which will likely result in
a silent expression error\index{expression error} when the integer overflows.

The coursing order, \verb+$O+\fspec{O}\index{coursing order}, is defined as
the place bell order\index{place bell order} raised to some power such that 
the two tenors are adjacent in the coursing order and the tenor is last.
Most of the time this gives the obvious definition --- for example, 
\verb+7532468+ for all regular major methods, but \verb+7542368+ for 
Single Canterbury.  As with \verb+$P+, the value of \verb+$O+ should generally
be treated as a string rather than an integer.

The \verb+$s+\fspec{s} variable is a measure of how static the method is.
\index{staticity}
Each consecutive blow that a pair of bells spends together beyond the first 
scores one point.  This means that a point scores one point, a dodge two, an $n$-pull dodge $2n$ and so on. For example, Plain Bob Minor scores six points 
--- two each for the 3--4 and 5--6 dodges, and two further for the seconds made
over the treble which lasts as long as a dodge.  By this measure, the most 
static plain method on all even stages is Double Oxford (jointly with 
a huge number of variations thereof); and unsurprisingly, Derwent Surprise
Major is the most static treble dodging major method with a score of 104.  

Ignoring falseness,
symmetry, treble path and other such constraints, the static index describes
the number of trivial variants\index{trivial variants} that exist for a method
--- a method with staticity $s$ has $2^s$ variants.  
(In case of Plain Bob, the dodges can be replaced with 
Reverse Canterbury--style places\index{Reverse Canterbury places},
or with three blows and a hunt or \textit{vice versa} giving four 
possibilities in each dodging position; 
similarly the seconds over the treble could be replaced with four blows over
each other, or with a point and a place or \textit{vice versa}.  This gives
$2^6=64$ possibilities.  Note that variants that alter the direction of bells
leaving a piece of work are not counted --- thus the difference between
Kent\index{Kent Treble Bob} and Oxford\index{Oxford Treble Bob} is not 
considered a trivial variant for this purpose.)

The method identifier, \verb+$i+, is read from the method libraries.  
In the Central Council's libraries it is the column headed `CCC';  
it is the method's number in the printed method collections.  It is of
most use in plain doubles where these identifiers have become the standard 
shorthand for referring to doubles methods.  Note that the lead head code
is not included in this --- thus to get 131B for Reverse Canterbury, the
format string \verb+$i$D+ should be used.

The method number, \verb+$#+, is simply a counter that is incremented for 
each method that is found.  (Note that methods discarded using the 
\verb+suppress+ keyword (\sref{exprkey})%
\index{suppress keyword@{\texttt{suppress} keyword}}
still get a number; this is the principle way in which \verb+suppress+
differs from \verb+-Q+.)

\subsection{Character escape sequences}\label{charesc}
\index{character escape sequence|(}

Format strings may also include \textit{character escape sequences}.  
These provide ways to include various \textitidx{control characters} in 
\methsearch's output.  Far and away the most common use is to include
\textit{tab characters}\index{tab character} in the output.  This is useful
for producing data that a spreadsheet\index{spreadsheet} can read
and is discussed in \sref{eg_doubles}.
The full list of character escape sequences is below,
together with their ASCII\index{ASCII} values and meanings. 

\begin{tabular}{llll}
\verb+\a+& $0$x$07$& BEL& alert (or bell) character\\
\verb+\b+& $0$x$08$& BS&  backspace character\\
\verb+\f+& $0$x$0$c& FF&  form feed character\\
\verb+\n+& $0$x$0$a& LF&  line feed character\\
\verb+\r+& $0$x$0$d& CR&  carriage return character\\
\verb+\t+& $0$x$09$& HT&  (horizontal) tab character\\
\verb+\v+& $0$x$0$b& VT&  vertical tab character\\
\verb+\x+\textit{NN}  &$0$x\textit{NN}&
  &arbitrary character by hexadecimal character number\\
\verb+\\+& $0$x$5$c&   &  a literal \verb+\+ character\\
\verb+\$+& $0$x$24$&   &  a literal \verb+$+ character\\
\verb+\%+& $0$x$25$&   &  a literal \verb+%+ character\\
\end{tabular}

A \verb+\+ at the very end of the format string end of the format 
string is used to suppress the line feed that is implicitly
added to the end of a format string.

To print information spanning multiple lines, almost invariably the
line feed (\verb+\n+)\index{line feed} should be used instead of the 
carriage return (\verb+\r+).\index{carriage return}%
\footnote{On systems, such as Windows, where the standard
line ending comprises a CR-LF pair,\index{Windows!line endings}
\methsearch\ automatically inserts a CR (\verb+\r+) before each LF (\verb+\n+) 
that it encounters.}  For example, \verb+-R"$q\n\t$N"+ would print the 
place notation on one line followed by a line containing the name 
indented by one tab stop.

As a backslash\index{backslash}%
\symidx{\textbackslash}{character escape sequence} 
introduces a character escape sequence, 
should you actually want to print a backslash, you need to double-up the 
backslashes.  For example, to format a table in \LaTeX\index{Latex@\LaTeX}, 
columns are separated by an \verb+&+ and rows are ended with two backslashes 
--- \verb+\\+.  
To get \methsearch\ to produce output ready for \LaTeX, you can use
something like \verb+-R"$q&$l&$N\\\\"+.  (As always, Linux users would use
single quotes instead of double quotes.)

For similar reasons, to include a literal \verb+$+ or \verb+%+ in the 
output, these should be escaped with a backslash too.%
\footnote{In fact, a \verb+$+ can be escaped with another \verb+$+ or 
with a \verb+%+ too.  Similarly with \verb+%+.  In other words, \verb+\$+,
\verb+$$+ and \verb+%$+ all produce the same output --- a single \verb+$+.
All combinations involving a \verb+%+ (other than \verb+\%+) are deprecated.}

The bell character, backspace character\index{backspace}, 
form feed\index{form feed}, carriage return and vertical tab are unlikely to be
of use, but are included for completeness.  (It might be tempting to try to 
use \verb+\a+ to get alerted whenever a particularly good method is found.
In practice, most modern terminals don't act on receiving a \verb+\a+, and
even if they did, it's easy to miss the alert.)

Hexadecimal escapes can be used to access additional functionality that
your terminal might support.  For example, most modern Unix-like terminals
support VT102\index{VT102} terminal escapes allowing coloured output%
\index{coloured output}, underlining, and many other effects.%
\footnote{Windows users wanting a terminal that supports such things
might like to try Cygwin,\index{Cygwin} a Unix emulation platform for Windows, 
which can be downloaded from \url{http://www.cygwin.com/}.}
This can be combined with \methsearch's other features in quite sophisticated
ways.  For example, the following command looks for musical unrung 
surprise major methods, and highlight the most musical ones in bold red:%
\index{example!major, musical unrung surprise}%
\footnote{The backslash at the very end of the line is the shell's 
\textit{line continuation character}.\index{shell!line continuation}  If you 
type the whole command on a single line, you should ignore this backslash.
Users of Unix-like shells also have the option of including it.}

\begin{Verbatim}[commandchars=@~~]
methsearch -b8 -SG1 -srefj -p2 -M'<4-runs>'      @hfill\
  -R'$[$M>35?"\x1b[1;31m":""]$2#)  $q \x1b[38G$d [$F]\x1b[0m' @hfill\
  -Q'$M>30 && $n eq "Untitled"' -Lsurp8.txt
\end{Verbatim}

Understanding complex examples like this is typically harder than
writing them.  The first part of the \verb+-R+ option is 
\verb+$[ $M>35 ? "\x1b[1;31m" : "" ]+.  This is an expression saying 
if the music score is greater than 35, 
then print the string `\verb"\x1b[1;31m"'.  Here, \verb+\x1b+
is the ASCII \textitidx{escape character}, and the whole string means
put the console into bold mode and set the 
foreground colour to red.  Similarly, the `\verb"\x1b[0m"'
at the end of the format string sets the console back to its default mode;
`\verb+\x1b[38G+' advances the cursor to column 38, in order to get nice
alignment.\footnote{The console codes\index{console codes} available
depends on your system, but some 
documentation can be found at
\url{http://www.kernel.org/doc/man-pages/online/pages/man4/%
console_codes.4.html}.}

At the time of writing, this example produces 21 interesting unrung methods.
Many suffer from unfortunate incidence of O falseness, but \#17 has an 
interesting line, a plain course with quite diverse music, and sane falseness:
\begin{Verbatim}
&-3-4.5-5.36.4-4.5.4.36.4.3,1
\end{Verbatim}
%\includegraphics[width=\textwidth]{unrungmajor.eps}
\index{character escape sequence|)}

For a mixture of historical reasons and niche use-cases, there are a few 
character escape sequences that are introduced by a \verb+$+ (or a \verb+%+,
though that is deprecated).  They are \verb+$$+, \verb+$%+, \verb+$)+ which
are alternative ways of producing a literal \verb+$+, \verb+%+ or \verb+)+.

\section{Expressions}\label{expr}
\index{expression|(}

The \verb+-Q+\oi{Q} option provides a way of specifying more 
complex requirements that methods must satisfy.  Its argument is an 
\textit{expression} which must, when evaluated and converted to a boolean, 
must be true in order that the method is found.  For example, 
\verb+-Q"$M >= 40"+ only prints methods where the music score 
(\verb+$M+)\fspec{M} is at least 40. 

Expressions can also be included in format strings by including them
in \verb+$[+\ldots\verb+]+.  In this case, the expression is evaluated and
its value printed.  For example, to normalise the music of
methods with different length plain courses, the
average music score per lead in the plain course could be printed with
\verb+-R"$[$M / $o]"+.\footnote{But note that because the \verb+/+ operator
performs integer division, this will be rounded to the nearest integer.}

Expressions are built up from operators (\sref{oper}) which can act 
on literals (\sref{lits}), variables (\sref{variables}) and 
command invocations (\sref{cmdinv}).  Finally there are a handful of 
special keywords that may be used in certain specific contexts 
(\sref{exprkey}).

\subsection{Operators}\label{oper}

Valid expressions are formed from the following operators\index{operators|(} 
which are tabulated in precedence\index{operators!precedence|(}%
\index{precedence|see{operators, precedence}} order, 
from highest precedence (at the top of the list) through to lowest 
precedence (at the bottom).  Entries on the same line have equal precedence.
The associativity\index{operators!associativity|(}%
\index{associativity|see{operators, associativity}} of each precedence level 
is marked in the second column.

\begin{tabular}{l@{\quad}l@{\quad}l}
\verb+*+ \verb+/+ \verb+%+& left& multiplication-type operators\\
\verb-+- \verb+-+ \verb+.+& left& addition-type operators\\
\verb+<+ \verb+>+ \verb+<=+ \verb+>=+ \verb+lt+ \verb+gt+ \verb+le+ \verb+ge+&
                            left& comparison operators\\
\verb+==+ \verb+!=+ \verb+eq+ \verb+ne+&
                            left& equality comparison operators\\
\verb+~~+&                  left& pattern matching\\
\verb+&&+&                  left& logical and\\
\verb+||+&                  left& logical or\\
\verb+?:+&                  right& the conditional (ternary) operator\\
\verb+,+&                   left& the comma (sequence) operator\\
\end{tabular}

In expression involving two \textit{binary operators}\index{operators!binary}
(that is, operators taking two arguments), for example, \verb-1+2*3-, 
the order of invocation of the two operators can be made explicit by the use
of parentheses.\index{parentheses}  For example, \verb-1+(2*3)- or 
\verb-(1+2)*3-.  In the absence of parentheses, the operators' 
\textit{precedence} determines how tightly operators bind, relative to each 
other.  In this example, \verb-1+2*3- is equivalent to \verb-1+(2*3)- because
\verb+*+ has higher precedence than \verb-+-.\index{operators!precedence|)}

When an parenthesesless expression involves two operators of equal 
precedence, the order of their invocation is determined by that 
precedence level's \textit{associativity}.  If the level is left-associative%
\index{left-associative|see{operators, associativity}}, 
the operators are evaluated from left to right; and if the level is 
right-associative\index{right-associative|see{operator, associativity}}, they 
are evaluated right-to-left.  So, for example, \verb+12/2*3+ evaluates to 18,
and not 2.\index{operators!associativity|)}

The operators \verb-+-, \verb+-+, \verb+*+ and \verb+/+ fulfil their
usual mathematical roles of addition\index{addition}, 
subtraction\index{subtraction}, multiplication\index{multiplication}
and division\index{division}, respectively, with the caveat that they
perform \textitidx{integer arithmetic}.  This is of particular relevance
to division where the result is rounded to the next integer towards zero ---
so, \verb+5/3+ evaluates to 1.

The \verb+%+ operator is the modulus operator\index{modulus operator}.
It returns the remainder left from the integer division of its arguments.
For example, \verb+8%3+ evaluates to 2.  It can be defined more mathematically
in terms of integer division by the equation:
\[ a \mathop{\verb+/+} b \mathop{\verb+*+} b 
   \;\mathop{\verb-+-}\; a \mathop{\verb+%+} b = a \]
Because integer division rounds towards zero rather than down, this means that
when $a$ is negative, the result of the modulus operator is also negative.%
\footnote{To those familiar with modular arithmetic, this may be surprising.
But this is behaviour is typical in most programming languages that use
integer arithmetic --- for example, the C language, and languages derived
from it, define the \verb+%+ operator in this way.}

The \verb+.+ operator does string concatenation.\index{concatenation}  
It simply returns its two arguments joined together.  It is more commonly
used on strings, but can also be used on integers too --- for example,
\verb+1.1+ evaluates to 11.  (As \methsearch\ does not support floating 
point numbers\index{floating point numbers} or any other form of non-integral
numbers, there is no ambiguity between the the concatenation operator and 
the decimal point.\footnote{A future version of \methsearch\ might introduce
place notation literals, but these will require \verb-+- or \verb+&+ prefixes.
This may result in some obscure incompatibilities, e.g. \verb-3+3.1- currently
evaluates to 93, but a future version may treat it as a place notation 
\verb+3.1.3.1.3.1+.  Inserting whitespace after the \verb-+-, or around the
\verb+.+ will force parsing as an integer.})

The six arithmetic comparison operators,\index{comparison operator} 
\verb+<+, \verb+>+, \verb+<=+, \verb+>=+, \verb+==+ and \verb+!=+ 
do an arithmetic comparison\index{arithmetic comparison} of two integers 
and evaluate to 1 (for true) or 0 (for false).   The operators 
\verb+lt+, \verb+gt+, \verb+le+, \verb+ge+, \verb+eq+ and \verb+ne+
are the equivalent forms that do lexicographic comparisons\index{lexicographic
comparison} of two strings.  This means that while \verb+11>2+ is true,
\verb+11 gt 2+ is false, as the two integers are simply treated as strings.

The \verb+~~+ operator does pattern matching on rows.  The left hand argument 
is parsed as a row, and the right hand argument is parsed as a
music pattern (\sref{muspattern}).  If the row matches the pattern, the
expression evaluates to \verb+1+; if not, it evaluates to \verb+0+.  If
the arguments are not well-formed as a row an a pattern, respectively, an
expression error (\sref{lits}) occurs.  For example, \verb+$l ~~ "12*"+ can
be used to filter slow course methods.\index{slow course methods}

The \verb+&&+ and \verb+||+ operators are the logical `and' and logical `or'
operators.\index{logical and}\index{logical or}  
In the former case, it evaluates to true if both its arguments 
evaluate to true; and in the latter case, it evaluates to true if either 
(or both) of its arguments evaluate to true.  Typically, though not always, 
the arguments will be comparisons, for example \texttt{\$o>=3 || \$u==1}.
Both of these operators evaluate their left-hand argument first and 
\textit{short circuit}\index{operators!short circuiting} 
based on the value of it --- this means that if
the result of the operator does not depend on value of the second argument,
then the second argument is not evaluated.  In practice this means that if
the first argument to \verb+||+ is true or if the 
the first argument to \verb+&&+ is false, then the second is not evaluated.
This is of particular importance when one of the arguments is a command
invocation (\sref{cmdinv}), the evaluation of which is potentially very
expensive. In such an example, by reordering the expression such that the
command invocation is on right, it is often possible to get a significant
speed improvement.

The penultimate operator is the \textitidx{conditional operator}.  It is a 
\textit{ternary operator}\index{operators!ternary} --- 
that is, an operator which takes three arguments.%
\footnote{Because it is the only ternary operator in most
programming languages, the term `conditional operator' is often treated
as synonymous with `ternary operator'.}  Its syntactic form is 
$a$ \verb+?+ $b$ \verb+:+ $c$, where $a$, $b$ and $c$ are its arguments.
The conditional operator evaluates its first argument.  If the first
argument evaluated to true, the second argument is evaluated and the
value of the whole conditional operator expression is the value of the 
second argument.  And if the first argument evaluated to false, the third
argument is evaluated and the expression takes its value.  The conditional
operator never evaluates both its arguments.  An example is 
\verb+$M>=40 ? "*" : ""+ which might be used to print an asterisk next
to particularly musical methods.

Finally, the \verb+,+ operator\index{operators!comma}
evaluates its first argument and ignores its
value.  It then unconditionally evaluates its second argument, and the 
argument of the whole expression is that of its second argument.  This
is only of use when evaluating the first argument has some effect, for example,
evaluating a \verb+suppress+ or \verb+abort+ keyword (\sref{exprkey}) or
invoking a command (\sref{cmdinv}).

Except for the conditional operator, all of \methsearch's supported operators
are \textit{binary operators}\index{operators!binary} 
--- i.e.\ they take two operators.  
\methsearch\ does not currently support any unary operators,%
\index{operators!unary, lack thereof} which results in
some notable omissions.   In particular, there is no unary 
minus\index{minus, unary, lack thereof} --- if you want to use a negative 
number you need to use a circumlocution such as \verb+0-5+.  
Similarly there is no logical `not' operator\index{logical not, lack thereof}.%
\footnote{This can also be worked around with the expression 
\verb+1-(+$x$\verb+)+, which exploits the fact that true and false are
represented as 1 and 0, respectively.}
This is likely to be addressed in a future version of \methsearch.

\index{operators|)}

\subsection{Literals}\label{lits}

The simplest expressions are \textitidx{literals} which come in two forms:
integer literals\index{integer literals} and string literals\index{string 
literal}.  Integers are just written in the normal way using the digits
\verb+0+--\verb+9+;\footnote{Neither hexadecimal nor octal integer literals
are supported --- in particular, \verb+010+ is the decimal number 10 written
with an unnecessary leading zero, and not the decimal number 8 written in 
octal.}  strings are enclosed in single or double quotes, e.g. 
\verb+"Untitled"+ or \verb+'Untitled'+.

The use of quotes for string literals potentially conflicts with
the use of quotes for quoting a command line option from the 
shell.\index{shell!quotation}  Generally this is not a 
problem because the shell quotation can be done with one type of quotes,
and the literal quoted with the other sort.  
However, there are circumstances when this luxury is not available, 
for example, because the whole shell command
line is itself quoted in single quotes, and you need to use double quotes for 
both.  To solve this, the inner two
need escaping \index{escaping|see shell!escaping}\index{shell!escaping} 
from the shell.  This is done by prefixing them with 
a backslash\index{backslash}: 
\verb+-Q"$n ne \"Untitled\""+.

\subsection{Type system}\label{type_sys}\index{type system}

Expressions in \methsearch\ are \textit{weakly typed}\index{weak typing}. 
This means that strings are integers are converted freely as required.
When an integer is converted to a string, the string value is simply the
decimal representation of the integer with no leading zeros.  
Strings can be converted to integers too.  If the string contains just
digits \verb+0+--\verb+9+ then the conversion occurs successfully;%
\footnote{In fact, a leading \verb-+- or \verb+-+ is permitted too.  
This is another way of getting around the lack of a unary minus operator.%
\index{minus, unary, lack thereof}  In builds of \methsearch\ on 32-bit
systems, integer value of the string, $i$, must satisfy 
$-2^{31} \le i \le 2^{31}-1$.  
A similar constraint with $2^{63}$ applies on 64-bit systems.}
otherwise an \textitidx{expression error} occurs.  
This means that no further evaluation
of the whole expression occurs; if the expression was in a format string,
its value is set to \verb+<ERROR>+; and if it was in a \verb+-Q+ option,
it is considered to have evaluated to false.  It is not possible to recover
from an expression error within the expression.  For example, 
\verb+"a"-1 > 42 ? "b" : "c"+ does not evaluate to either 
\verb+"b"+ or \verb+"c"+.
An expression error also occurs when division by zero is attempted,%
\index{division!by zero} for example in \verb+1/0+.

There is no specific boolean type\index{boolean type} --- any non-zero
integer is considered to mean true.  The comparison operators and logical 
operators all return an integer which is either \verb+0+ (for false) or 
\verb+1+ (for true).

All variables (and command invocations) are considered to be strings,
even those which are logically integers (such as \verb+$o+).  However,
as strings are freely convertible to integers, this is rarely detectable.%
\footnote{This allows \methsearch\ to treat \verb+$c+\fspec{c} and \verb+$#+
as unsigned 64-bit integers on many 32-bit systems.  On such systems, 
in an extremely long search where these might exceed $2^{31}$, 
\verb+$#+ will succeed where \verb+$[$#]+ (which forces a conversion to a
32-bit signed integer) will fail.}
\methsearch\ does not currently have specific types representing rows or
changes.  This is likely to change in a future version of \methsearch, 
as it is likely that the expression mechanism described in this section will
be unified with the row expression\index{row expression} mechanism of 
\sref{rowexpr}.  An upshot of this is that it is not possible to do row-level
operations, such as permutation, within an expression.

\subsection{Special keywords}\label{exprkey}\index{keyword}

\begin{tabular}{ll}
\verb+suppress+& Suppress the current method\\
\verb+abort+&    Abort the whole search\\
\end{tabular}

If the \verb+suppress+\index{suppress keyword@{\texttt{suppress} keyword}}
is evaluated in an expression, then display of that method is suppressed. 
This keyword is different in number of ways to the \verb+-Q+
option --- methods that are suppressed still have a \verb+$#+ number assigned,
and are still included in any statistics (\sref{stats}) and counted in the
\verb+-C+\oi{C} method count.
If the \verb+abort+\index{abort keyword@{\texttt{abort} keyword}} 
is evaluated in an expression, then the method is not printed and the
whole search is aborted.

In order to prevent these keywords from being evaluated unconditionally,
they need to be used in conjunction with an operator that does short circuiting%
\index{operators!short circuiting} --- namely, \verb+&&+, \verb+||+ or 
\verb+?:+.
%
\index{expression|)}

\section{Statistical output}\label{stats}
\index{statistics|(}

The \verb+-H+ option\oid{H}{frequencies} provides a mechanism for gathering 
simple statistical information on methods.  
The option's argument is a format string 
(\sref{fmtstr}).  For each method that is found, the format string is expanded,
any variables (\sref{variables}) except \verb+$c+\fspec{c}, 
expressions (\sref{expr}) or command invocations (\sref{cmdinv}) are evaluated 
and their values substituted.  The format string value is then stored
and a count kept of how many times that string occurs.  At the end search,
the \verb+$c+ variable gets replaced with the number of times that string 
occurs.

A simple example of this is counting the number of methods of each class
within the `standard' 147 trebled dodging minor methods.%
\index{example!minor, treble-dodging, statistics|(}

\begin{Verbatim}
methsearch -b6 -PG1 -srf -p2 -l2 -qH'$2c $C'
\end{Verbatim}

(Windows users should use double quotes rather than single quotes.)
This produces the output:

\begin{Verbatim}
77 Delight
41 Surprise
29 Treble Bob
\end{Verbatim}

Certain variables --- namely \verb+$p+, \verb+$q+,
\verb+$Q+, \verb+$n+, \verb+$N+, \verb+$i+, \verb+$T+ and \verb+$#+ --- 
cannot be included directly in statistical output.  
This is not a technical restriction \textit{per se}, rather that as 
place notations are unique to a method, and names nearly so, it would rarely
serve any purpose, and as the whole statistical output has to be stored
in memory until the search completes, it would consume a large amount of
memory and possibly result in \methsearch\ running out of memory and crashing.
These variables can all be referenced in expressions,%
\footnote{This gives you a way of circumventing the restriction on placing
these variables directly in statistical output: \verb+-H"$[$n]"+.  But 
don't do this.}
for example, to see
how many of the 2400 treble dodging minor methods in the Central Council's
2008 \textit{Treble Dodging Minor Methods}\index{Central Council!Treble
Dodging Minor Methods@\textit{Treble Dodging Minor Methods}}
are still unnamed, you could run:

\begin{Verbatim}[commandchars=@<>,commentchar=~]
methsearch -b6 -PG1 -sp2 -Lsurp6.txt -Ldel6.txt -Ltb6.txt @hfill\
  -qH'$4c $[$n eq "Untitled" ? "Unnamed" : "Named"]' 
~ -- a comment to get vim's syntax highlighter back in sync:  $
\end{Verbatim}
(Again, Windows users should use double quotes, and also escape the inner
double quotes using a backslash.\index{shell!escaping})
At the time of writing, this produces the output:
\begin{Verbatim}
1728 Named
 672 Unnamed
\end{Verbatim}
\index{example!minor, treble-dodging, statistics|)}

\methsearch's ability to do useful statistical analysis of methods is 
quite limited.  In most cases, it is likely that you will want to either
use (or at least combine \methsearch's statistical abilities with those of) 
some other package, such as a spreadsheet\index{spreadsheet} or standard 
command line utilities such as \verb+awk+\index{awk@\texttt{awk}}, 
\verb+sort+\index{sort@\texttt{sort}} or \verb+grep+\index{grep@\texttt{grep}}.
This is consistent with the Unix philosophy outlined in \sref{unixphil} ---
specifically that \methsearch\ should do one thing and do it well.  That thing
is searching for methods; plenty of other programs exist that can do
interesting statistical analysis, and \methsearch\ aims to make interfacing
with them as easy as possible rather than attempting to subsume their 
functionality.

\index{statistics|)}


\index{output|)}

\chapter{Falseness}\label{falseness}
\index{falseness|(}

By default \methsearch\ only finds methods that are true within a lead.
The \verb+-F+ option\oid{F}{falseness} provides a way of changing this 
behaviour.  The basic options governing whether the method should have 
a true plain course, true leads, etc., are documented in \sref{basic_false}.
Analysis of false course heads is discussed in \sref{fch}, and ways of 
checking that an extent is possible is covered in \sref{extent}.
The final two sections (\sref{avoidrow}--\ref{pends}) discuss using 
\methsearch\ to find methods that are true against other methods.


\section{Basic falseness levels}\label{basic_false}

\methsearch\ supports five basic levels of falseness checking wich 
can be specified by a single-letter argument to the \verb+-F+ option.  
Longer, more descriptive, synonyms are also provided.

\begin{tabular}{lll}
\verb+-Fc+& \verb+-Fcourse+&      Require a true plain course\oidF{c}{course}\\
\verb+-Fl+& \verb+-Flead+&        Only require truth within a lead [default]\\
\verb+-Fh+& \verb+-Fhalf-lead+&   Only require truth within each half-lead\\
\verb+-Fn+& \verb+-Fnone+&        Do not check falseness\\
\verb+-Fx+& \verb+-Freally-none+& Allow even trivially false methods\\
\end{tabular}

Although \verb+-Fn+ option\oidF{n}{none} does not check the falseness of 
the method, two forms of falseness are still excluded.  These are referred to
as \textit{trivial falsenesses}\index{trivial falseness}. 
The \textitidx{null change} (that is, the change in which no bells move) 
is still not permitted, nor is the immediate repetition of change.  
To allow these trivial falsenesses, the \verb+-Fx+ option\oidF{x}{really-none}
should be used.

When checking a lead with \verb+-Fl+,\oidF{l}{lead} \methsearch\ proves all 
the rows between the opening rounds and the lead end row inclusive --- it 
also requires that the lead head is either distinct, or the same as the 
initial lead head.
Similarly, \verb+-Fh+\oidF{h}{half-lead} proves all the rows from the opening 
rounds to the half lead end, and separately from the half lead head to the 
lead end.  (In practice, if either \verb+-s+ or 
\verb+-d+ is specified, only one half-lead is tested --- the truth of the
other is guaranteed by symmetry.)  With \verb+-Fh+, the half lead head row must
either be distinct from all rows in the first half lead, or the same as the 
half lead end row.

\methsearch\ contains a number of optimisations to speed up proving falseness
in the most common cases.  For example, it knows that in a palindromic 
plain method, there is no possibility of internal falseness.

\section{False course heads}\label{fch}
\subsection{A digression — the theory of falseness}%
\label{falsetheo}\index{falseness!theory of|(}

If a course of a method starting from course head, $r$, is false against the 
plain course, then $r$ is known as a \textitidx{false course head}.  This 
can be explored mathematically.  Throughout this section I shall use the
convention that $xy$ means $x$ transposed by $y$, rather than the $x$ 
transposition operating on $y$.  (Neither convention has universal uptake
amongst ringing theoreticians, although the one used here seems increasingly
popular.)

Let $C$ denote the set of rows in the plain course of a method, 
and define the product of a row, $r$, and set of rows, $C$ to mean 
the set containing $r$ multiplied by each member of $C$:
\[ rC = \{ r x : x \in C \}. \]
Thus $rC$ is set of rows in the course starting from course head, $r$.
If $C$ and $rC$ contain a row (or more) in common, then $r$ is a false 
course head as defined above.  Introducing the notation $F(C)$ for the 
\textitidx{falseness set} 
--- the set of all false course heads of the method $C$ --- 
this can be written mathematically,
\[ C \cap rC \ne \emptyset \iff r \in F(C). \]
After some manipulation, this can be written as a closed formula for $F(C)$,
\[ F(C) = \{ a b^{-1} : a, b \in C \}. \]
It can be seen from this definition that if $r$ is a false course head then
so too is $r^{-1}$.  (This is because $(a b^{-1})^{-1} = b a^{-1}$ and we
are free to relabel $a$ and $b$.)

It can be useful to introduce notation to refer to the set of 
inverse elements\index{set!of inverses}
\[ \bar{C} = \{ a^{-1} : a \in C \}; \]
the multiplication of two sets can also be defined in the obvious way 
--- as the set composed from each element of the first set multiplied by each 
element of the second set:\index{set!multiplication}
\[ AB = \{ a b : a \in A, b \in B \}. \]
Using this notation, the falseness set is simply $F(C) = C\bar{C}$.

As most methods have internal structure by virtue of being palindromic%
\index{symmetry!palindromic} and being composed from identical leads, 
the falseness set also has structure.  
Consider a palindromic method with lead end row, $e$, and lead end change, $h$.
The set, $L$, of lead heads and lead ends of the method is the 
dihedral group\index{group!dihedral, $D_n$} generated by these two elements.
\[ L = \langle e, h \rangle 
     = \{ 1,\, e,\, eh,\, ehe,\, eheh,\, ehehe,\, \ldots \}. \]

The rows in the plain course of the method can thus be expressed in terms of
the rows, $H$, in the first half lead of the methods, $C = LH$, thus allowing
the falseness set to be expressed as 
\[ F(C) = L H \bar{H} \bar{L} = L\; F(H)\; L. \]
(Note that $L=\bar{L}$ because $L$ is a group.%
\footnote{Specifically a set of rows, $X$, is a group if and only if
$X=\bar{X}$ and $XX=X$.\index{group!definition}  
These follow from the inverse element and closure group axioms.})
This partitions the falseness set (and the 
set of all possible course heads) into \textitidx{double cosets} of $L$:
if $r$ is a false course head, then so are all of $LrL$.  However, it has
already been noted that if $r$ is a false course head, so is $r^{-1}$.  
Combining these two results pairs certain double cosets such that if
one occurs so must the other.  These sets have the form $L\{r,r^{-1}\}L$ 
and are known as \textit{falseness groups}\index{falseness groups|(}.%
\footnote{This term is a misnomer as falseness groups are not groups 
in the mathematical sense.}

\index{falseness!theory of|)}

\subsection{Falseness groups}

The falseness groups for a method are useful because they can be used to
locate compositions that are true to that method.  A composition is described
as \textitidx{universally true} to certain falseness groups if it is 
necessarily true to all methods having at most those falseness groups.%
\footnote{The advent of computer programs that allow people to search for 
compositions particularly tailored to a specific method has somewhat 
reduced the use  for falseness groups.  
However not everyone has access to or the inclination
to use such programs, and collections of universal compositions are still
published; for example, the Central Council's 2001 
\textit{Collection of Universal Compositions of Treble Dodging Major Methods}.%
\index{Central Council!Collection of Universal Compositions of Treble Dodging 
Major Methods@\textit{Collection of Universal Compositions of Treble Dodging 
Major Methods}}}

For compositions that are \textitidx{tenors-together} (generally defined as
meaning bells 7 and higher remain coursing throughout), it is only necessary
to look at falseness groups that include tenors-together course heads;
similarly, a bobs-only composition only requires falseness groups including
in-course\index{in-course|see{parity}}\index{parity} 
course heads to be considered.

On eight bells (assuming a fixed treble and a seven-lead course), 
there are 28 such falseness groups of which 25 include tenors-together courses.
On higher stages the total number dramatically increases, but the number
with tenors-together courses does not significantly and are related to eight
bell ones.
For regular major methods, these have 
conventional names using the letters \verb+A+--\verb+U+ (except \verb+J+
and \verb+Q+) to denote the tenors-together groups containing in-course rows,
\verb+a+--\verb+f+ for out-of-course\index{out-of-course|see{parity}} 
tenors-together groups and \verb+X+, \verb+Y+ and \verb+Z+ for the three 
remaining groups which have no tenors-together rows.  

On ten or more bells, certain tenors-together groups split.  This is 
indicated by a numeric suffix on the falseness group code.  The only 
difference between falseness groups on ten bells and higher stages is that 
on ten, the \verb+D1+ and \verb+B1+ groups combine.  Table~\ref{tab:fchs}
gives all of the tenors-together false course heads and their corresponding 
falseness group.

\begin{table}[h]\centering
{\tt\tiny\begin{tabular}{ll ll ll ll ll ll ll ll}
23456 A&25463 M&34256 U1&36452 T1&45236 F1&52463 e&56234 R&63452 D2\\
23465 a1&25634 C1&34265 C2&36524 O1&45263 T2&52634 N3&56243 H2&63524 O2\\
23546 a2&25643 b&34526 K2&36542 H2&45326 d&52643 O1&56324 f&63542 G\\
23564 M&26345 F2&34562 N1&42356 U1&45362 S&53246 T1&56342 P2&64235 S\\
23645 M&26354 T1&34625 S&42365 C2&45623 R&53264 N2&56423 G&64253 K2\\
23654 B2&26435 M&34652 d&42536 c&45632 N3&53426 D2&56432 O2&64325 K3\\
24356 D2&26453 a2&35246 c&42563 L2&46235 T2&53462 H1&62345 N1&64352 P1\\
24365 B1&26534 b&35264 U2&42635 U2&46253 D1&53624 K1&62354 d&64523 P2\\
24536 T1&26543 L1&35426 U1&42653 c&46325 N1&53642 O2&62435 N2&64532 f\\
24563 F2&32456 B2&35462 N2&43256 B2&46352 K2&54236 d&62453 T1&65234 N3\\
24635 E2&32465 E1&35624 N3&43265 E1&46523 H2&54263 N1&62534 R&65243 O1\\
24653 T1&32546 D1&35642 R&43526 T1&46532 O1&54326 P1&62543 H2&65324 I\\
25346 T1&32564 T2&36245 L2&43562 e&52346 K2&54362 K3&63245 e&65342 f\\
25364 E2&32645 T2&36254 c&43625 N2&52364 S&54623 f&63254 U1&65423 O2\\
25436 B2&32654 F1&36425 e&43652 U1&52436 U1&54632 I&63425 H1&65432 K1\\
\end{tabular}}
\caption{\label{tab:fchs}%
False course heads and their corresponding falseness group.}
\end{table}

Although it is possible to generalise these falseness groups codes to 
irregular methods, \methsearch\ does not currently support it.

The \verb+A+ falseness group (that is, the one including \verb+23456+ as a
course head) can be interpreted in different ways.%
\index{A falseness@\texttt{A} falseness}  It is sometimes
taken as the identity falseness, in which all methods have it; and it 
is sometimes taken to mean that the method is internally false within the
plain course.  \methsearch\ uses the former definition.

\subsection{Restricting falseness}\label{falserest}

The \verb+$F+\fspec{F} method variable prints all the falseness groups of
a method including \verb+A+, any out-of-course groups, and, on eight bells 
only, the three split-tenors groups.  For example, for Cambridge Surprise 
Major, it will print \verb+ABDEe+.

\methsearch\ allows you to restrict the search to methods with (at most)
certain falseness groups.  This is done with the \verb+-F+ option.\oi{F}
The syntax is \verb+-F:+\textit{groups}.%
\oidx{F:groups}{\texttt{F:}\textit{groups}}
A `\verb+-+' may also be used to
allow ranges for falseness groups.  The \verb+A+ group is implicitly included.
For example, on eight bells, \verb+-F:BDEa-f+ finds methods with only 
\verb+B+, \verb+D+ and \verb+E+ falseness in-course, 
but arbitrary out-of-course groups.  It does not, however, find any major 
methods with \verb+X+, \verb+Y+ or \verb+Z+ falseness unless these are
explicitly allowed.

\methsearch\ has a short hand notation for searching for \textitidx{clear
proof scale} methods.\index{CPS|see{clear proof scale}}
These are methods with no in-course tenors-together
falseness, but allowing arbitrary other falseness.  This is written
\verb+-FCPS+\oidx{FCPS}{\texttt{FCPS}}\oi{F} and on eight bells is 
equivalent to \verb+-F:a-fXYZ+.  (It is a fairly common misconception that
clear proof scale methods and only clear proof scale methods can produce an 
extent.  Neither is, in general, the case.)

\index{falseness groups|)}

\section{Extent viability}\label{extent}\index{extents, viability of|(}

On lower stages (particularly on seven or fewer bells) it is common to
want methods which are capable of producing an extent.  Testing this is, 
in general, a difficult problem.  However, one prerequisite is that 
a large enough set of mutually true leads exist.  \methsearch\ is able
to check for certain special cases of this.

The \verb+-Fe+ option,\oi{F} also spelt \verb+-Fextent+,\oidF{e}{extent}
checks the $(n-1)!$ possible fixed-treble lead heads on $n$ bells to see 
whether a mutually true set of $\half(n-1)!$ leads exists.
The \verb^-Fe+^ option, also spelt \verb+-Fpositive-extent+,%
\oidF{e+}{positive-extent} checks the $\half(n-1)!$ possible in-course 
fixed-treble lead heads to see whether a mutually true set of 
$\quarter(n-1)!$ leads exists.  These options imply \verb+-Fl+ and should
not be used with any lower basic falseness level; the \verb+-Fc+ option can,
however, usefully be used in conjuction with \verb+-Fe+ or \verb^-Fe+^.

In practice, \verb+-Fe+ should be used for non-palindromic plain methods 
where it ensures that sufficient true leads exist for an extent to be possible.
(The option is safe but unnecessary with palindromic plain methods as these
necessarily have enough true leads to form an extent.)
Similarly, the \verb^-Fe+^ option should be used with treble-dodging methods
where it ensures that sufficient true leads exist for a bobs-only exist.
Neither option actually check that the leads can be joined into an extent. 

These options are implemented by considering the graph\index{graph theory} 
whose vertices are the possible lead heads and edges exist between each pair
of mutually false leads.  Mathematically, this is the Cayley graph%
\index{Cayley graph} $\Gamma(S_{n-1}, F(M))$ where $S_{n-1}$%
\index{group!symmetric, $S_n$} is the set of
possible fixed-treble lead heads on $n$ bells.  An example of the 
largest possible set of mutually true leads corresponds to a \textitidx{maximal
independent set} of this graph.  In the general case, finding the size of this 
is NP-complete;\index{NP-complete}\footnote{Actually, while the case of
the maximal independent set of a general graph is known to be NP-complete,
it is not known whether the special case of the maximal independent set of 
a vertex transitive graph (such as a Cayley graph) is NP-complete.}
however it is simple to test whether the maximal independent set is exactly 
half the size of the graph by testing whether the graph is 
\textitidx{bipartite} which can be tested easily.  This is what the \verb+-Fe+
option does; the \verb^-Fe+^ option does the same for the in-course lead heads,
$A_{n-1}$\index{group!alternating, $A_n$}.

For palindromic treble-dodging minor methods that have a double change at 
the half lead (as all five-lead ones will), the \verb^-Fe+^ option is a
overkill.  A better strategy is to require that in each half lead, there 
is one even parity\index{parity} row and one odd parity row for each position
of the treble.  The \verb+-P+\oi{P} option enforces this.
Thus a method can start \verb+-34-+ or \verb+34-34+ which have 
parity structures $+{}+{}-{}-$ or $+{}-{}-{}+$ respectively; 
but a \verb+-3456-+ start has only even parity rows when the treble leads, 
and only odd parity rows when it is in seconds place: $+{}-{}+{}-$.

If you wish to check whether an extent actually does exist, you can 
get \methsearch\ to invoke the \texttt{touchsearch} program on each 
method.  See \sref{touchsearch} for more details.

\index{extents, viability of|)}

\section{Avoiding specific rows}\label{avoidrow}

The \verb+-Fr+\oidx{Fr}{\texttt{Fr}} option allows you to specify
specific rows that must be avoided in the the method.%
\index{rows, avoiding specific|see{\texttt{-Fr}}}
The syntax is \verb+-Fr=+\textit{row} --- for example, \verb+-Fr=12435678+
will force the method not to contain the row \verb+12435678+.
Depending on the basic falseness level specified (\verb+-Fc+, 
\verb+-Fl+ or \verb+-Fh+\oi{Fc}\oi{Fl}\oi{Fh} --- see \sref{basic_false}), 
\methsearch\ will avoid the specified rows in the plain course, 
first lead or first half lead of the method.

The initial rounds is counted as part of the first lead;
the lead head at the end is not --- therefore, \verb+-Fl+ 
\verb+-Fr=135264+ will not prevent Plain Bob Minor from being found.

Multiple \verb+-Fr+ options work in the obvious way.  As \verb+-Fr+ works by 
pruning the search tree rather than filtering the results, adding a few 
\verb+-Fr+ options for unwanted rows that might occur early on in the 
method can vastly speed up a search.

Note that there is no interaction between this and the \verb+-Mlead+ and 
related options.  If you wish to prevent a specific row from occur in a 
different lead, you have two basic strategies.   Either you can use a 
\verb+-Fs+ option to specify the starting row of the method;
or you can work out the corresponding row to be avoided in the opening lead.  

The syntax of the \verb+-Fs+\oidx{Fs}{\texttt{Fs}} option for specifying
the starting row is similar to that of \verb+-Fr+ except that the
argument to \verb+-Fs+ must be a literal row (row expressions are not 
permitted), and multiple \verb+-Fr+ options are not allowed.

Suppose you want to ring the method being searched for starting at lead-head 
$s$, and you want that lead not to contain the row $r$, then this is 
equivalent to saying that the first lead of the plain course should not 
contain the row $s^{-1} r$.
Fortunately \methsearch\ provides a syntax to avoid you having to
compute these manually: \verb+-Fr="+$s$\verb+\+$\,r$\verb+"+ (Linux users
will want single quotes).  This is equivalent to \verb+-Fs=+$s$ \verb+-Fr=+$r$.
The details of this syntax are explained in \sref{rowexpr}.

\subsection{Row expressions}\label{rowexpr}\index{row expression}

The argument to \verb+-Fr+ is actually a \textit{row expression}%
\index{expression|seealso{row expression}} --- an expression that evaluates 
to a row or set of rows.  Row expressions also occur in the \verb+-FP+ 
option (\sref{pends}), the \verb+-Mlead+ family of options (\sref{muswhere}) 
and in the \verb+rowcalc+ utility (\sref{rowcalc}).
These row expressions are considerably simpler than the more general 
expressions in \sref{expr}, and are handled by entirely separate code in
\methsearch.  A future version of \methsearch\ is likely to merge both
into a single expression language, but until that happens, the row expression
syntax will remain very limited --- unless noted, all operators act only on 
rows, and there is no facility for integer arithmetic.

Valid expressions are formed from the following operators, 
arranged in precedence\index{operators!precedence} levels from 
highest to lowest.  The operators' associativity\index{operators!associativity}
is also given.  Parentheses may also be used to make the order of invocation
explicit.\index{parentheses}

\begin{tabular}{l@{\quad}l@{\quad}l}
\verb+^+&                   right& exponentiation\\
\verb+*+ \verb+/+ \verb+\+& left&  multiplication-type operators\\
\verb+-+&                   left&  set difference --- see \sref{rowexprmulti}\\
\end{tabular}

The multiplication operator, \verb+*+,\index{multiplication} 
permutes\index{permute} one row by another.  The convention used by 
\methsearch\ is that $x$\verb+*+$y$ means $x$ transposed by $y$, rather
than the $x$ transposition acting on $y$, thus \verb+1432*2143+ is 
\verb+4123+ and not \verb+2341+.\footnote{Neither convention is used 
universally by ringing theoreticians and both have advantages.}

Because row multiplication is non-commutative, 
division\index{division} isn't uniquely defined.  
You need to specify whether the dividend is right- or
left-multiplied by the inverse of the divisor.  As a result, it is the
general convention to define
\[ a/b = a b^{-1} \]
as right multiplication by the inverse of the divisor.  By analogy, I 
define the notation the notation
\[  a\backslash b = a^{-1} b \]
to mean left multiplication by inverse of the divisor.  The \verb+/+ and
\verb+\+ operators perform these two types of division, respectively.
The latter occurs in ringing-related problems considerably more frequently 
than the former.

The \verb+*+ and \verb+\+ operators may need quoting\index{shell!quotation}
to prevent the shell from trying to interpret them.  As usual, on Windows, 
double quotes should be used; and on Unix-like systems, single quotes will 
be necessary.

The \verb+^+ operator is \textit{overloaded}\index{operators!overloaded} 
with two meanings.  If the right-hand argument is a row, 
then it denotes \textitidx{conjugation}, and if the right-hand
argument is an integer then it denotes \textitidx{exponentiation};
in both cases, the left-hand argument must be a row.  In the mathematical
literature these operations are often both denoted by right superscripts
and this convention is followed here,%
\footnote{One motivation for using the exponentiation syntax for conjugation
is that both satisfy the identity $(a^r)^s =  a^{(rs)}$.  However the analogy 
shouldn't be taken too far.  If $a$ and $r$ are rows and $n$ is a integer,
then $(a^r)^n = (a^n)^r$, and one might use $a^{nr}$ as syntactic shorthand 
for this.  Nevertheless, the product, $nr$, of a integer and a row is 
otherwise meaningless and \methsearch\ does not support this usage.}
but using $x$\verb+^+$y$ in place of $x^y$.
The conjugation and exponentiation operations are defined by
\begin{align*}
a^r = r^{-1} a r, && 
a^n = \underbrace{a\,a\,a\,\dotsm a}_{\text{$n$ times}}, &&
a^{-n} = (a^{-1})^n,
\end{align*}
where $a$ and $r$ are rows, $n$ is a positive integer, and
$a^{-1}$ denotes the inverse\index{inverse} of $a$.

Row expressions may also contain row literals and integer literals.
Row literals\index{literals} may have the treble omitted --- for example,
\verb+2345+ is equivalent to \verb+12345+.  This does not generalise
further so \verb+345+ is not a valid row literal.  Rows may also have 
an arbitrary number of tenors omitted, thus the two preceding examples
are equivalent to \verb+123+ or even just \verb+1+.  
Integer literals may optionally be prefixed by a \verb-+- or \verb+-+ sign.

An ambiguity exists between rows and integers, although in practice
\methsearch\ almost always gets it right.  The heuristics by which rows
and integers are disambiguated are as follows:
\begin{itemize}\renewcommand{\labelitemi}{---}
\item in contexts where integers would be invalid (i.e.\ anywhere other
than the right-hand argument to a \verb+^+ operator), it is parsed as a row;
\item if a leading \verb-+- or \verb+-+ is given, it will parse as an integer;
\item if the token is only a single character long, it is parsed as an 
integer;\footnote{This case is only relevant in one case --- the digit 
\verb+2+.  Without this rule this would parse as rounds with an implicit 
treble and we would find that $1 = 2$.  (Technically, it also changes \verb+1+
from rounds to the number 1, but in practice this is undetectable.)}
\item if the token is a valid row (possibly omitting the treble), it is 
parsed as a row;
\item otherwise it is parsed as an integer.
\end{itemize}

\subsection{Multiple rows}\label{rowexprmulti}

Most of the time, there isn't just one row that you wish to avoid.
The obvious use is when you have a particular composition in mind and have
already selected some of the methods.  In this case, you'll probably have
hundreds of rows that you wish to avoid.
If so, listing them all on the command-line is likely to be unsatisfactory.
(Not only is it going to be tedious to write them, and error-prone unless
you can automate the process, but there are technical reasons for not
wanting the command line to grow too long.)
\methsearch\ has four short-hand notations for sets of rows: 
\textit{set literals}, \textit{group generator lists},
\textit{file inputs} and \textit{command invocations}.

A \textitidx{set literal} is specified by enclosing a list of rows 
(or row expressions) in braces and separating the rows with columns --- thus, 
\verb+-Fr="{+$a$\verb+,+$b$\verb+,+$c$\verb+}"+ is a set literal equivalent 
to three  separate options, one each for each row in the set $\{a, b, c\}$.

A \textitidx{group generator list} is specified by enclosing
a list of rows (or row expressions) in angle brackets; these are used to
generate a group\index{group!generators}.  For example, \verb+-Fr="<34562>"+
generates the cyclic group\index{group!cyclic, $C_n$} on the five working 
bells, and \verb+-Fr="<342,23465>"+ generates a group of order six isomorphic
to $C_3 \times C_2$.  A future version of \methsearch\ is likely to provide
an alternative syntax for specifying groups as it is not always readily 
apparent what group is generated by a particular set of generators.  
In particular, generating large groups like $S_3 \times S_5$ (which
occur in \verb+-FP+ options for differentials\index{differentials}) is
error-prone with the current syntax.

A \textitidx{file input} reads rows from a file --- the 
syntax for this is \verb+-Fr="<(+\textit{filename}\verb+)"+.  
The file format is quite flexible: a sequence of rows separated by some
combination of spaces, tabs, new lines, commas, semicolons or colons;
however a standard \textit{row stream}\index{row stream} (\sref{rowstream}) 
with one row per line is recommended.  As a special case, if \textit{filename}
is \verb+-+, instead of reading from a file, \methsearch\ reads from standard
input.  If you do this, you must ensure that nothing else is also reading
from standard input or you will end up with bizarre undefined behaviour.
In particular, you can only have one \verb+<(-)+ anywhere, and you cannot
use this in conjunction with \verb+-I+\oi{I} (\sref{misc_opt}).

A \textit{command invocation} has the syntax 
\verb+-Fr="$(+\textit{command line}\verb+)"+ and is discussed in detail 
in \sref{cmdinv}.

Note that the sequence \verb+<(+ is always treated as the start of a file
input even if this results in a parse error.  
If you wish to put parentheses inside a group generator list, 
you should put a space between the \verb+<+ and \verb+(+.%
\footnote{This will be familiar to many programmers as the 
\textitidx{maximal munch} principle which requires that a sequence of 
characters is parsed as the longest possible token.  For example, in many 
C-like languages, \verb+1--1+ is an error as it treats \verb+--+ as a single
unary operator, but \verb+1 - -1+ is parsed successfully (and evaluates to 2).}

Sets (using any of these syntaxes) can appear in expressions.  If an
expression contains just one set, the expression is evaluated multiple
times, once for each member of the set.\footnote{This behaviour might be
counter-intuitive for \verb+{+$a$\verb+,+$b$\verb+,+$\cdots$\verb+}^+$n$,
and one might anticipate this to be equal to 
\verb+<+$a$\verb+,+$b$\verb+,+$\cdots$\verb+>+ for sufficiently large $n$.
It is not, and $A$\verb+^2+${} \not= AA$.}
For example, \verb+-Fr="{ 124365, 132465, 213465 }^654321"+
is equivalent to \verb+-Fr=214356 -Fr=213546 -Fr=214565+.

An \verb+-Fr+ option may contain two (or more) sets of rows.  Using
the definitions, $AB$ and $\bar{A}$, introduced in \sref{falsetheo}, 
these are defined as follows for sets $A$ and $B$:

\begin{tabular}{c@{$\quad\iff\quad$}c}
\verb+-Fr="+$A$\verb+*+$B$\verb+"+  &  \verb+-Fr="+$A B$\verb+"+ \\
\verb+-Fr="+$A$\verb+/+$B$\verb+"+  &  \verb+-Fr="+$A \bar{B}$\verb+"+ \\
\verb+-Fr="+$A$\verb+\+$B$\verb+"+  &  \verb+-Fr="+$\bar{A} B$\verb+"+ \\
\verb+-Fr="+$A$\verb+^+$B$\verb+"+  &  \verb+-Fr="+$A^B$\verb+"+ \\
\end{tabular}

In the last case (conjugation), the set $A^B$ is defined
\[ A^B = \{ b^{-1} a b : a \in A, b \in B \}. \]
Thus $A^B \not= \bar{B}AB$ because the same element of $B$ is used on both
sides of $A$.  This was the major motivation for adding support to 
\methsearch\ for conjugation with the \verb+^+ operator.

Note that $A$\verb+\+$B$ does \textit{not} denote the difference
between two sets which is often denoted by a backslash in the mathematical
literature.   The \verb+-+ operator\index{set!difference} serves that role,
and the difference (or complement), $A-B$, is defined in the usual way as
all elements of $A$ that are not elements of $B$:
\[ A-B = \{ a : a \in A, a \not\in B \}. \]
The \verb+-+ operator can also be used when one or both of its arguments 
are rows rather than sets.  In such a case, the row is silently replaced 
with a set containing just that one row.  
This is consistent with \methsearch's general policy that a row
and a set containing one row should always be treated the same.

These relations are equally valid irrespective of whether the set is a 
set literal, read from a file, or generated by a command invocation.
Using these set products, it's easy to generate very large row sets.  
\methsearch\ needs to be able to hold all the distinct \verb+-Fr+ rows 
in memory at one time.

The \ttcmdidx{rowcalc} utility (\sref{rowcalc}) provides a handy way 
of checking that any complicated expressions passed to \verb+-Fr+
are producing the correct output.

\subsection{An example — course splices}
\index{example!minor, course splices}

As an example, suppose we want to find \textitidx{course splices} of
Beverley Surprise Minor.  These are methods that can be dropped into an
extent of Beverley in place of a whole course of Beverley and still 
leave a true extent.

Suppose we have a file called \verb+Beverley.txt+ containing all the rows in 
a plain course of Beverley.  We can then use \methsearch\ to calculate
all of the rows in the five other courses (i.e.\ those besides the plain 
course), and ask \methsearch\ to find methods not containing any of these
rows.  This can be done as follows.

\begin{Verbatim}[commandchars=@~~]
methsearch -b6 -PG1 -sr -p2 -R"$q  $n" -Lsurp6.txt -Fc @hfill\
  -Fr="{134256,142356,152436,154326,153246} * <(Beverley.txt)"
\end{Verbatim}

The set literal contains the five non-plain course heads.  When
multiplied by the plain course, this generates a set of 600 rows
to avoid, and the \verb+-Fc+ option checks the whole plain course
against these 600 rows.  It generates the following output.

\begin{Verbatim}[obeytabs=true]
&-3-4-2-3.4-2.5,2	Surfleet
&-3-4-2-3.4-2.5,1	Hexham
&-3-4-2-3.4-34.1,2	Durham
&-3-4-2-3.4-34.5,2	Beverley
&-3-4-2-3.4-34.5,1	Berwick
\end{Verbatim}

Beverley and Surfleet are \textitidx{lead splices} which are a special case
of a course splice;  
Berwick and Hexham are their sixth-place variants and so contain all the same
rows in the course (although are not typically referred to as a course splice);
Durham is a genuine course splice of Beverley.

\section{Part end groups}\label{pends}\index{part end group}

\methsearch\ can also require a method to be true when rung in composition
based on a multi-part structure.  Only multi-part structures where the
parts form a mathemtical group\index{group} are supported, 
but in practice virtually all sets of part ends are groups.  
A simple example might be a three-part
structure with part ends $\{\hbox{\verb+123456+}$, $\hbox{\verb+134256+}$,
$\hbox{\verb+142356+}\}$.

The parts are specified with the \verb+-FP+\oidx{FP}{\texttt{FP}} option. 
Its argument is a row expression, and the rows passed to \verb+-FP+ are used 
to generate\index{generators, of group} the part end group.  
Thus for the three-part group above only one option is needed 
\verb+-FP=134256+.

This option can be combined with \verb+-Fs+,\oi{Fs} \verb+-Fr+\oi{Fr} 
(see \sref{avoidrow}) and \verb+-Fc+,\oi{Fc} 
\verb+-Fl+\oi{Fl} or \verb+-Fh+\oi{Fh} (see \sref{basic_false}) 
with powerful effect.  These interact as follows.
The \verb+-Fc+, \verb+-Fl+ or \verb+-Fh+ option is used to 
determine whether to prove the whole course,
the first lead (the default), or the first half lead.  
Let $M$ denote the rows in this section of method.
The \verb+-Fs+ option specifies the starting row, denoted here by $s$;
the \verb+-FP+ options generate the part end group, $P$; and 
the \verb+-Fr+ options enumerate any rows to be avoided, $A$.
\methsearch\ requires that the following rows are mutually true:%
\footnote{This is mathematically sloppy as a set, by definition,
contains no duplications.  What we means is that there is no intersection
between each $psM$ for all $p \in P$, and also between each of these and $A$.
A cleaner way of expressing this is with the cardinality%
\index{cardinality, set} of the sets: 
\[ \left| P s M \cup A \right| 
 = \left|P\right| \times \left|M\right| + \left|A\right|. \]
} % end footnote
\[ PsM \cup A. \]
% Stonebow

The \verb+-FP+ option also provides a way of searching for methods that
can produce an extent.\index{extents, viability of}
For example, if you can find a group, $G$, and a treble path of length $m$ 
(whether in the half-lead, lead or course) such that $\left|G\right|m = n!$
then it might be possible to use the group as the part-end group for an extent.
In other words, it might be possible to find a method, $M$, that satisfies
$GM=S_n$.  

For example, most treble dodging minor methods produce extents
that have the {60} in-course rows of the form \verb+1.....+ as lead-ends
and lead-heads.  The \verb+-P+ option guarantees that the methods will be
true with a composition of this style.  The same effect can be 
achieved with \verb+-FP=124365 -FP=132546 -FP=132465+ \verb+-Fh+.  
(These three rows, which are the changes \verb+12+, \verb+16+ and \verb+14+ 
generate the group $A_5$ of in-course fixed-treble rows.)  
In practice, the \verb+-P+ option is significantly faster and there is no 
reason to prefer \verb+-FP+ in this case.\oi{P}

It makes no difference whether you use multiple \verb+-FP+ options, or to put 
all the generators together with a set literal.  For example, the generators
for $A_5$ given above could have been written 
\verb+-FP="{124365,132546,132465}"+.  The quotes may be necessary to prevent
the shell from interpreting the braces itself.\index{shell!quotation}

However, the \verb+-FP+ options can be used in other situations too.  For
example, there is a second group of {60} in-course rows on six bells ---
this is \textit{Hudson's group}\index{Hudson's group|see{group, Hudson's}}%
\index{group!Hudson's} which is generated by the changes
\verb+12+, \verb+16+ and \verb+34+.

\begin{Verbatim}[commandchars=@~~]
methsearch -b6 -G1 -p3 -l2 -srf -R"$q  $N" -Ldel6.txt@hfill\
  -FP=124365 -FP=132546 -FP=213465 -Fh
\end{Verbatim}

Interestingly, this just generates one method:%
\footnote{This method was originally named Hudson Delight Minor.%
\index{Hudson Delight Minor|see{Norwich Delight Minor}}
However, in their wisdom, the Central Council Methods Committee have
decided that this method should be renamed Norwich Delight Minor because
of some perceived similarity to a major method of that name.}
\begin{Verbatim}[obeytabs=true]
&3-3.4-2-1.4-4.5,2	Norwich Delight Minor
\end{Verbatim}
\index{Norwich Delight Minor}

\subsection{Differentials}\index{differentials|(}

The usual way to get a palindromic differential method capable of producing 
an extent is to arrange for all of the bells in one cycle to ring in every 
position relative to each other during half a lead.  For example, in a 
2,3 doubles differential, the two trebles would ring in each of the 
$\textrm{5}\times\textrm{4}\div\textrm{2} = \textrm{10}$ relative positions 
during half a lead.  The two trebles swap over at the half lead or lead head
to produce a lead head of \verb+21453+ or \verb+21534+.  During the course
(which is a whole extent), the twelve lead ends and lead heads are:
\begin{center}
\begin{tabular}{cccccc}
12345&21453&12534&21345&12453&21534\\
21435&12543&21354&12435&21543&12354\\
\end{tabular}
\end{center}
These twelve rows have every combination of the the two trebles in both 
possible orders and the three tenors in all six orders.  It is therefore
the group $S_2 \times S_3$.  
To get \methsearch\ to search for 2,3 doubles differentials, you ask
it to look for palindromic methods with 20 rows per lead, 6 leads per course, 
and a half-lead that is true to this group:%
\footnote{Such methods were first catalogued by Jonathan 
Deane\index{Deane, Jonathan} and published in the \textit{Ringing World}
[1994, p.~407].  He recognised that it was impossible to avoid four blows
in one place, but wanted limit this the occurence of multiple consecutive 
blows in one place to just four blows at lead across the lead end.  
\methsearch\ can duplicate his analysis by adding \verb+-p2+ 
\verb+--long-le-place+ \verb+-Q"$19h ne '145' || $20h ne '125'"+ 
to the command line.  The latter \verb+-Q+
option is to avoid methods with four blows behind over the lead end.}%
\index{example!doubles, differential}
\begin{Verbatim}
methsearch -b5 -AU0 -n20 -sFh -FP="{12435,21453}" -Q"$o==6"
\end{Verbatim}
In this case, \verb+12435+ and \verb+21453+ generate the group 
$S_2 \times S_3$.   The need to specify the part-end group in terms
of its generators\index{group!generators} can be error-prone 
and a future version of \methsearch\ is likely to introduce an alternative 
syntax for specifying groups.  
In the meantime, the \ttcmdidx{rowcalc} utility (\sref{rowcalc})
help check that the group produced is the intended one.  For example,
\begin{Verbatim}
rowcalc -b5 -c "<12435,21453>"
\end{Verbatim}
prints \verb+12+, the size of the group.  (Removing the \verb+-c+ causes it
to print the whole group.)

This extends to differentials on higher numbers.  Double Helix 
Differential Major\index{Double Helix Differential Major} is a 3,5 major
differential, so it uses the part-end group $S_3 \times S_5$ which 
is generated by $\{ \texttt{23156487}, \texttt{13287654} \}$.  This 
is an example where the full search space may be too large to enumerate 
exhaustively (depending on exactly what additional criteria are used),
however it is a good example of when the random sampling techniques in
\sref{random} can be used.\loi{random}\loi{timeout}\loi{loop}%
\index{example!major, differential}
\begin{Verbatim}[commandchars=@~~]
methsearch -b8 -AU0 -n112 -sFh -FP="{23156487,13287654}"    @hfill\
  -Q"$o==15" -dkj -p2 --random --loop --timeout=1
\end{Verbatim}

The correct choie of timeout is critical to getting such a search to
run efficiently.  This is because there are large regions of the search space 
with no methods, and if a loop enters into such a region, it's highly 
unlikely to ever leave it.  Equally, if a loop enters a region with lots 
of methods, it needs long enough that it will find one of them.  
Too high a timeout and we waste time in barren parts of the search space: 
too low and we fail to notice methods when they are there.

In the general case, \methsearch\ runs very slowly when the part-end
group, $P$, is large.  
Each node of search tree requires an operation that's linear in
the size of group.  However, for the
most common case where the group is just a direct product\index{direct 
product} of symmetric groups\index{group!symmetric, $S_n$}, 
(i.e.\ if $P = \prod S_{n_i}$), a constant-time algorithm is used.
What this means in practice is that it is fast to search for differentials
using the techniques discussed here, but that other, seemingly-similar 
searches can end up very much slower.

\index{differentials|)}
\index{falseness|)}

\chapter{Music}\label{music}\index{music|(}

\methsearch\ is able to analyse the music present in a method.  
This is controlled by the \verb+-M+\oid{M}{music} option and is used to
generate a musical score which is displayed by the \verb+$M+\fspec{M}
variable.  The first section (\sref{muspattern}) documents how to 
analyse music in the plain course; more general analysis is discussed
in \sref{muswhere}.

\section{Music patterns}\label{muspattern}

At its simplest, \methsearch\ simply counts the number of rows in
the plain course matching a particular \textitidx{music pattern}.  
Each pattern is specified with an option of the form 
\verb+-M+\textit{pattern}.  A pattern may contain literal bell symbols 
(using the bell symbols listed in \sref{pn}) or \textitidx{wildcards} that 
can match various bells.  Three forms of wildcard are supported:
a \verb+?+ represents any single bell, a \verb+*+ represents any number
(including zero) arbitrary bells, and \verb+[+\textit{list}\verb+]+ matches
exactly one bell from the list of symbols.

For example, on eight bells, \verb+-M????5678+ matches a roll-up\index{roll-up} 
at the back; this can be written more succinctly as \verb+-M*5678+.  A 
combination roll-up (CRU)\index{CRU|see{combination roll-up}}%
\index{combination roll-up} on eight
has two of \verb+4+, \verb+5+ and \verb+6+ rolling-up followed by \verb+78+.
It can be written \verb+-M*[456][456]78+.

If a pattern is shorter than the row and contains no \verb+*+ to match the
surplus bells, then it matches at the start of the row.  For example,
\verb+-M5678+ matches the row \verb+56781234+ but not \verb+12345678+.  To
match a sequence of bells at any place in the row, place a \verb+*+ on
both sides of the pattern: \verb+-M*5678*+.

\subsection{Named music patterns}\label{musnamed}

There is also a shorthand notation for matching particular commonly requested
patterns such as queens, CRUs, or four-bell runs.  This is written
\verb+-M<+\textit{name}\verb+>+.  Note that it will generally be necessary to 
enclose the \verb+<+ and \verb+>+ signs (and, for convenience, everything 
between them in quotes): \verb+-M"<+\textit{name}\verb+>"+.%
\index{shell!quotation}
At present, the following named patterns are supported.

\begin{tabularx}{\textwidth}{lX}
\texttt{rounds}\index{rounds}
  &The bells in order; e.g.\ \texttt{1234567} on seven bells.\\
\texttt{queens}\index{queens}
  &The odd bells in order, followed by the even bells in order;
  e.g.\ \texttt{1357246} on seven bells.%
  \footnote{The examples are given on an odd number of bells because these
  named rows are ambiguous at odd stages.  The particular choice of queens
  as \texttt{1357246} instead of \texttt{2461357} is because odd bell methods
  are more usually rung with a tenor covering.}\\
\texttt{kings}\index{kings}
  &The odd bells in reverse, followed by the even bells in order;
  e.g.\ \texttt{7531246} on seven bells.\\
\texttt{tittums}\index{tittums}
  &The front bells in sequence, alternating with the back bells
  in order; e.g.\ \texttt{1526374} on seven bells.\\
\texttt{reverse-rounds}\index{reverse rounds}&
  The bells in reverse order; e.g.\ \texttt{7654321} on 
  seven bells.\\
\texttt{CRUs}\index{combination roll-up}&
  Any four bells, followed by two of $\{ \hbox{\tt4}, \hbox{\tt5}, 
  \hbox{\tt6}\}$, followed by the back bells in order.\\
\texttt{front-}$n$\texttt{-runs}\index{runs}&
  A run of $n$ bells in sequence 
  (forwards or backwards) off the front of the row.\\
\texttt{back-}$n$\texttt{-runs}&A run of $n$ bells in sequence
  (forwards or backwards) at the back of the row.\\
$n$\texttt{-runs}&A run of $n$ bells in sequence (forwards or backwards) at
  an arbitrary position in the row.\\
\end{tabularx}

With the $n$\verb+-runs+ named pattern, it is possible for a row to match
the pattern more than once.  For example, the row \verb+32145678+ matches
\verb+-M<4-runs>+ twice because it has a five-bell run and both of its 
constituent four-bell runs (\verb+4567+ and \verb+5678+) match.  This means
a longer run scores more than a shorter run.

\subsection{Customising scores}\label{musscore}

By default, each match adds 1 to the music score.  Multiple patterns
may be specified, in which case each row is matched against each pattern
allowing rows to match multiple times.  It is possible to change the score
awarded to each match of a pattern.  The syntax for this is 
\verb+-M+\textit{score}\verb+:+\textit{pattern}, where \textit{pattern}
may be a simple pattern or a named pattern enclosed in \verb+<+\ldots\verb+>+.

This allows you to specify the relative merits of particular rows.  For 
example, \verb+-M<4-runs>+\hfill\verb+-M5:*7568+\hfill\verb+-M5:*5678+\hfill%
\verb+-M5:8765+\hfill\verb+-M5:5678+\hfill\verb+-M10:<queens>+\\
will award ten points to queens,
five points to each of four particularly musical types of runs, and one point
for other four-bell runs.  (As a result, rounds would score 10 --- one each
for the five four-bell runs in \verb+12345678+, and a further five for 
the \verb+5678+ at the back.)

It is also possible to score rows at handstroke and backstroke differently.
This can be useful for excluding 87s at backstroke (on eight bells, or 
equivalent on other stages).  The syntax is 
\verb+-M+$h$\verb+,+$b$\verb+:+\textit{pattern} where $h$ is the score
to award if the pattern matches at handstroke\index{handstroke} and 
$b$ the score to award for a match at backstroke\index{backstroke}.%
\footnote{\methsearch\ assumes that the method is rung with the first 
non-rounds change being a handstroke.}
By setting $h=0$ and $b$ to a large negative number, it makes it easy to locate
and reject methods with a particular row at backstroke.  E.g.
\verb+-M0,-1000:*87+.

\section{Music outside the plain course}\label{muswhere}

Sometimes we are interested in music other than in the plain course
and \methsearch\ can accommodate this via the 
\verb+-M+\textit{type}\verb+=+\textit{row} options.  The possible
values of \textit{type} are as follows:

\begin{tabularx}{\textwidth}{lX}
\texttt{course}\oidM{course}
  &Analyse a whole course starting from course head \textit{row}.\\
\texttt{lead}\oidM{lead}
  &Analyse one lead starting from lead head \textit{row}, 
  excluding the closing lead head row.\\
\texttt{halflead}\oidM{halflead}
  &The first half of the lead starting from lead head \textit{row}\\
\texttt{2halflead}\oidM{2halflead}
  &The second half of the lead, starting with \textit{row} as the 
  half lead head row.\\
\texttt{2rhalflead}\oidM{2rhalflead}
  &The second half of the lead, ending with \textit{row} as the
  lead end row.\\
\texttt{rhalflead}\oidM{rhalflead}
  &The first half of the lead, ending with \textit{row} as the 
  the half lead end row.%
  \footnote{This option is largely included for completeness.  It is not 
  anticipated that \texttt{rhalflead} will often be used.}\\
\end{tabularx}

Multiple \verb+-M+\textit{type}\verb+=+\textit{row} options may be provided.
The \verb+=+\textit{row} component of these options may be omitted,
in which case it defaults to the \textit{row} option to the previous
such option if there is one, or to rounds if no such previous option exists.  
For example,
\texttt{-Mlead -Mlead=13426578 -Mlead} is equivalent to
\texttt{-Mlead=12345678 -Mlead=13426578 -Mlead=13426578}.

(Note that there is a potential, though highly improbable, ambiguity between
these options and the \verb+-M+\textit{pattern} options described
in \sref{muspattern}.  This only applies on 22 or more bells and is because
\verb+-Mlead+, for example, is also a pattern matching a row with the bells
22, 11, 13, 16{} in order at the front.  In the case of such an ambiguity, 
the meaning described in this section is chosen.  If this is of concern 
--- for example, if you are automatically generating patterns for some purpose
other than conventional musical analysis --- a pattern that does not already
contain a \verb+*+ can always have one appended without changing its meaning, 
and this will force the option to be interpreted as a pattern.  Similarly,
a pattern can always have the implicit \verb+1:+ scoring prefix stated 
explicitly which will also force interpretation as a pattern.)

These \verb+-M+\textit{type} options do not, of themselves, specify what
music to look for --- they need using in conjunction with the 
music pattern options of \sref{muspattern}.  In the simplest usage, a single
\verb+-M+\textit{type} option will be followed by a \verb+-M+\textit{pattern}
option which will apply to the block specified.  For example,
\texttt{-Mlead=15634827 -M"<queens>"} will look for queens in the row 
starting with \verb+15634827+.

\subsection{Batches of music patterns}\index{music!batches|(}

In more complicated scenarios, it can be desirable to search for 
multiple types of music in different places.  This is possible too.
The \verb+-M+ options from the command line are considered in sequence.%
\footnote{This is one of the few places where the order of command line 
options is important.}
In general, \methsearch\ divides the \verb+-M+ options into a sequence of 
batches, each of which consists of one or more \verb+-M+\textit{type} option 
followed by one or more \verb+-M+\textit{pattern} option.  
Optionally, the first batch may have no \verb+-M+\textit{type} options 
specified, in which case \verb+-Mcourse+ is assumed;
similarly, the last batch may have no \verb+-M+\textit{pattern} options
specified, in which case \verb+-M<CRU>+ is assumed.  (Thus, if no \verb+-M+
options are given at all, \methsearch\ looks for CRUs in the plain course.)

Within each batch of \verb+-M+ options, each of the music pattern options
(whether named patterns with \verb+-M<+\textit{name}\verb+>+ or 
regex-style patterns with \verb+-M+\textit{pattern}) 
applies to all of the \verb+-M+\textit{type} options within that 
batch of \verb+-M+ options.  Some examples may make this clearer.

\begin{Verbatim}
-M"<4-runs>" -Mlead -M2:"<4-runs>" -M10:"<reverse-rounds>"
\end{Verbatim}
This example contains two batches of \verb+-M+ options.  The first batch
looks will score one point for each four-bell run in the plain course.
The second batch re-analyses just the first lead, scoring another two points 
for each four-bell run in that lead, and another further ten points if back 
rounds occurs in the first lead.  
Because rows can match multiple patterns, an eight bell method such as
Plain Bob which has reverse rounds at the first lead will score 
$\textrm{1}\times\textrm{5} + \textrm{2}\times\textrm{5} 
  + \textrm{10}\times\textrm{1} = \textrm{25}$ points for this row alone.%
\footnote{Reverse rounds on $n$ bells matches the \texttt{<$m$-run>} pattern 
$n-m+1$ times, once for each of the separate $m$ bell runs.}

\begin{Verbatim}[commandchars=@~~]
-Mlead=15634827 -Mlead=16745238 -Mlead=17856342 @hfill\
-Mlead=18267453 -Mlead=12378564 -Mlead=13482675 @hfill\
-Mlead=14523786 -M"<4-runs>" -M10:"<queens>"
\end{Verbatim}
This contains a single batch of \verb+-M+ options.  For each of the seven 
leads listed, it scores four-bell runs with an extra bonus for queens.%
\footnote{This example will be of particular interest to anyone
who wishes to replace Stonebow\index{Stonebow, replacing} in David Hull's
cyclic 23-spliced composition.\index{Hull, David!23 spliced}}

In practice, however, such a circumlocution is unnecessary because the 
\textit{row} argument to the \verb+-M+\textit{type}\verb+=+\textit{row} 
options is actually a \textit{row expression} (\sref{rowexpr}).%
\index{row expression}  
This means that the group generator\index{group!generators} syntax can be used 
to write this example more concisely and informatively as:
\begin{Verbatim}
-Mlead="<8234567>*5634827" -M"<4-runs>" -M10:"<queens>"
\end{Verbatim}

More powerful music analysis of touches can be achieved by getting 
\methsearch\ to invoke \texttt{gsiril} to print the touch.  This is 
discussed further in \sref{gsirilrow}; there are often several ways to
achieve the desired analysis and the discussion in \sref{efficiency} may
be helpful in understanding the relative efficiencies of each way.

\section{Falseness and music}

The \verb+-M+ options are entirely independent from the \verb+-F+ options.
This has the unfortunate result that for many real-world searches, it will be
necessary to duplicate information between \verb+-Fr+ options and
\verb+-Mlead+ options so that both the falseness engine and the music
analysis engine know which leads to look at.  This is something that a future
version of \methsearch\ might address.



\index{music!batches|)}
\index{music|)}

\chapter{Interoperability}\label{interop}

As mentioned in \sref{unixphil}, the Unix paradigm\index{Unix philosophy} 
is not one of huge monolithic programs.  Instead, it favours separate
small programs that do one thing and do it well.  This tends to result in
greater flexibility, allowing programs to be used in novel combinations to
achieve things that their original authors hadn't considered.  This is
the approach that \methsearch\ attempts to take.

\methsearch\ interacts with external programs in several main ways.  First, it
can filter input (in the form of a list of methods) from another source;
this behaviour is activated by the \verb+-I+\oi{I} option was 
discussed in \sref{misc_opt}.  Secondly, it can output a list of methods
which is parsed by another program; an example of this was also given in
the discussion of filtering \sref{misc_opt}.  The final ways are when 
\methsearch\ invokes an external program via a \textit{command invocation}
(\sref{cmdinv}).  

In order that programs can pass information between each other, 
it is necessary that they share common way of representing the information.  
\methsearch, and the other utilities in the Ringing Class Library%
\index{Ringing Class Library}, have two main interchange formats:
\textit{row streams}\index{row stream} (\sref{rowstream}) 
for passing lists of rows about, and 
\textit{method streams}\index{method stream} (\sref{methstream}) 
for passing lists of methods.%
\index{stream|see{row stream \textit{and} method stream}}

Also discussed in this chapter are some of more useful utility programs
that can generate or parse these streams, however this document does not
aim to fully document these utilities.

\section{Command invocations}\label{cmdinv}\index{command invocation|(}

External commands can be invoked by \methsearch\ in three situations.
First, a command invocation may be embedded directly in a format string
(\sref{fmtstr}); second, it may be inside an expression (\sref{expr}),
which itself will either be in a format string or as the argument to a 
\verb+-Q+\oi{Q} option; and third, it can be used to 
generate a list of rows within a row expression 
(\sref{rowexpr}--\ref{rowexprmulti}).  

In the first case, the command is invoked once per method found.
In the second case, the command is invoked once per method under consideration
(as the output of the command invocation is used to determine whether the
method is satisfactory).  But in the third case, the command is only
invoked once, at program start up.

In each case, the syntax is the same: 
\verb+$(+\textit{command line\ldots}\verb+ )+.  Here, \textit{command line}
is a command (or sequence of commands, such as a pipe line\index{pipe})  
that can be executed by the shell\index{shell}.%
\footnote{The particular shell that is used to handle command invocations is
governed by the \verb+$SHELL+\fspec{SHELL} \textitidx{environment variable}.
If this is set, then the command line is passed as a single argument to a
\verb+$SHELL -c+ process.  If it is not set, on Unix-type systems, the 
default is to use \ttcmdidx{sh}\verb+ -c+,
and on Windows, the default is to use \ttcmdidx{cmd.exe}\verb+ /c+.%
In most situations, the default behaviour
should be fine as the \verb+$SHELL+ environment variable is normally set 
correctly by the shell.}
Standard output\index{standard output} from the command is read by 
\methsearch.\footnote{Anything written to standard error\index{standard error} 
is ignored by \methsearch, but will be displayed to the user.  It is 
therefore recommended that if the program would routinely write to
standard error this is discarded, for example on a Linux system
by appending \verb+2> /dev/null+ to the command.  The exit status of the 
command being invoked is currently ignored.  This is likely to change.}
In a row expression this is expected to yield a row stream
as documented in \sref{rowexprmulti} and \sref{rowstream}.  
A straightforward example is 
when it just invokes a single command, for example:
\begin{Verbatim}
-Fr='$(printmethod -b8 -s 13578264 -F "&-3-4-5-6-2-3-4-7,2")'
\end{Verbatim}
This invokes the \ttcmdidx{printmethod} command to enumerate all of the rows
in a lead of Yorkshire starting from the lead head \verb+13578264+.  This 
would be useful when trying to find a method for a composition of spliced
that already contains this lead.

When a command is invoked from a (normal, non-row) expression (\sref{expr}) 
or format string, \methsearch\ will substitute any method variables 
(\sref{variables}) or nested expressions (enclosed in \verb+$[+\ldots\verb+]+)
that it finds on the command line.  (For obvious reasons this
does not apply when evaluating row expression --- there is no method
under consideration, so the method variables would have an undefined value.)
The command can generate arbitrary output which is like string literal 
(\sref{lits}).  However in most practical situations the string will 
subsequently be converted to an integer.
For example, the \ttcmdidx{touchsearch} program (\sref{touchsearch})
could be invoked by a \verb+-Q+ option to check that a touch 
of the method is actually possible:
\begin{Verbatim}[commandchars=@~~]
-Q'$(touchsearch -b6 -r -l720 -q --raw-count --limit=1 @hfill\
       -Cb=4 -Cs=$[$12h eq "12" ? "234" : "456"] $p)'
\end{Verbatim}
This command invocation contains both a variable 
(the place notation, \verb+$p+) and a more complicated expression used to
determine whether the method is a seconds place one, in which case a \verb+234+ 
single is standard, or a sixths place one, in which case \verb+456+ single
is required.  The \verb+--raw-count+ option ensures that the command
only prints the number of matching touches without surrounding text, and
the \verb+--limit=1+ option halts the search after the first touch is found
as we do not care about the number of matching touches.

In practice, this example can be handled more efficiently using the 
\verb+--filter+ option to \texttt{touchsearch} and using this to filter
the output of \methsearch.

Within format strings or non-row expressions, \methsearch\ also accepts
the sequence \verb+$)+ as a way of escaping a parenthesis so it can be 
included in the command to be executed.\index{parentheses!escaping}%
\index{escape character}  This is necessary because otherwise 
\methsearch\ would interpret the \verb+)+ as terminating the command 
invocation.  The same does not apply to command invocations in row
expressions, a deficiency that will likely be addressed in a future 
version of \methsearch.


\subsection{Efficiency}\label{efficiency}

Overuse of per-method command invocation can have a severe impact on 
efficiency.  A good example of this is with music analysis (\sref{music}).
It is possible to get all of the power of the various \verb+-M+ options (and
more beyond) by invoking various external programs to analyse the music.
For example, the following search for musical double surprise major methods%
\index{example!major, musical double surprise|(}
\begin{Verbatim}
methsearch -b8 -SG1 -srdk -p2 -R'$q $M' -M'<4-runs>'
\end{Verbatim}
can be rewritten to invoke the external \ttcmdidx{printmethod}
(\sref{printmethod}) and \ttcmdidx{musgrep} (\sref{musgrep}) programs 
instead of using \methsearch's in-built \verb+$M+ variable:%
\footnote{Both of these examples are presented here
with Unix-style quoting.\index{shell!quotation}  
On Windows, the outer single quotes will need replacing with double quotes.}
\begin{Verbatim}[commandchars=@~~]
methsearch -b8 -SG1 -srdk -p2 @hfill\
  -R'$q $(printmethod -b8 -cS $p | musgrep -b8 -s \"<4-runs>\")'
\end{Verbatim}
\index{example!major, musical double surprise|)}
However, doing so slows the search dramatically.\index{efficiency}
Timing the two commands on a 1.5\ GHz laptop running Linux shows the former
to take about 6.4\ s, and the latter, 71.2\ s --- a factor of 11 times slower.%
\footnote{This is due to the overhead of \textit{forking}%
\index{forking, overhead of} (that is, creating) a new process to evaluate 
the music.  
In this particular example, three processes are created for every one of 
the 4408 methods evaluated --- one shell\index{shell} process, 
one \texttt{printmethod} process, and one \texttt{musgrep} process.}

Examples with falseness can be even more extreme as \methsearch\ is able to
use knowledge of the falseness to prune large branches of the search tree
in one go.  A good instance of this is searching for 
CPS\index{clear proof scale} treble dodging major methods.
\index{example!major, treble dodging, CPS}
\begin{Verbatim}
methsearch -b8 -G1 -srfp2 -FCPS
\end{Verbatim}
On the same laptop, this command took 77.7\ s to complete.  If instead the
methods were filtered using an external program, as in the previous example,
it is likely that this search would take days, if not weeks, to complete.

These examples illustrate that following the Unix philosophy%
\index{Unix philosophy} of separate programs to its logical conclusion 
can sometimes have unfortunate consequences for efficiency.  \methsearch\ 
attempts to be pragmatic where a dogmatic separation of functionality would
result in extreme inefficiency.

Another efficiency bottleneck can come from writing data to disk.  \methsearch\ 
only directly writes data to disk when a \verb+-o+ option (\sref{output_opt})
is given, but indirect writing to disk can occur in two situations:
when the machine has insufficient memory and \textitidx{swaps} (that is,
starts using the disk as additional memory), and when standard output%
\index{standard output} is redirected\index{shell!redirection} by the shell
to a file.  \methsearch\ is generally quite economical with its memory usage
(an exception being when exceptionally lengthy \verb+-H+ statistics are
being gathered as discussed in \sref{stats}).

\subsection{Exit status}\label{exitstatus}

When a command exits it communicates an \textit{exit status}\index{exit status}
back to whatever launched it.  This is an integer, and, by convention, zero
is used to denote success and any non-zero number represents failure.  
Frequently programs will ignore the exit status of any other programs they
spawn, as those programs are able to print a much more descriptive error 
messages in the event of failure.  This is what \methsearch\ does.
However, occasionally it's useful to be able to find out whether or not a 
command exited successfully and the \verb+$?+ variable stores the exit 
status of the most recent command to be executed.

Before each format string is printed and any command invocations in it are 
executed, \verb+$?+ gets set to zero.  The format string is then evaluated
left-to-right with command invocations being invoked as they're encountered.  
This can be seen using the standard Unix programs 
\verb+true+\index{true@\texttt{true}} and 
\verb+false+\index{false@\texttt{false}} which simply exit 
successfully and unsuccessfully, respectively:
\verb+"$? $(false)$? $? $(true)$?"+ will evaluate to \verb+"0 1 1 0"+, with the 
value of \verb+$?+ changing through the format string.

To use the exit status instead of the output of a command in an expression
(such as the argument to a \verb+-Q+ option),
it's usually necessary to use the comma operator%
\index{operators!comma} to discard the output of the command.  For example
\verb+-Q"$(true), $?"+ will always succeed, despite the fact that \verb+true+
does not provide any output; similarly, \verb+-Q"$(false), $?"+ will always
fail.  However, trying to use \verb+-Q"$(true)"+ will give an error as 
the textual output of \verb+true+ is neither true nor false.

\index{command invocation|)}

\section{Row streams}\label{rowstream}\index{row stream|(}

Row streams are used to stream a list of rows between programs.  
The standard syntax is very straightforward: one row per line.%
\footnote{Some programs will accept more permissive formats: for example
by allowing (and ignoring) annotations after the row, 
by allowing multiple rows per line, or by accepting alternative separators.
You should consult the documentation for the specific program before using 
these.}
The only place where \methsearch\ uses row streams is when reading the output 
from command invocations in row expressions, for example in \verb+-Fr+ or 
\verb+-Mlead+ options.

The \ttcmdidx{extent} (\sref{extentutil}), 
\ttcmdidx{rowcalc} (\sref{rowcalc}),
\ttcmdidx{printmethod} (\sref{printmethod}) and 
\ttcmdidx{gsiril} (\sref{gsirilrow}) programs can
be used to generate row streams.  
The \ttcmdidx{musgrep} (\sref{musgrep}) program acts upon a row stream.%
\footnote{The experimental \ttcmdidx{ringmethod} program also acts on a row 
stream which it converts into an audio\index{audio} WAV file%
\index{wav file@\texttt{WAV} file} by synthesising bell sounds; however, there
is very little call for this program to be used in conjunction with 
\methsearch, so it is not documented here.  The \verb+--help+ text gives a
brief summary of how to use it.}
Both \texttt{rowcalc} and \texttt{musgrep} can be optionally be used as a 
row filter --- that is, a program that takes one row stream and uses it to 
generate another.

\subsection{Enumerating rows — \texttt{extent}}\label{extentutil}

The \ttcmdidx{extent} program is a very simple utility for listing all
of the rows on a given number of bells.  It accepts the following options:

\begin{tabularx}{\textwidth}{llX}
\texttt{-b}&\texttt{--bells=N}&The total number of bells\\
\texttt{-h}&\texttt{--hunts=N}&The number of fixed hunt bells at the front
  of each row\\
\texttt{-t}&\texttt{--tenors=N}&The number of fixed tenors at the back 
  of each row\\
\texttt{-i}&\texttt{--in-course}&Only list in-course (even parity) rows\\
\end{tabularx}

For example, \verb+extent -b8 -h1 -t2 -i+ lists the 
in-course\index{parity} tenors-together\index{tenors-together} lead heads
of the form \verb+1.....78+ of which there are 60.  
The rows are ordered lexicographically\index{lexicographic order}:%
\footnote{Strictly, this is not quite lexicographic 
order as \verb+E+ and \verb+T+ (for 11 and 12) are found before \verb+A+ (13).
So \verb+1234567890ETA+ comes before \verb+1234567890EAT+.}
\verb+12345678+, \verb+12346578+, \verb+12354678+, \ldots, \verb+16543278+.

It is rare to want to use this program in conjunction with \methsearch.
However, one use for this program is, in conjunction with \texttt{musgrep}, for
calculating the maximum possible musical score.  Assuming no rows 
have been given a negative score and that the stroke on which rows occur
is irrelevant then the full extent must have the maximum score possible
in a true touch.
For example, the following command will confirm that the maximum 4-run score
is 1200:\footnote{This has lead to a surprising degree of confusion.  These
runs are contained within 1002 rows, but because of the multiple counting of
longer runs, some rows score more than once.}
\begin{Verbatim}
extent -b8 | musgrep -b8 -s "<4-runs>" 
\end{Verbatim}


\subsection{Calculating rows — \texttt{rowcalc}}\label{rowcalc}

The \ttcmdidx{rowcalc} program is a stand-alone calculator for evaluating
row expressions (\sref{rowexpr}).  The only command line option is the 
optional \verb+-b+ option which specifies the number of bells. 
If this is supplied, the results are padded or truncated to that number of 
bells (and an error given if truncation produces an invalid row).
The command line must contain a row expression.  
For example, the following command generates a list
of the seven Plain Bob lead ends\index{Plain Bob lead ends} on eight bells:
\begin{Verbatim}
rowcalc "<13254768*12436587>"
\end{Verbatim}

If the row expression contains the special \verb+<(-)+ file input 
(\sref{rowexprmulti}), it will read a row stream from standard input
and apply the calculation to each row in turn.  A interesting example of 
this might be calculating which leads of Cambridge Surprise Major contain
reverse rounds.  If $l$ is a lead head and $M$ is the set of rows in the 
first lead, then $\hbox{\texttt{87654321}} \in lM$ implies that lead $l$
contains reverse rounds.  This can be expressed as follows:
\begin{Verbatim}[commandchars=@~~]
printmethod -b8 -F "&-3-4-25-36-4-5-6-7,2" @hfill\
  | rowcalc "87654321 / <(-)" | grep ^1 
\end{Verbatim}
The final \texttt{grep} command is there to restrict the list of lead ends
to those that have the treble leading.

\subsection{Printing methods — \texttt{printmethod}}\label{printmethod}

The \ttcmdidx{printmethod} program takes a method place notation and uses
it to generate a list of rows in a lead or course.
The following options are of particular relevance:%
\footnote{For a full list of options, run \verb+printmethod --help+.}

\begin{tabularx}{\textwidth}{llX}
\texttt{-b}&\texttt{--bells=N}&The total number of bells\\
\texttt{-c}&\texttt{--course}&Print a whole course instead of just a lead\\
\texttt{-F}&\texttt{--omit-final}&Omit the final row 
  to avoid duplication of the lead head or course head\\
\texttt{-S}&\texttt{--omit-start}&Omit the starting row 
  to avoid duplication of the lead head or course head\\
\texttt{-s}&\texttt{--start=ROW}&Start from \texttt{ROW} instead of rounds\\
\end{tabularx}

The \verb+-F+ option is of especial importance.  By default 
\texttt{printmethod} will print the rows in a lead from lead head 
to lead head, inclusive.  For many applications this behaviour is not
what is wanted and the \verb+-F+ option suppresses the final lead head.

This program is of particular use in command invocations in 
\verb+-Q+\oi{Q} or \verb+-R+\oi{R} options.
A example of this was given in \sref{efficiency} as an illustration
of the degree to which unnecessary command invocations slow \methsearch\ down.
Nevertheless, if \methsearch's built-in falseness or music analysis isn't 
sufficiently powerful, this can be a useful technique.


\subsection{Printing a touch — \texttt{gsiril}}\label{gsirilrow}

\ttcmdidx{gsiril} is a tool for proving and printing compositions.  
It is designed to be familiar to anyone who has used other peal
proving software such as MicroSiril\index{MicroSiril} or 
Sirilic\index{Sirilic}.  It also offers a significant
number of additional features.

The composition must be specified in the \texttt{gsiril} language.
It is beyond a scope of this manual to document \texttt{gsiril}.%
\footnote{Currently the most detailed source of documentation for 
\texttt{gsiril}, and in particular its language, can be found at 
\url{http://ex-parrot.com/~richard/gsiril/}.  In the future, a manual
similar to this will probably be produced.}
By default this is read from standard input, however the \verb+-f+ option
can be used to specify a file to read the composition from, or the \verb+-e+ 
option can be used to specify the composition on the command line.

If the \verb+-E+ option is passed to \texttt{gsiril}, then instead of 
proving the composition, the rows in it are printed without being proved.
This row stream can then be processed by other programs.  For example,
Charles Middleton's\index{Middleton, Charles!5600 Cambridge S.\ Major} 
5600 Cambridge Surprise Major can be defined as 
follows:
\begin{Verbatim}
8 bells

m ?= &-3-4-25-36-4-5-6-7
p  = m, +2
b  = m, +4

M  = 2p,b,4p
MW = 2p,2b,3p
WH = 3p,b,2p,b
H  = 6p,b

prove 5(M,MW,WH,H,H)
\end{Verbatim}
If this is stored in a file named \texttt{middletons.gsir}, then the 
following command will count the CRUs in it:
\begin{Verbatim}
gsiril -E -f middletons.gsir | musgrep -b8 -c "<CRUs>"
\end{Verbatim}

It is sometimes useful to invoke this from within \methsearch\ --- for 
example, to see which method is most musical for a given composition.  As
it stands, the file contains the place notation of
Cambridge hard-coded into it.  This can easily be resolved by removing
the definition of \verb+m+ from the file and putting it on the command line
instead.  This allows it to be substituted by \verb+$Q+\fspec{Q} in a
command invocation, such as:\index{example!major, music to Middleton's}
\begin{Verbatim}[commandchars=@~~]
-R'$q  $(gsiril -b8 -Dm="$Q" -E -f middletons.gsir @hfill\
            | musgrep -b8 -c "<CRUs>")'
\end{Verbatim}

It is important to note, however, that this does not ensure that the 
composition is true --- it just analyses the music until it runs false or 
comes round.\footnote{When run with \verb+-E+, \texttt{gsiril} gives no visual
indication of whether the composition is true or false: it just stops printing
rows if it runs false.  The exit status of \texttt{gsiril} will indicate 
whether or not the touch was true, but in POSIX-compliant shells (which most
Unix shells are), the exit status of a pipeline\index{pipe!exit status of} is
that of its last command.  In the \verb+bash+ shell\index{bash@\texttt{bash}} 
this can be changed by preceding the command with \verb+set -o pipefail;+ which
causes the pipeline to exit unsuccessfully if \texttt{gsiril} fails.  This
can then be tested with \verb+$?+, say by doing \verb+$[$? && suppress]+.%
\index{suppress keyword@{\texttt{suppress} keyword}}
But such circumlocutions are best avoided where possible.}
For details on how to use \texttt{gsiril} to prove compositions, see 
\sref{gsirilmeth}.

It's often possible to rewrite examples like this so that instead of using
a per-method command invocation it only invokes \texttt{gsiril} once when 
\methsearch\ starts, and uses \methsearch's internal music testing abilities.
In this case, we would use \texttt{gsiril} to generate a list of lead-heads
in Middleton's and pass that to a \verb+-Mlead+ option.  First, we modify
the definition of \verb+p+ and \verb+b+ in \texttt{middletons.gsir}:
\begin{Verbatim}
p  = "@", m, +2
b  = "@", m, +4
\end{Verbatim}
This makes \texttt{gsiril} print each lead head, which can then be passed
to a \verb+-Mlead+ option.  If lots of methods are being tested, this will
speed the search up significantly.
\begin{Verbatim}[commandchars=@~~]
-Mlead='$(gsiril -qf middletons.gsir)' -M'<CRUs>' -R'$q $M'
\end{Verbatim}

\subsection{Analysing music — \texttt{musgrep}}\label{musgrep}
\index{music|(}

The \ttcmdidx{musgrep} program reads a row stream and analyses the
music in it.  The following options are of relevance:

\begin{tabularx}{\textwidth}{llX}
\texttt{-b}&\texttt{--bells=N}&The total number of bells\\
\texttt{-p}&\texttt{--positive}
   &Print the number of rows with a positive score\\
\texttt{-n}&\texttt{--negative}
  &Print the number of rows with a negative score\\
\texttt{-c}&\texttt{--count}&Print the number of matching rows\\
\texttt{-s}&\texttt{--score}&Print the score for matching rows\\
\texttt{-i}&\texttt{--in-course}&Match only in-course rows\\
\texttt{-o}&\texttt{--out-of-course}&Match only out-of-course rows\\
\end{tabularx}

In addition, the command line should contain one or more music pattern.
These may be basic patterns (\sref{muspattern}), named patterns 
(\sref{musnamed}) or customised music scores (\sref{musscore}). 
The syntax for these is as given in chapter \ref{music},
except without the leading \verb+-M+.  When giving a negative score to 
a pattern the syntax (e.g.\ \verb+-1:*28+ to negatively score \verb+28+s
at the back) looks like the standard form of an option because of the 
leading minus sign.  This is resolved by preceding the music with
the special option \verb+--+.  This tells \texttt{musgrep} to stop looking
for options and to parse everything after it on the command line as an 
argument.\index{options!ending}  This syntax is standard to most command
line tools.

If no \verb+-p+, \verb+-n+, \verb+-c+ or \verb+-s+ is supplied, 
\texttt{musgrep} prints out all of the matching rows; 
if any of these are supplied, only the request count or counts are printed 
on a single line separated by tabs in the order 
\verb+-p+, \verb+-n+, \verb+-c+, \verb+s+.

The score, \verb+-s+, is the total number of points assigned to the piece
of ringing as described in \sref{music}.  The count, \verb+-c+, is the 
number of matching rows.  The two can be different because a row can match
more than one pattern (e.g.\ \verb+42135678+ is both a CRU and a 4-run), 
because a pattern can be assigned a score other than one 
(e.g.\ with \verb+2:*5678+ to score \verb+5678+s at the back double),
or because a pattern can match multiple times (e.g.\ \verb+21345678+ matched
the 4-run pattern three times).

It is also possible for the positive and negative counts, 
\verb+-p+ and \verb+-n+, not to sum to count, \verb+-c+.  This can happen
if a row matches a pattern but has zero net score.  

The \verb+-i+ and \verb+-o+ options can be used to filter the input stream
to only in-course or out-of-course rows, respectively.  This is applied before
any pattern matching is applied.  Even if \texttt{musgrep} is only being 
used to filter a row stream based on parity, a pattern must still be supplied:%
\footnote{Using \verb+?+ as a catch-all pattern can be easier than \verb+*+
as most shells do not require \verb+?+ to be quoted, whereas \verb+*+ generally
does need quoting.}
\begin{Verbatim}[commandchars=@~~]
musgrep -b8 -i ?
\end{Verbatim}

\index{music|)}
\index{row stream|)}

\section{Method streams}\label{methstream}\index{method stream|(}

Method streams are used to stream a list of methods between programs.  
The standard syntax is again very straightforward: methods are given,
one per line, in any recognised form of place notation (\sref{pn}).  
Whitespace is not permitted in the place notation, and any text after
the first whitespace character on the line is ignored.\footnote{%
\methsearch\ considers the space, tab, carriage return and line feed
to be whitespace.}\index{whitespace}
When invoked with the \verb+-I+ option (\sref{misc_opt}), 
\methsearch\ requires a method stream on standard input.\index{standard input}
Frequently, \methsearch's output is a method stream — specifically
whenever a \verb+-R+ option starting with \verb+$p+ or \verb+$q+ is given
(or no \verb+-R+ is given) and no statistical output or counts are requested.%
\footnote{A \verb+-u+ option is permitted as the status is written 
to standard error rather than standard output.%
\index{standard output}\index{standard error}}

The \verb+touchsearch+ (\sref{touchsearch}) 
and \verb+gsiril+ (\sref{gsirilmeth}) programs, 
as well as \methsearch\ itself (\sref{misc_opt}),
can be used as a filter on methods.  
In each case, this behaviour is activated with the program's \verb+--filter+ 
option (which for \methsearch\ itself, can be abbreviated to \verb+-I+).\oi{I}  
The programs all read a method stream from standard input,
test whether each method has the desired properties, and if so, writes
the method to its output method stream.

\subsection{Searching for touches — \texttt{touchsearch}}\label{touchsearch}

The \ttcmdidx{touchsearch} program is a simple program to searching for
touches in a single method.  It accepts the following options.

\begin{tabularx}{\textwidth}{llX}
\texttt{-b}&\texttt{--bells=N}&The default number of bells.\\
\texttt{-C}&\texttt{--call=CHANGE}&Specify a call\\
\texttt{-r}&\texttt{--ignore-rotations}
  &Filter out touches only differing by a rotation\\
\texttt{-l}&\texttt{--length=MIN-MAX}
  &Length or range of lengths required\\
\texttt{-P}&\texttt{--part-end=ROW}&Specify a part end\\
\texttt{-m}&\texttt{--mutually-true-parts}
  &Only require mutually true parts without requiring them to be joined\\
\texttt{-q}&\texttt{--quiet}&Don't output the touches\\
\texttt{-c}&\texttt{--count}&Count the number of touches found\\
           &\texttt{--raw-count}&
  Count the number of touches found, and print it without surrounding text\\
           &\texttt{--limit=N}&
  Limit the search to the first \texttt{N} touches\\
           &\texttt{--filter}&Run as a filter on a method library\\
\end{tabularx}

In its most usual operation, \texttt{touchsearch} is invoked with the place
notation for a single as a command line argument, and 
(unless \verb+-q+ is given) it outputs all the compositions it finds for 
that method.   By using \verb+-q --raw-count --limit=1+ together, this
behaviour can be adapted to simply check whether any compositions exist.
An example of this is given in \sref{cmdinv}.

However it is \texttt{touchsearch}'s \verb+--filter+ option that is of 
interest here.  This allows a single invocation of \texttt{touchsearch} to
search a series of methods for one that can produce a touch of the specified
type.  For example, the following command searches for asymmetric plain minor
methods that can actually produce an extent.%
\index{example!minor, asymmetric extents}
\begin{Verbatim}[commandchars=@~~]
methsearch -b6 -Fe -Q'$y eq ""' -rf -m'*2' @hfill\
  | touchsearch -b6 --filter -r -l720 -Cb=4 -Cs=234 --limit=1
\end{Verbatim}

The \verb+-Cb=4+ and \verb+-Cs=234+ options are the two import ones.  
They define lead end calls using their place notation 
and provide them with names, `\verb+b+' and `\verb+s+' respectively.  
The names are optional, and \verb+-Cb=4+ can be simply written \verb+C4+.  
However, if no name is supplied and the touch is to be printed out, 
the resulting output can be confusing.

The \verb+-r+ and \verb+--limit=1+ options don't actually change the 
result of the search but do speed it up: in the first case by rotationally
pruning the search space, and in the latter, by stopping once one touch 
has been found.  (A future version of \texttt{touchsearch} will probably
handle this optimisation automatically.)
The \verb+-l720+ says that a touch of exactly 720 changes is required.  
Ranges are also permitted in the \verb+-l+ option's argument: e.g.
\verb+-l540-720+ requires any touch of between 540 and 720 changes, inclusive.

The \verb+-P+ option looks for multi-part compositions.  For 
example, \verb+-P1342+ looks for a three-part composition with 2,3,4 cycling.
Note that, like \methsearch's \verb+-FP+ option (\sref{pends}), this option
specifies a part end group 
— the specified part end does not have to occur as the first part end.
In this case, a composition bringing up 
the first part end \verb+142356+ would be acceptable.
If \verb+-m+ is 
additionally specified, no effort is made to join the parts as long
as a set of mutually true parts exist.

\subsection{Proving a touch — \texttt{gsiril}}\label{gsirilmeth}

\ttcmdidx{gsiril} has already been mentioned in \sref{gsirilrow}.  As
mentioned there, it is also possible to use \texttt{gsiril} as a filter
to discover which methods are true to a particular composition.   This
is achieved by using the \verb+--filter+ option to \texttt{gsiril}.

In this mode, \texttt{gsiril} reads methods from its input stream
and sets the \verb+m+ and \verb+lh+ variables%
\footnote{The names of these symbols can be overridden with             
\verb+--lead-symbol+ and \verb+--lh-symbol+ options, respectively.}
to the method place notation excluding the lead-end change 
(i.e.\ \methsearch's \verb+$Q+),  and the lead-end change respectively.
A composition file in the \texttt{gsiril} language should be specified with 
the \verb+-f+ option, and the number of bells specified with \verb+-b+.  
(The latter requirement is likely to be removed in a future version of 
\texttt{gsiril}.)

For example, to find right place methods true to Middleton's 5600 of 
Cambridge Surprise Major,\index{Middleton, Charles!5600 Cambridge S.\ Major}
the following invocation could be used:%
\index{example!major, music to Middleton's}
\begin{Verbatim}[commandchars=@~~]
methsearch -b8 -SG1 -p2 -l2 -srfw @hfill\
  | gsiril -b8 --filter -f middletons.gsir
\end{Verbatim}

In this example, the file \verb+middletons.gsir+ needs to slightly different
from the version used in the example in \sref{gsirilrow}.  First, it needs
to accommodate arbitrary seconds place leads, and secondly, it shouldn't 
define the place notation of the method (as this is passed via the \verb+m+
symbol).  The following \texttt{gsiril} code will do the job.
\begin{Verbatim}
8 bells

m ?= &-3-4-25-36-4-5-6-7
lh ?= +12

p = m,lh,"  @"
b = m,+4,"- @"

H = repeat( {b, /*8/:   b, break; p} )
W = repeat( {b, /*8?/:  b, break; p} )
M = repeat( {b, /*8??/: b, break; p} )

finish = repeat({rounds: break}, p)

prove 5(2M,2W,3H)
\end{Verbatim}

There are several things to note in this example.  First, \verb+m+ and
\verb+lh+ are given conditional definitions via the \verb+?=+ operator.
This means that the definitions from the file are only used if \texttt{gsiril}
does not provide a definition.  This means that the file can be tested
with the command:

\begin{Verbatim}
gsiril -f middletons.gsir
\end{Verbatim}

Secondly, more complex definitions of the 
middle\index{middle|see{calling position}}, 
wrong\index{wrong|see{calling position}} 
and home\index{home|see{calling position}} 
calling positions\index{calling position} are needed
to work with arbitrary lead ends including 8ths
place methods.\footnote{A full explanation of how the these definitions work
is beyond the scope of this document.  The \texttt{gsiril} 
documentation found at \url{http://ex-parrot.com/~richard/gsiril/} contains 
a similar example in the section entitled `Rule-based touches'.}

\index{method streams|)}

%\chapter{Some longer examples}
%
%\section{Replacing Stonebow}
%
%David Hull's peal of 23 spliced surprise major is a moderately 
%popular composition.\index{Hull, David!23 spliced} It is given below.
%
%\begin{center}\scriptsize\begin{tabular}{ccl}
%\texttt{ }&\texttt{2345678}&Rutland\\
%\texttt{ }&\texttt{4263857}&Coney Street\\
%\texttt{ }&\texttt{6482735}&Weatheroak\\
%\texttt{-}&\texttt{3578264}&Yorkshire\\
%\texttt{ }&\texttt{8654327}&Ipswich\\
%\texttt{ }&\texttt{7325486}&Jersey\\
%\texttt{-}&\texttt{3586742}&Superlative\\
%\texttt{-}&\texttt{5642378}&Glasgow\\
%\texttt{ }&\texttt{6257483}&London\\
%\texttt{-}&\texttt{4562378}&Bristol\\
%\texttt{-}&\texttt{6452378}&Wembley\\
%\texttt{ }&\texttt{5634827}&Stonebow\\
%\texttt{ }&\texttt{8375264}&Brislington\\
%\texttt{-}&\texttt{2783456}&Cassiobury\\
%\texttt{-}&\texttt{3564782}&Double Dublin\\
%\texttt{-}&\texttt{6354782}&Little Shambles\\
%\texttt{-}&\texttt{5634782}&Cambridge\\
%\texttt{ }&\texttt{4862573}&Ashtead\\
%\texttt{ }&\texttt{3576248}&Ealing\\
%\texttt{ }&\texttt{7325864}&Peterborough\\
%\texttt{-}&\texttt{8273456}&Lincolnshire\\
%\texttt{ }&\texttt{3526847}&Lindum\\
%\texttt{ }&\texttt{2385764}&Falmouth\\\hline
%\texttt{-}&\texttt{7823456}&\\
%\end{tabular}\end{center}
%
%However one method in particular --- Stonebow --- is quite unpopular with 
%ringers because of its unmemorable line.

\chapter{History of \methsearch}

\section{Early development}

I have no precise record of when I began writing \methsearch\ though it 
must have been sometime in the first quarter of 2002.  It was around
this time that interest was developing in cyclic methods, largely
driven by Philip Earis\index{Earis, Philip}, and I found 
myself writing C++ code to explore the space of cyclic methods in search 
of methods that we might want to ring.  The earliest reference I have 
found to \methsearch\ is from an email exchange with Philip on 5 April 2002
when I enumerated (incompletely as it turned out) rotationally-symmetric
cyclic plain major methods.  

The first version of \methsearch, or \texttt{meth-search} as it was then spelt, 
that I made available for anyone else was on 22 May 2002, and even at this 
stage it already had a surprising amount of functionality.  \methsearch\ 
releases have always been somewhat \textit{ad hoc} and have not had 
meaningful release numbers.  The following entries are dates when I 
advertised new Windows binaries and produced an accompanying list of new
features.  For the first two months this happened very frequently before
slowing to a more steady rate.  The earlier releases also involved a fair
amount of bug fixing that is not mentioned below.

\subsection{22 May 2002}

In the first public version of \methsearch,
the \verb+-b+, \verb+-G+, \verb+-S+, \verb+-p+, \verb+-l+, \verb+-f+, 
\verb+-r+, \verb+-c+, \verb+-e+, \verb+A+, \verb+-s+, \verb+-k+, \verb+-d+, 
\verb+-q+, \verb+-L+, \verb+-H+ and \verb+-P+ options all existed in 
much their present form.
The \verb+-w+ option supported right-place methods on even stages, and
the current \verb+-R+ functionality existed, though the option was called
\verb+-F+. 

At this time, by default \methsearch\ would only look for methods with
no more than three places made at the lead-end or lead-head.  A \verb+-h+
option disabled this behaviour and allowed arbitrary places.  
\methsearch's handling of falseness was still ill-defined, though probably
behaved much like the current \verb+-Fn+ option.  There was no facility 
for analysing music.

Format strings were supported for both the \verb+-F+ (now \verb+-R+) and 
\verb+-H+ options and could only contain literal text, variables
and a limited number of character escape sequences (namely \verb+\t+ and 
\verb+\n+).
Variables were written with a \verb+%+ rather than a \verb+$+ — the former
is still supported but is deprecated, and only the following were available:
\verb+%l+, \verb+%p+, \verb+%n+, \verb+%N+, \verb+%C+, \verb+%S+, \verb+%%+ 
and (probably) \verb+%c+ and \verb+%b+.

\subsection{27 May 2002}
\begin{list}{\labelitemi}{\leftmargin=1em\itemsep=0em}
\item There's now a \verb+-j+ option which prohibits adjacent places 
within the lead.
\item Similarly to \verb+-S+, \verb+-T+ searches for treble bob methods.
\item It now has a status option (\verb+-u+) which keeps a display of the 
the place notation currently being considered.
\item The \verb+-F+ option has been renamed \verb+-R+ so that I can use 
\verb+-F+ for falseness.
\item The new \verb+-F+ options (\verb+-Fx+, \verb+-Fn+, \verb+-Fh+, 
\verb+-Fl+, \verb+-Fc+) can be used to filter out methods that are 
false with the lead or plain course.  The default is \verb+-Fh+.
\item It recognises \verb+$+ as an alternative to \verb+%+ in format strings.
\end{list}

\subsection{5 June 2002}
\begin{list}{\labelitemi}{\leftmargin=1em\itemsep=0em}
\item New \verb+%+$n$\verb+h+ and (probably) \verb+%+$n$\verb+r+ variables 
to print the $n$th change and row.
\item The \verb+--limit+ option aborts the search after a specified number
have been found.
\item There's now the start of a \verb+-M+ option for musical analysis.  Its 
argument is a musical pattern, e.g.\ \verb+-M*1234*+.  The \verb+%M+ variable
prints the musical score.
\end{list}

\subsection{19 June 2002}
\begin{list}{\labelitemi}{\leftmargin=1em\itemsep=0em}
\item Multiple hunt bells supported with \verb+-U+.
\item Right-place methods (\verb+-w+) now supported on odd stages too.
\item Testing for extent viability with \verb+-Fe+.
\item Counting the total number of methods with \verb+--count+.
\item Music analysis extended to handle named music including \verb+<4-runs>+.
\end{list}

\subsection{21 June 2002}
\begin{list}{\labelitemi}{\leftmargin=1em\itemsep=0em}
\item Testing for clear proof scale falseness: \verb+-FCPS+.
\item Preventing `U' falseness\index{U falseness} 
\verb+-FU+,\oidx{FU}{\texttt{FU}} 
generalised to requiring a round block of homes\index{home} 
(specified using \verb+--bob+\loid{bob}) to be true.%
\footnote{This functionality has since been removed.}
\item Added \verb+%q+ for more concise place notation.
\end{list}

\subsection{2 September 2002}
\begin{list}{\labelitemi}{\leftmargin=1em\itemsep=0em}
\item Added \verb+-m+ to partially specify the place notation.
\item Generalised \verb+-M<4-runs>+ to \verb+-M<+$n$\verb+-runs>+.
\item Added \verb+--start-at+.
\item Added \verb+--hl-change+\loid{hl-change} and 
\verb+--le-change+\loid{le-change}.%
\footnote{Both options have been since removed.}
\item Added \verb+--cyclic-hlh+, \verb+--cyclic-hle+ and their 
\verb+--rev-+\ldots{} variants.
\item Support for MicroSiril libraries.
\end{list}

\section{In the Ringing Class Library}

On 28 November 2002, the code was added to the Ringing Class Library%
\index{Ringing Class Library} CVS\index{CVS} repository, and, at the
same time, renamed \verb+methsearch+ (without a hyphen).  This is the 
earliest code snapshot that is still available.  From this point onwards,
there is a complete version history of \methsearch.

\subsection{5 December 2002}
\begin{list}{\labelitemi}{\leftmargin=1em\itemsep=0em}
\item Program renamed from \texttt{meth-search} to \methsearch.
\item Code added to the Ringing Class Library CVS repository.
\item Support for principles with \verb+-U0+ and \verb+-n+.
\item Changed \verb+-j+ to take an optional argument.
\item Added the variables \verb+%F+ (only in-course tenors-together codes
for regular palindromic major methods and cyclic rotationally-symmetric
major methods) and \verb+%o+.
\end{list}

\subsection{10 February 2003}
\begin{list}{\labelitemi}{\leftmargin=1em\itemsep=0em}
\item Added \verb+-M<front-+$n$\verb+-runs>+ and 
\verb+-M<back-+$n$\verb+-runs>+.
\item Added \verb+--offset-cyclic+.
\end{list}

\subsection{13 May 2003}
\begin{list}{\labelitemi}{\leftmargin=1em\itemsep=0em}
\item Added \verb+--regular-hls+.
\item Generalised \verb+%F+ to regular palindromic methods on 
any even stage greater than six.
\item Removed \verb+%F+ support for cyclic rotationally-symmetric
major methods.
\item Added \verb+%F+ support for out-of-course falseness groups and `A' 
falseness.
\item Allow expressions in format strings using \verb+$[+\ldots\verb+]+.
\item Added the \verb+suppress+ and \verb+abort+ keywords.
\item Improved masks: alternative blocks (e.g.\ \verb+(36|34)+) and 
over/underworks (e.g.\ \verb+&34.1.5.1.5/*+).
\item Definition of \verb+-FCPS+ changed on six bells.
\item Removed \verb+--le-change+ and \verb+--hl-change+.
\item Removed \verb+-FU+ and \verb+--bob+.
\item Added \verb+--require+.
\end{list}

\subsection{14 May 2003}
\begin{list}{\labelitemi}{\leftmargin=1em\itemsep=0em}
\item Allow string literals in expressions to be quoted with 
\verb+'+\ldots\verb+'+.
\item Added \verb+$d+ for the lead head code.
\end{list}

\subsection{23 April 2004}
\begin{list}{\labelitemi}{\leftmargin=1em\itemsep=0em}
\item Removed the \verb+-h+ option.
\item Added \verb+-E+, \verb+-o+, \verb+-O+ and \verb+--raw-count+.
\item Added support for XML output.
\item The arguments to \verb+-l+ and \verb+-p+ are now optional — 
the default is 2.
\item The \verb+-j+ option now applies to the lead head and lead end.
\item Added \verb+$Q+, \verb+$u+, \verb+$y+, \verb+$O+ and \verb!-Fe+!.
\item Music patterns may now take a score.
\item Expressions may now contain the \verb+.+ operator.
\item Support for command invocations \verb+$(+\ldots\verb+)+.
\end{list}

Relatively little development happened during the next few years 
and it wasn't until 2009 that I next produced a release of \methsearch.

\subsection{15 June 2009}
\begin{list}{\labelitemi}{\leftmargin=1em\itemsep=0em}
\item Added \verb+-Mcourse+ and \verb+-Mlead+.
\item Added \verb+-Fr+ and support for row expressions, including
file inputs and command invocations (though not set literals or group 
generator literals).
\item Falseness groups can now be found with \verb+-F:+\textit{groups}.
\item The \verb+-C+ option is now accepted as shorthand for \verb+--count+.
\item Added \verb+--node-count+, \verb+--prefix+, \verb+--filter+.
\item Multiple \verb+--require+ options can be given.
\end{list}

\subsection{18 September 2009}
\begin{list}{\labelitemi}{\leftmargin=1em\itemsep=0em}
\item Added \verb+-Mhalflead+, \verb+-M2halflead+, \verb+-Mrhalflead+
and \verb+-M2rhalflead+.
\end{list}

\subsection{30 April 2010}
\begin{list}{\labelitemi}{\leftmargin=1em\itemsep=0em}
\item Wrote the first version of this manual.
\item Little method: \verb+-Z+ (\sref{stage_class}).
\item Symmetric sections: \verb+-y+ (\sref{int_struct}).
\item Mirror symmetry: \verb+--mirror+ (\sref{symmetry}).
\item Improved falseness: \verb+-Fs+ (\sref{avoidrow}), 
  \verb+-FP+ (\sref{pends}).
\item Library filtering: \verb+--filter-lib+ (\sref{misc_opt}).
\item Random ordering: \verb+--random+, \verb+--random-count+ [removed], 
  \verb+--seed+ (\S\ref{random}).
\item Method staticity: \verb+$s+ (\sref{fmtstr}).
\item Support for three-bell methods.
\item Set and group literals: \verb+{+\ldots\verb+}+, \verb+<+\ldots\verb+>+
  (\sref{rowexprmulti}).
\end{list}

\section{Pending release}

\methsearch\ is getting close to the point where I shall build a release
branded version 1.0 and announce it publicly on appropriate mailing lists.
The following new features are now implemented ready for 1.0.  The main 
features that are still missing are support for alliance methods%
\index{alliance methods} and improved support for 
principles\index{principles} and differentials.\index{differentials}

\begin{list}{\labelitemi}{\leftmargin=1em\itemsep=0em}
\item Method identifiers: \verb+$i+ (\sref{variables}).
\item Filter payloads: \verb+$a+ (\sref{misc_opt} and \sref{variables}). 
\item Filter inverting: \verb+--invert-filter+ (\sref{misc_opt}).
\item Delight methods: \verb+--delight+, and (for minor)
\verb+--3rds-place-delight+ and \verb+--4ths-place-delight+ 
(\sref{stage_class}).
\item Historical classes: \verb+--strict-delight+, \verb+--exercise+, 
\verb+--strict-exercise+, \verb+--pas-alla-tria+ and \verb+--pas-alla-tessera+
(\sref{hist_class}).
\item Fourths-place doubles and minor lead-end symbols:
  \verb+$D+ (\sref{variables}).
\item Conjugation and set difference in row expressions: 
  \verb+^+ (\sref{rowexpr}) and \verb+-+ (\sref{rowexprmulti}).
\item Music pattern matching in expressions: \verb+~~+ (\sref{oper}).
\item Comma operator (for sequencing) in expressions: \verb+,+ (\sref{oper}).
\item Abbreviations for the \verb+--require+ and \verb+--filter+ options:
  \verb+-Q+ and \verb+-I+, respectively (\sref{misc_opt}). 
\item Command exit status: \verb+$?+ (\sref{exitstatus}).
\item Number of bells: \verb+$B+ (\sref{variables}).
\item Response files: \verb+@+\textit{filename} (\sref{respfile}).
\item Customising bell symbols: \verb+$BELL_SYMBOLS+ environment variable 
  (\sref{pn}).
\item Restricting or prohibiting changes: \verb+--changes+ (\sref{pn}).
\item Support for Unicode output in UTF-8: \verb+-Outf8+ (\sref{output_opt}).
\item Status update frequencies: \verb+--status-freq+ (\sref{output_opt}).
\item Allowing exceptions to \verb+-p+$n$ across the lead-end: 
  \verb+--long-le-place+ (\sref{int_struct}).
\item Improved symmetry handling for principles: 
  \verb+--floating-sym+ (\sref{symmetry}).
\item Improved random sampling: \verb+--loop+, \verb+--timeout+ 
  (\sref{random}); \verb+--random-count+'s functionality subsumed in
  \verb+--limit+ and \verb+--loop+ and the option removed.
\item The current time: \verb+$T+ (\sref{variables}).
\end{list}

\clearpage
\phantomsection
\footnotesize
\printindex[default][\normalsize 
\addcontentsline{toc}{chapter}{Index}
Underlined page references refer to the main definition of that option
or variable.  Subjects are not generally indexed when they appear
in the discussion of a similarly-named option.  For example, `surprise 
methods' do not appear in the index because they are only discussed in 
conjunction with the \texttt{--surprise} option.
]

\normalsize
\chapter*{Colophon}
\addcontentsline{toc}{chapter}{Colophon}

This manual was typeset using \XeTeX\index{Xetex@\XeTeX}, a document 
typesetting package written by Jonathan Kew.  It is derived from 
\LaTeX, written by Leslie Lamport, but includes support for the Unicode 
character set and OpenType fonts.
Both are derived from Donald Knuth's original \TeX\ typesetting engine.  
These are all open source products and are freely available on a variety of 
operating systems, include GNU/Linux. 

The main font used is Linux Libertine, an open source OpenType font designed
by Philipp Poll.  The sans serif font used in chapter and section headings
is Linux Biolinum, also designed by Philipp Poll.  Despite their names,
neither font is specific to Linux.  Equations are set using Donald Knuth's 
Computer Modern font.  The monospaced font used for code fragments and
row literals is the Computer Modern Teletype font, again by Donald Knuth.%
\index{fonts}

All of the software described in this manual (except some passing
references to a certain product gestated in Redmont) are themselves open
source software.  All of the ringing-related programs are available as 
part of the Ringing Class Library project,\index{Ringing Class Library}
and \methsearch\ uses it for manipulation rows, changes and methods, 
reading method libraries, analysing music, proof and much more.  
The Ringing Class Library was started by Martin Bright and contains
contributions from Richard Smith and Mark Banner.

The source code for this manual is available from the Ringing Class Library
CVS repository and is itself open source.  Any additions or corrections 
to the manual will be gratefully received.\index{open source}

\vfil
\footnotesize
Permission is granted to copy, distribute and/or modify this
document under the terms of the GNU Free Documentation License,
Version 1.3 or any later version published by the Free Software
Foundation; with no Invariant Sections, no Front-Cover Texts and
no Back-Cover Texts.\index{licence}


\end{document}
